%%% file `xehangulcweb.tex`
%%
%% (C) Copyright 2007 Dohyun Kim <nomos@ktug.or.kr>
%%                    Soojin Nam <sjnam@ktug.or.kr>
%%
%% This work may be distributed and/or modified under the
%% conditions of the LaTeX Project Public License, either version 1.3c
%% of this license or (at your option) any later version.
%% The latest version of this license is in
%%  http://www.latex-project.org/lppl.txt
%% and version 1.3c or later is part of all distributions of LaTeX
%% version 2006/05/20 or later.
%%

\input xetexko.sty

\ifpdftex
  \ifdefined\pdfgentounicode
    \input glyphtounicode
    \pdfgentounicode=1
  \fi
\fi

\baselineskip 15pt

% 기본 폰트를 한글을 포함하도록 다시 정의합니다.
\hangulfont="HCR Batang LVT:mapping=tex-text" at 10pt
\def\hfontsize#1{\ifx\empty#1\empty\else
 \edef\temp{\noexpand\hfontsizex\fontname\hangfnt\space\noexpand\nil}%
 \temp{#1}\fi}
\def\hfontsizex"#1" #2\nil#3{\font\hangfnt"#1" at #3}
\def\hfontname#1{\ifx\empty#1\empty\else
 \edef\temp{\noexpand\hfontnamex\fontname\hangfnt\space\noexpand\nil}%
 \temp{#1}\fi}
\def\hfontnamex"#1" #2\nil#3{\font\hangfnt"#3:mapping=tex-text" #2}
\def\hfont#1#2{\hfontname{#1}\hfontsize{#2}}

% 기본 글꼴 설정
\hfont{HCR Batang LVT}{10pt}
\expandafter\def\expandafter\tt\expandafter{\tt\hfontname{HCR Dotum LVT}}
\expandafter\def\expandafter\sl\expandafter{\sl\hfontname{HCR Dotum LVT}}
\expandafter\def\expandafter\it\expandafter{\it\hfontname{HCR Dotum LVT}}
\expandafter\def\expandafter\bf\expandafter{\bf\hfont{HCR Batang LVT ExtraBold}{10pt}}

% 각종 폰트를 한글을 포함하도록 재정의
\let\otitlefont=\titlefont
\def\titlefont{\hfont{HCR Batang LVT Bold}{13pt}\otitlefont}
\let\ottitlefont=\ttitlefont
\def\ttitlefont{\hfont{HCR Dotum LVT Bold}{13pt}\ottitlefont}
\let\orgninerm=\ninerm
\def\ninerm{\hfont{HCR Batang LVT}{9pt}\orgninerm}
\let\orgeightrm=\eightrm
\def\eightrm{\hfont{HCR Batang LVT}{8pt}\orgeightrm}
\let\orgsc=\sc
\def\sc{\hfont{HCR Batang LVT}{8pt}\orgsc}
\let\orgtentex=\tentex
\def\tentex{\hfont{HCR Dotum LVT}{10pt}\orgtentex}

\expandafter\edef\csname AT(@):catcode\endcsname
  {\noexpand\catcode`@\the\catcode`@}
\catcode`\@=11

% names of the sections
\def\fin{\par\vfill\eject % this is done when we are ending the index
  \ifpagesaved\null\vfill\eject\fi % output a null index column
  \if L\lr\else\null\vfill\eject\fi % finish the current page
  \parfillskip 0pt plus 1fil
  \def\grouptitle{마디 목차}
  \let\topsecno=\nullsec
  \message{Section names:}
  \output={\normaloutput\page\lheader\rheader}
  \setpage
  \def\note##1##2.{\quad{\eightrm\ifacro{\pdfnote##2.}\else{##2}\fi##1.}}
  \def\Q{\note{번 마디에서 인용된다}} \let\Qs=\Q
  \def\U{\note{번 마디에서 사용된다}} \let\Us=\U
  \def\I{\par\hangindent 2em}\let\*=*
  \ifacro \def\outsecname{\bschar376\bschar377\bschar271\bschar310\bschar265\bschar024\bschar000 \bschar272\bschar251\bschar314\bschar050}% 마디 목차
    \let\Xpdf\X
  \ifpdftex
    \edef\0{\string\0}%
    \edef\1{\string\1}%
    \edef\2{\string\2}%
    \edef\3{\string\3}%
    \makebookmarks \pdfdest name {NOS} fitb
    \pdfoutline goto name {NOS} count -\secno {\outsecname}
    \def\X##1:##2\X{\Xpdf##1:##2\X \firstsecno##1.%
      {\toksF={}\makeoutlinetoks##2\outlinedone\outlinedone}%
      \pdfoutline goto num \the\toksA \expandafter{\the\toksE}}
  \else\ifpdf
    \special{pdf: outline -1 << /Title (\outsecname)
      /Dest [ @thispage /FitH @ypos ] >>}
    \def\X##1:##2\X{\Xpdf##1:##2\X \firstsecno##1.%
      {\toksF={}\makeoutlinetoks##2\outlinedone\outlinedone}%
      \special{pdf: outline 0 << /Title (\the\toksE)
        /A << /S /GoTo /D (\romannumeral\the\toksA) >> >>}}
  \fi\fi\fi
  \readsections}

% CWEB USER MANUAL의 부록C 14.에 해당하는 모든 매크로를 재정의
\def\inote#1#2#3.{\Y\noindent{\hangindent2em%
  \baselineskip10pt\eightrm#1~\ifacro{\pdfnote#3.}\else#3\fi#2.\par}}
\def\rnote#1#2.{\Y\noindent{\hangindent2em%
  \baselineskip10pt\eightrm\ifacro{\pdfnote#2.}\else#2\fi#1.\par}}
\def\A{\rnote{번 마디도 살펴보라}} \let\As=\A
\def\Q{\inote{이 코드는}{번 마디에서 인용된다}} \let\Qs=\Q
\def\U{\inote{이 코드는}{번 마디에서 사용된다}} \let\Us=\U
\def\ET{,~} \def\ETs{\ET}
\def\ATH{{\acrofalse\X\kern-.5em:전처리기 정의들\X}}
\def\ch{\note{Change 파일에서 변경된 마디들:}
  \let\*=\relax}
\def\today{\number\year 년 \number\month 월 \number\day 일}
\def\datethis{\def\startsection{\leftline{\sc\today\ \ \hours}\bigskip
  \let\startsection=\stsec\stsec}}
\def\datecontentspage{% changed in version 3.66
  \def\botofcontents{\vfill
   \centerline{\covernote}
   \bigskip
   \leftline{\sc\today\ \hours}}} % timestamps the contents page

% 단순한 한글화로 목차 재정의
\def\con{\par\vfill\eject % finish the section names
% \ifodd\pageno\else\titletrue\null\vfill\eject\fi % for duplex printers
  \rightskip 0pt \hyphenpenalty 50 \tolerance 200
  \setpage \output={\normaloutput\page\lheader\rheader}
  \titletrue % prepare to output the table of contents
  \pageno=\contentspagenumber
  \def\grouptitle{목차}
  \message{Table of contents:}
  \topofcontents \startpdf
%  \line{\hfil \tt 마디\hbox to3em{\hss 쪽}}
  \line{\hfil 마디\hbox to3em{\hss 쪽}}
  \let\ZZ=\contentsline
  \readcontents\relax % read the contents info
  \botofcontents \end} % print the contents page(s) and terminate

\ifacro\else\expandafter\endinput\fi % pdf mode에서만...
% ascii문자는 앞에 zero 붙여 ucs string으로 만든다.
\def\hanguladdFend{hanguladdFend}
\def\addF#1{\hanguladdF#1\hanguladdFend\tokprocessedtrue}
\def\hanguladdF#1{\ifx#1\hanguladdFend\else
    \ifx#1\string
      \expandafter\expandafter\expandafter\hanguladdFall
    \else
      \addtokens\toksF{\bschar000#1}%
      \expandafter\expandafter\expandafter\hanguladdF
    \fi
  \fi}
\def\hanguladdFall#1\hanguladdFend{\addtokens\toksF{\bschar000\string#1}}
\def\makeoutlinetoks{\Tnumfalse
  % BOM(0xfeff)을 outline string 맨 앞에 첨가한다.
  \addtokens\toksF{\bschar376\bschar377}%
  \afterassignment\makeolproctok\let\nxt= }
% 한글을 위해 \makeolproctokchar 재정의
\def\makeolproctokchar#1 #2 #3{\ucsoltooctal{`#3}}
\def\ucsoltooctal#1{%
  \count@\dimexpr.00006103515625\dimexpr#1sp\relax\relax
  \edef\ucsoltempa{\number\count@}%
  \@tempcnta\numexpr#1-'40000*\count@\relax
  \count@\dimexpr.00048828125\dimexpr\@tempcnta sp\relax\relax
  \edef\ucsoltempa{\ucsoltempa\number\count@}%
  \@tempcnta\numexpr\@tempcnta-'4000*\count@\relax
  \count@\dimexpr.00390625\dimexpr\@tempcnta sp\relax\relax
  \edef\ucsoltempa{\ucsoltempa\number\count@}%
  \@tempcnta\numexpr\@tempcnta-'400*\count@\relax
  \count@\dimexpr.015625\dimexpr\@tempcnta sp\relax\relax
  \edef\ucsoltempb{\number\count@}%
  \@tempcnta\numexpr\@tempcnta-'100*\count@\relax
  \count@\dimexpr.125\dimexpr\@tempcnta sp\relax\relax
  \edef\ucsoltempb{\ucsoltempb\number\count@}%
  \edef\ucsoltempb{\ucsoltempb\number\numexpr\@tempcnta-'10*\count@\relax}%
  \addtokens\toksF{\bschar\ucsoltempa\bschar\ucsoltempb}\tokprocessedtrue
}

\csname AT(@):catcode\endcsname

\endinput

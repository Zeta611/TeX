% This file generates the user manual; TeX it, don't read it!

\def\tangref{3} % where the main explanation of CTANGLing is given
\input cwebmac
\input kotexcweb
\acrofalse\pdffalse\pdftexfalse
\def\page{\box255 } \normalbottom
\parskip 0pt plus 1pt
\def\RA{\char'31 } % right arrow
\def\hang{\hangindent 4em\ignorespaces}
\font\eightrm=cmr8
\font\ninerm=cmr9
\font\ninett=cmtt9
\font\eighttt=cmtt8
\font\twelvett=cmtt12
\font\quoterm=cmssq8
\font\quoteit=cmssqi8
\font\authorfont=cmr12
\font\sectionfont=cmbx12

\def\pb{\.{|...|}}
\def\v{\.{\char'174}} % vertical (|) in typewriter font
\def\lpile{\def\cr{\hfill\endline}\matrix} % I only use \lpile by itself
\abovedisplayskip=.6\abovedisplayskip
\belowdisplayskip=.6\belowdisplayskip
\abovedisplayshortskip=.6\abovedisplayshortskip
\belowdisplayshortskip=.6\belowdisplayshortskip
\advance\baselineskip by -.5pt
\advance\pageheight by \baselineskip % the manual just got a bit longer
\advance\fullpageheight by \baselineskip
\setpage

\outer\def\section #1.{\penalty-500\bigskip
        \centerline{\sectionfont\hfont{nanummjb}{at 12pt}%
        \def\.##1{{\twelvett##1}} #1}\nobreak\vskip 6pt
        \everypar{\hskip-\parindent\everypar{}}}

\def\lheader{\mainfont\the\pageno\hfill\runninghead\hfill}
\def\rheader{\hfill\runninghead\hfill\mainfont\the\pageno}
\def\runninghead{{\tt CWEB} \hfont{nanummjm}{at 9pt}사용자 설명서 (버전 3.64)}

% This verbatim mode assumes that ! marks are !! in the text being copied.
\def\verbatim{\begingroup
  \def\do##1{\catcode`##1=12 } \dospecials
  \parskip 0pt \parindent 0pt \let\!=!
  \catcode`\ =13 \catcode`\^^M=13
  \tt\hfontname{nanumgtm} \catcode`\!=0 \verbatimdefs \verbatimgobble}
{\catcode`\^^M=13{\catcode`\ =13\gdef\verbatimdefs{\def^^M{\ \par}\let =\ }} %
  \gdef\verbatimgobble#1^^M{}}

\null\vfill
\baselineskip 12pt
\centerline{\titlefont The {\ttitlefont CWEB} System of
    Structured Documentation}
\vskip 18pt\centerline{(Version 3.64 --- February 2002)}
\vskip 24pt
\centerline{\authorfont Donald E. Knuth and Silvio Levy}
\vfill

\noindent
\TeX\ is a trademark of the American Mathematical Society.

\noindent
Acrobat Reader is a trademark of Adobe Systems Incorporated.

\bigskip\noindent
The printed form of this manual is copyright \copyright\ 1994
  by Addison-Wesley Publishing Company, Inc.  All rights reserved.
\smallskip\noindent
The electronic form is copyright \copyright\ 1987, 1990, 1993, 2000
  by Silvio Levy and Donald E. Knuth.

\bigskip\noindent
Permission is granted to make and distribute verbatim copies of the
electronic form of this document provided that the electronic copyright
notice and this permission notice are preserved on all copies.

\smallskip\noindent
Permission is granted to copy and distribute modified versions of the
electronic form of this document under the conditions for verbatim copying,
provided that the entire resulting derived work is distributed under the terms
of a permission notice identical to this one.

\smallskip\noindent
Individuals may make copies of the documentation from the electronic files
for their own personal use.

\smallskip\noindent
Internet page \.{http://www-cs-faculty.stanford.edu/\char`\~knuth/cweb.html}
contains current info about \.{CWEB} and related topics.

\pageno=-1 \titletrue\eject


\null\vfill
\baselineskip 15pt

\centerline{\titlefont 구조적 문서화의 {\ttitlefont CWEB} 시스템}
\vskip 18pt\centerline{(버전 3.64 --- 2002년 2월)}
\vskip 24pt
\centerline{\hfont{nanummjm}{at 11pt} 도널드 어빈 크누스와 실비오 레비}
\vfill

\noindent
\TeX은 미국 수학 협회의 상표이다.

\noindent
Acrobat Reader는 어도비 시스템즈사의 상표이다.

\bigskip\noindent
이 설명서의 인쇄본은 에디스-웨슬리 출판사에 의해 1994년 저작권으로 보호된다.

\smallskip\noindent
전자 양식은 실비오 레비와 도널드 어빈 크누스에 의해 1987, 1990, 1993, 2000녀 저작권으로
보호된다.

\bigskip\noindent
본 문서의 전자 형식의 사본을 제공하고 배포 할 수있는 권한이 있다. 단, 전자 저작권 고지 및
허가 고지가 모든 사본에 보존되어야한다.

\smallskip\noindent
이 문서의 전자 형식 수정 버전을 그대로 복사 조건으로 복사 및 배포 할 수 있다.
단, 결과 파생 저작물 전체가 이 문서와 동일한 허락 고지의 조건에 따라 배포되어야한다.

\smallskip\noindent
개인은 자신의 개인 용도로 전자 파일에서 문서의 사본을 만들 수 있다.

\smallskip\noindent
\.{CWEB}과 그와 관련된 최신 정보는 
\.{http://www-cs-faculty.stanford.edu/\char`\~knuth/cweb.html}에 있다.

\pageno=0 \titletrue\eject

\titletrue
\centerline{\titlefont 구조적 문서화의 {\ttitlefont CWEB} 시스템}

\vskip 15pt plus 3pt minus 3pt
\centerline{\hfont{nanummjm}{at 11pt}도널드 어빈 크누스, 실비오 레비}

\vskip 24pt plus 3pt minus 3pt
\noindent
%This document describes a version of Don Knuth's \.{WEB} system,
%adapted to \CEE/ by Silvio Levy.  Since its creation in 1987, \.{CWEB}
%has been revised and enhanced in various ways, by both Knuth and Levy.
%We now believe that its evolution is near an end; however, bug
%reports, suggestions and comments are still welcome, and
%should be sent to Levy (\.{levy@math.berkeley.edu}).
이 문서는 돈 크누스의 \.{WEB} 시스템을 실비오 레비가 \CEE/에 적용한 버전을 기술한다.
\.{CWEB} 시스템이 만들어진 1987년 이래로, 크누스와 레비는 다양한 방법으로 시스템을 수정하고
보완하여왔다. 우리는 현재의 시스템이 진화를 거듭한 끝에 거의 막바지에 다다랐다고 믿지만,
여전히 오류 신고, 제안 사항, 도움말들이 있으면, 주저하지 말고 레비
(\.{levy@math.berkeley.edu})에게 연락하기 바란다.

%Readers who are familiar with Knuth's memo ``The \.{WEB} System of Structured
%Documentation'' will be able
%to skim this material rapidly, because \.{CWEB} and \.{WEB} share
%the same philosophy and (essentially) the same syntax.  In some respects
%\.{CWEB} is a simplification
%of \.{WEB}: for example, \.{CWEB} does not need \.{WEB}'s features
%for macro definition and string handling, because \CEE/ and its
%preprocessor already take care of macros and strings. Similarly, the \.{WEB}
%conventions of denoting octal and hexadecimal constants by \.{@'77}
%and \.{@"3f} are replaced by \CEE/'s conventions \.{077} and
%\.{0x3f}. All other features of \.{WEB} have been
%retained, and new features have been added.
크누스의 메모 ``구조적 문서화를 위한 \.{WEB} 시스템''이란 문서에 익숙한 독자는 이 문서를
중간 중간 건너뛰면서 빠르게 읽어 나갈 수 있다. 왜냐하면  \.{CWEB}과 \.{WEB}은 같은 철학과
(근본적으로는) 같은 문법을 공유하기 때문이다. 어떤 면에서 보면 \.{CWEB}은 \.{WEB}의
단순화된 형태이다. 예를들면, \.{CWEB}은 \.{WEB}의 매크로 정의와 문자열 처리 기능이 필요
없는데, 이는 \CEE/ 언어와 그의 전처리기가 매크로와 문자열 처리를 담당하고 있기 때문이다.
비슷한 이유로, \.{@'77}과 \.{@"3f}처럼 8진수와 16진수를 표시하던  \.{WEB}의 방식은
\.{077}나 \.{0x3f}처럼 \CEE/의 방식으로 바뀌었다.
그 외의 다른 모든 \.{WEB}의 기능들은 유지되면서 새로운 기능이 추가되었다.

%We thank all who contributed suggestions and criticism to
%the development of \.{CWEB}. We are especially grateful to Steve
%Avery, Nelson Beebe, Hans-Hermann Bode, Klaus Guntermann, Norman Ramsey,
%Joachim Schnitter, and Saroj Mahapatra,
%who contributed code, and to Cameron Smith, who
%made many suggestions improving the manual.  Ramsey has made
%literate programming accessible to users of yet other languages by means of
%his \.{SPIDER} system [see {\sl Communications of the ACM\/ \bf32} (1989),
%1051--1055]. The book {\sl Literate Programming\/} by Knuth (1992) contains
%a comprehensive bibliography of related early work.
%Bode, Schnitter, and Mahapatra adapted \.{CWEB} so that it works
%for \CPLUSPLUS/ as well; therefore in the text below you can read \CPLUSPLUS/
%for \CEE/ if you so desire.
\.{CWEB} 개발에 많은 제안과 비평을 해준 모든 분들께 감사의 말을 전한다. 특히, 코드 작성에
기여한 스티브 애버리, 넬슨 비베, 한스-헤르만 보데, 클라우스 군테르만, 노만 램지, 조킴 슈니터,
사로 마하파트라, 그리고 이 문서 작성에 많은 도움을 준 카메론 스미스 에게 감사의 말을 전한다.
램지는 그가 만든 \.{SPIDER} 시스템으로 [{\sl Communications of the ACM\/ \bf32}
(1989), 1051--1055] 다른 프로그래밍 언어로 문학적 프로그래밍을 하도록 해주었다. 크누스의
책인 {\sl\hfontsize{at 9pt}문학적 프로그래밍\/} (1992)은 문학적 프로그래밍과 관련된 많은
참고 문헌들과 초창기 작업의 풍부하고 흥미로운 내용을 담고 있다. 보데, 슈니터, 마하파트라는
\.{CWEB}에서 \CPLUSPLUS/도 사용할 수 있도록 하여서, 독자들은 원하기만 하면, 이 글에서
\CEE/를 \CPLUSPLUS/로 바꾸어 생각해도 된다.

%\section Introduction.
%The philosophy behind \.{CWEB} is
%that programmers who want to provide the best
%possible documentation for their programs need two things
%simultaneously:  a language like \TEX/ for formatting, and a language like
%\CEE/ for programming. Neither type of language can provide the
%best documentation by itself. But when both are appropriately combined, we
%obtain a system that is much more useful than either language separately.
\section 서 문.
\.{CWEB}의 배경이 되는 철학은 자신이 작성하는 프로그램에 가능한 최상의 문서를 제공하고자 하는
프로그래머는 동시에 두 가지가 필요한데, 이는 조판을 위한 \TEX/과 같은 언어와 프로그래밍을 위한
\CEE/와 같은 프로그래밍 언어이다. 두 가지 언어 중 어느 한 가지만 가지고는 최상의 문서를
제공할 수 없다. 하지만 이 두 가지 언어를 적절히 조합하면, 각각의 언어가 발휘할 때 보다 더욱
유용한 기능을 갖춘 시스템을 얻게된다.

%The structure of a software program may be thought of as a ``web'' that is
%made up of many interconnected pieces. To document such a program, we want
%to explain each individual part of the web and how it relates to its
%neighbors. The typographic tools provided by \TEX/ give us an opportunity
%to explain the local structure of each part by making that structure
%visible, and the programming tools provided by \CEE/ make it possible
%for us to specify the algorithms formally and unambiguously. By combining
%the two, we can develop a style of programming that maximizes our ability
%to perceive the structure of a complex piece of software, and at the same
%time the documented programs can be mechanically translated into a working
%software system that matches the documentation.
소프트웨어 프로그램의 구조는 내부적으로 많은 작은 부분들이 서로 긴밀히 연결되어 있는
``웹(web)'' 이라고 생각할 수 있다. 그러한 프로그램을 문서화 하기 위해서, 웹을 구성하는 각각의
개별적인 부분들을 설명하고, 그 부분들이 그의 이웃들과 어떠한 연관이 있는지를 설명하고자 할
것이다. \TEX/이 제공하는 인쇄 도구는 각 부분의 구조를 눈에 보이게 하면서 설명할 기회를 주고,
\CEE/가 제공하는 프로그래밍 언어는 알고리즘을 수학적으로 명확하게 기술하도록 해준다.
이 두 언어를 잘 조합함으로써, 우리는 소프트웨어의 복잡한 구조를 쉽게 파악 할 수 있는 능력을
극대화 하는 프로그래밍하는 방법을 개발 할 수 있고, 동시에 문서화된 프로그램은 기계적으로 그
문서에 대응되는 실행 가능한 소프트웨어로 번역 될 수 있다.

%The \.{CWEB} system consists of two programs named \.{CWEAVE} and \.{CTANGLE}.
%When writing a \.{CWEB} program the user keeps the
%\CEE/ code and the documentation in the same file, called the \.{CWEB}
%file and generally named \.{something.w}.  The command
%`\.{cweave} \.{something}' creates an output file \.{something.tex}, which
%can then be fed to \TEX/, yielding a ``pretty printed'' version of
%\.{something.w} that correctly handles
%typographic details like page layout and the use of indentation,
%italics, boldface, and mathematical symbols. The typeset output also
%includes extensive cross-index
%information that is gathered automatically.  Similarly, if you run the
%command `\.{ctangle} \.{something}' you will get a \CEE/ file \.{something.c},
%which can then be compiled to yield executable code.
\.{CWEB} 시스템은 \.{CWEAVE}와 \.{CTANGLE}이라고 불리는 두 개의 프로그램으로 구성되어
있다. \.{CWEB} 프로그래밍을 할 때는, \.{CWEB} 파일이라고 불리는 하나의 파일에 \CEE/ 코드와
문서화를 동시에 작성하고 그 파일의 이름은 일반적으로 \.{something.w}과 같이 확장자를
\.{.w}로 한다. 명령어 `\.{cweave} \.{something}'를 통해서 \.{something.tex} 파일을
생성하고, 그 파일은 \TEX/으로 컴파일하여 페이지 지면 배열과 프로그램의 들여쓰기, 그리고
예약어나 변수명 혹은 수학 기호들을 각각 다른 폰트로 표현하는 등 우리가 쉽게 읽을 수 있도록
멋진 출력물로 바뀐다. 이 출력물에는 \.{cweave} 명령어가 자동으로 만든 광범위한 상호 참조도
들어있다. 비슷한 방법으로, `\.{ctangle} \.{something}' 이란 명령을 실행하면
\.{something.c}란  이름의 \CEE/ 파일을 얻을 수 있고, 이 파일을 컴파일하여 실행 파일을
얻을 수 있다.

%Besides providing a documentation tool, \.{CWEB} enhances the \CEE/
%language by providing the
%ability to permute pieces of the program text, so that a large system can
%be understood entirely in terms of small sections and their local
%interrelationships.  The \.{CTANGLE} program is so named because it takes a
%given web and moves the sections from their web structure into the order
%required by \CEE/; the advantage of programming in \.{CWEB} is that the
%algorithms can be expressed in ``untangled'' form, with each section
%explained separately.  The \.{CWEAVE} program is so named because it takes
%a given web and intertwines the \TEX/ and \CEE/ portions contained in
%each section, then it knits the whole fabric into a structured document.
%(Get it? Wow.)  Perhaps there is some deep connection here with the fact
%that the German word for ``weave'' is ``{\it webe\/}'', and the
%corresponding Latin imperative is ``{\it texe\/}''!
\.{CWEB}은 \CEE/ 프로그래밍에 단순히 문서화 도구를 제공하는 뿐만 아니라, \CEE/ 언어를
보강하기도 하는데, 이는 프로그램을 여러 작은 조각들로 나누어 큰 시스템을 작은 부분들과 그
각각의 관계로 전체를 이해하는 것이 가능하다. \.{CTANGLE}이란 프로그램은 주어진 웹에서 각
부분들을 그 구조로 부터 \CEE/가 요구하는 순서의 구조로 옮겨서 이름에 그렇게 지어졌다.
\.{CWEB}으로 프로그래밍 하는 잇점은 알고리즘을 얽히지 않은 형태로 표현하고 각 부분을
개별적으로 설명하는 것이다. \.{CWEAVE}라는 프로그램은 웹 파일을 입력받아서 각 부분을
구성하는 \TEX/ 코드와 \CEE/ 코드를 서로 얽히게 놓은 다음에 뜨개질이나 직물을 짜내듯이
하나의 멋진 구조화된 문서를 만들어 내기 때문에 그러한 이름을 얻었다.(그렇죠? 와우.)
영어 단어 ``weave''에 해당하는 독일 단어가 ``webe''이고, 그에 해당하는 라틴어가
``texe''라는 것에는 아마도 일종의 깊은 연관이 있을 것이다. 

%A user of \.{CWEB} should be fairly familiar with the \CEE/
%programming language.  A minimal amount of acquaintance with \TEX/ is also
%desirable, but in fact it can be acquired as one uses \.{CWEB}, since
%straight text can be typeset in \TEX/ with virtually no knowledge of
%that language.  To someone familiar with both \CEE/ and \TEX/ the amount of
%effort necessary to learn the commands of \.{CWEB} is small.
\.{CWEB}을 하기 위해서는 반드시 \CEE/ 프로그래밍 언어를 잘 알아야 한다. 최소한의 \TEX/에
대한 지식도 있으면 좋겠지만, \.{CWEB} 프로그래밍을 하면서 \TEX/에 대한 지식은 자동으로 얻게
될 것인데, 이는 \TEX/에서 사용되는 단순한 글은 \TEX/에 대한 지식이 거의 없어도 가능하기
때문이다. 물론 \CEE/ 언어와 \TEX/에 모두 익숙한 독자는 적은 노력으로 \.{CWEB} 명령어를
배우게 된다.

%\section Overview.
%Two kinds of material go into \.{CWEB} files: \TEX/ text and \CEE/ text.
%A programmer writing in \.{CWEB} should be thinking both of the
%documentation and of the \CEE/ program being created;
%i.e., the programmer should be instinctively aware of the different
%actions that \.{CWEAVE} and \.{CTANGLE} will perform on the \.{CWEB} file.
%\TEX/ text is essentially copied without change by \.{CWEAVE}, and it is
%entirely deleted by \.{CTANGLE}; the \TEX/ text is ``pure
%documentation.'' \CEE/ text, on the other hand, is formatted by
%\.{CWEAVE} and it is shuffled around by \.{CTANGLE}, according to rules that
%will become clear later. For now the important point to keep in mind is
%that there are two kinds of text. Writing \.{CWEB} programs is something
%like writing \TEX/ documents, but with an additional ``\CEE/ mode''
%that is added to \TEX/'s horizontal mode, vertical mode, and math mode.
\section 개 요.
\.{CWEB} 파일에는 \TEX/ 텍스트와 \CEE/ 텍스트라는 두 종류의 소재가 들어간다.
\.{CWEB} 프로그래머는 만들고자 하는 문서와 \CEE/ 프로그램을 동시에 생각해야
한다. 즉, 프로그래머는 본능적으로 \.{CWEAVE}와 \.{CTANGLE} 프로그램이
\.{CWEB} 파일을 어떻게 처리할지를 미리 알고 있어야 한다. \.{CWEAVE}는
\TEX/ 파일을 만들 때, \.{CWEB} 파일 내의 \TEX/ 텍스트를 그대로
\TEX/ 파일에 복사하지만, \.{CTANGLE}은 \CEE/ 파일을 만들 때, \TEX/
텍스트를 모두 제거하여 전혀 포함하지 않는다. \TEX/ 텍스트는 순전히
문서화를 위한 텍스트로써, \.{CTNALGE}은 이를 완전히 무시하는 반면에,
\.{CWEAVE}는 \CEE/ 텍스트를 보기 좋게 문서화 하고, \.{CTNALGE}은 그것을
본래 \CEE/ 문법에 맞게 재배치한다. 재배치 규칙은 나중에
자세히 설명하게 될 것이다. 지금 이 순간, 기억해야 할 것은 두 종류의
텍스트가 있다는 것이다. \.{CWEB} 프로그램을 작성한다는 것은 보통의 \TEX/
문서를 만든다는 것과 같고, 단지 \TEX/의 기본적인 모드인 수평 모드,
수직 모두, 수학 모드에 ``\CEE/ 모드''가 더해진 것이라고 생각하면 된다.

%A \.{CWEB} file is built up from units called {\sl sections\/} that are more
%or less self-contained.  Each section has three parts:
\.{CWEB} 파일은 {\sl\hfontsize{at 9pt}마디}라고 하는 하나의 온전한 의미가 있는 단위들로
구성된다. 각 마디는 다음과 같은 세 개의 부분을 가지고 있다.
%부(部, part)를 가지고 있다.

%\yskip\item{$\bullet$} A \TEX/ part, containing explanatory material about
%what is going on in the section.
\yskip\item{$\bullet$} \TEX/ 부분, 마디가 어떤 일을 하는지에 대한 설명을 담고 있다.

%\item{$\bullet$} A middle part, containing macro definitions that serve as
%abbreviations for \CEE/ constructions that would be less comprehensible
%if written out in full each time. They are turned by \.{CTANGLE} into
%preprocessor macro definitions.
\item{$\bullet$} 중간 부분, \CEE/언어에서 제공하는 매크로 정의 부분을 담고
있다. 이 부분은 \.{CTNAGLE}에 의해서 전처리기 매크로 정의로 변환된다.

%\item{$\bullet$} A \CEE/ part, containing a piece of the program that
%\.{CTANGLE} will produce. This \CEE/ code should ideally be about a
%dozen lines long, so that it is easily comprehensible as a unit and so
%that its structure is readily perceived.
\item{$\bullet$} \CEE/ 부분, \.{CTANGLE}이 만들어내는 \CEE/ 프로그램 코드이다.
\CEE/ 코드는 하나의 단위로 쉽게 이해될 수 있을 정도의 십여 줄 내외의 프로그램 텍스트이다. 

%\yskip\noindent The three parts of each section must appear in this order;
%i.e., the \TEX/ commentary must come first, then the middle part, and
%finally the \CEE/ code. Any of the parts may be empty.
\yskip\noindent 위의 세 부분은 마디에서 반드시 위의 순서대로 나열되어야 한다. 
즉 \TEX/ 부분이 가장 먼저 오고, 그 다음에 중간 부분, 마지막으로 \CEE/ 코드가 온다.
세 개의 부분들 중 어느 것이나 생략할 수 있다.

%A section begins with either of the symbols `\.{@\ }' or `\.{@*}', where
%`\.{\ }' denotes a blank space. A section ends
%at the beginning of the next section (i.e., at the next
%`\.{@\ }' or `\.{@*}'), or at the end of the file, whichever comes first.
%The \.{CWEB} file may also contain material that is not part of any section
%at all, namely the text (if any) that occurs before the first section.
%Such text is said to be ``in limbo''; it is ignored by \.{CTANGLE}
%and copied essentially verbatim by \.{CWEAVE}, so its function is to
%provide any additional formatting instructions that may be desired in the
%\TEX/ output. Indeed, it is customary to begin a \.{CWEB} file with
%\TEX/ code in limbo that loads special fonts, defines special macros,
%changes the page sizes, and/or produces a title page.
하나의 마디는 `\.{@\ }' 또는 `\.{@*}' 기호로 시작된다. 여기서 `\.{\ }'는 빈칸을
나타낸다. 마디는 그 다음 마디가 시작하는 곳에서 끝나거나
(즉, 다음의 `\.{@\ }' 또는 `\.{@*}') 파일의 끝에서 끝난다. 또한, \.{CWEB} 
파일에는 어떠한 마디에도 포함되지 않는 텍스트가 있다. 이른바 맨 첫 번째
마디 앞에 나오는 텍스트로써 그러한 텍스트를 ``인림보(in limbo)''라고 부른다.
\.{CTNAGLE}은 인림보를 완전히 무시하지만 \.{CWEAVE}는 \TEX/ 파일에
글자 그대로 복사한다. 인림보의 기능은 원하는 \TEX/ 출력 결과를 얻기 위한 문서
포멧팅 명령어들을 제공하는 것이다. 인림보에 특별한 글꼴을 불러오거나 매크로를 정의 
하거나 페이지의 크기를 정한다거나 혹은 타이틀 페이지를 만드는 코드 등을
인림보에 담아서 \.{CWEB}파일을 시작하는 관례이다.

%Sections are numbered consecutively, starting with 1. These numbers appear
%at the beginning of each section of the \TEX/ documentation output by
%\.{CWEAVE}, and they appear
%as bracketed comments at the beginning and end of the code generated by that
%section in the \CEE/ program output by \.{CTANGLE}.
마디들은 1부터 시작해서 순차적으로 번호가 붙는다. 이 번호들은 \.{CWEAVE}가
만들어 내는 \TEX/ 문서의 각 마디의 맨 앞에 붙어서 마디의 시작을 나타낸다.
그리고 이 번호들은 \.{CTANGLE}이 만들어내는 \CEE/ 프로그램에서 해당 마디에
의해서 만들어진 코드의 시작부와 끝 부분에 \CEE/의 주석문으로 나타나기도 한다.

%\section Section Names.
%Fortunately, you never mention these numbers yourself when you are writing
%in \.{CWEB}. You just say `\.{@\ }' or `\.{@*}' at the beginning of each
%new section, and the numbers are supplied automatically by \.{CWEAVE} and
%\.{CTANGLE}. As far as you are concerned, a section has a
%{\sl name\/} instead of a number; its name is specified by writing
%`\.{@<}' followed by \TEX/ text followed by `\.{@>}'. When \.{CWEAVE}
%outputs a section name, it replaces the `\.{@<}' and `\.{@>}' by
%angle brackets and inserts the section number in small type. Thus, when you
%read the output of \.{CWEAVE} it is easy to locate any section that is
%referred to in another section.
\section 마디 이름. 
%마디 이름과 관련해서 좋은 소식과 나쁜 소식이 있다. 먼저 좋은 소식은
다행히도 \.{CWEB} 파일을 작성할 때, 마디 앞에 붙는 마디 번호를 매기느라 수고할 필요가 전혀
없다. 그저 단순히 각 마디의 시작부에 `\.{@\ }' 혹은 `\.{@*}'를 붙여주면, 마디 번호는
\.{CWEAVE}와 \.{CTANGLE}이 알아서 자동적으로 매겨준다. 마디는 번호로 인식되는 것이 아니라
이름({\sl name\/})으로 인식된다. 마디 이름은 `\.{@<}', \TEX/ 텍스트, `\.{@>}' 순으로
구성된다. \.{CWEAVE}가 마디 이름을 출력할 때는 `\.{@<}'와 `\.{@>}'를 꺾쇠 괄호로 바꾸고
그 안에 작은 글꼴로 마디 번호를 매겨 넣는다. 따라서 \.{CWEAVE}가 만들어낸 출력물을 읽을 때
하나의 마디에서 언급된 다른 마디는 번호를 이용해서 쉽게 찾을 수 있다.

%For expository purposes, a section name should be a good description of the
%contents of that section; i.e., it should stand for the abstraction
%represented by the section. Then the section can be ``plugged into'' one or
%more other sections in such a way
%that unimportant details of its inner workings
%are suppressed.  A section name therefore ought to be long enough to convey
%the necessary meaning.
마디가 어떤 일을 하는지 한눈에 알아보기 위해서, 마디 이름은 그 마디의 내용을 잘 설명할
수 있는 것이어야 한다. 즉, 마디 이름은 그 이름만 보고도 이 마디가 무슨 일을 하는 마디인지 알
수 있을 정도의 일목요연한 문구이어야 한다. 마디는 다른 마디에 포함될 수 있는데, 이때 마디를
포함하는 마디에서 중요하지 않지만 상세한 설명이 필요한 부분에 그 설명에 해당하는 마디를 끼워
넣어서 마디를 보다 읽기 좋고 간결하게 할 수 있다. 따라서 마디 이름은 그 마디가 전달하고자 하는
의미를 분명히 반영할 수 있을 정도의 충분한 길이이어야 한다.

%Unfortunately, it is laborious to type
%such long names over and over again, and it is also difficult to specify a
%long name twice in exactly the same way so that \.{CWEAVE} and \.{CTANGLE}
%will be able to match the names to the sections. To ameliorate this situation,
%\.{CWEAVE} and \.{CTANGLE} let you abbreviate a section name, so long as
%the full name appears somewhere in the \.{CWEB} file; you can type simply
%`\.{@<$\alpha$...@>}', where $\alpha$ is any string that is a prefix of
%exactly one section name appearing in the file. For example, `\.{@<Clear
%the arrays@>}' can be abbreviated to `\.{@<Clear...@>}' if no other section
%name begins with the five letters `\.{Clear}'.  Elsewhere
%you might use the abbreviation `\.{@<Clear t...@>}', and so on.
%나쁜 소식은 
불행하게도, 위 설명대로 마디가 하는 일을 요약해서 마디 이름을 정하면, 마디 이름이 다소 길어질
경우가 생기는데, 마디 이름이 길어지면 나중에 혹은 다른 곳에서 그 마디를 언급할 필요가 있을
때마다 마디의 긴 이름을 매번 자판을 두드려 입력해야 하는 매우 수고로운 작업을 해야 한다.
이런 수고를 덜기 위해서, \.{CWEAVE}와 \.{CTANGLE}은 마디 이름을 간단하게 하는 기능을
가지고 있다. \.{CWEB} 파일 내의 어디에선가 정확한 마디 이름이 한번이라도 언급이 되면, 단순히
`\.{@<$\alpha$...@>}'처럼 입력해 넣으면 된다. 여기서 $\alpha$는 하나의 마디를 정확하게
구분해 낼 수 있는 최소 길이의 접두어이다. 예를 들면, `\.{@<Clear the arrays@>}'라는 마디
이름은 `Clear'라는 다섯 글자로 시작되는 다른 마디 이름이 없다면 `\.{@<Clear...@>}'로
간단하게 할 수 있다. 만약에 `Clear'로 시작하는 다른 마디가 있다면, `\.{@<Clear t...@>}'
등과 같이 이 마디를 다른 마디 이름과 구별할 수 있는 최소의 길이로 간략화하면 된다.

%Section names must otherwise match character for character, except
%that consecutive characters of white space (spaces, tab marks, newlines,
%and/or
%form feeds) are treated as equivalent to a single space, and such spaces are
%deleted at the beginning and end of the name. Thus, `\.{@< Clear { }the
%arrays @>}' will also match the name in the previous example.
%Spaces following the ellipsis in abbreviations are ignored as well, but
%not those before, so that `\.{@<Clear t ...@>}' would not match
%`\.{@<Clear the arrays@>}'.
두 개의 마디 이름이 같은 것인지 비교 할 때는 마디 이름을 구성하는 문자 하나하나를 서로
매치해가면서 비교하게 된다. 이때 한 가지 예외가 있는데, 여러 개의 연속적인 공백 문자(스페이스,
탭, 개행문자 또는 용지먹임(form feed)들은 하나의 빈칸으로 인식되고 마디 이름의 앞과 뒤에
나오는 공백 문자들은 제거된다. 따라서 마디 이름 `\.{@< Clear the arrays@>}'는 앞서 예와
같은 이름이다. 간략화된 이름에서 생략기호($\ldots$) 다음에 나오는 공백도 마찬가지로
무시되지만, 생략기호 앞에 나오는 공백문자는 공백으로 인식된다. 따라서
`\.{@<Clear t ...@>}'는 `\.{@<Clear the arrays@>}'와 서로 다른 마디이다.

%\section What \.{CTANGLE} Does.
%We have said that a section begins with `\.{@\ }' or `\.{@*}', but we
%didn't say how it gets divided up into a \TEX/ part, a middle part,
%and a \CEE/ part. The middle part begins with the first appearance of
%`\.{@d}' or `\.{@f}' in the section, and the \CEE/ part begins with
%the first appearance of `\.{@c}' or `\.{@<section name@>=}'.  In the
%latter case you are saying, in effect, that the section name stands
%for the \CEE/ text that follows. Alternatively, if the \CEE/ part
%begins with `\.{@c}' instead of a section name, the current section is
%said to be {\sl unnamed}.
\section \.{CTANGLE}이 하는 일.
마디 이름이 `\.{@\ }' 또는 `\.{@*}'로 시작된다고 하였는데, 마디가 어떻게 \TEX/ 부분,
중간 부분, \CEE/ 부분의 세 부분으로 나뉘는지에 대해서는 설명하지 않았다.  마디 내에서
중간 부분은 `\.{@d}' 또는 `\.{@f}'가 처음 나타나는 곳 부터 시작되고, \CEE/ 부분은
`\.{@c}' 또는 `\.{@<마디 이름@>=}'이 처음 나타나는 곳 부터 시작된다. 후자의 경우로
\CEE/ 부분을 시작한다는 것은 마디 이름이 \CEE/ 텍스트가 하는 일을 요약 설명한다는 것을
의미한다. \CEE/ 부분이 마디 이름 대신에 `\.{@c}'로 시작하면 그 마디는 마디 이름이 없다는
것을 뜻한다.

%The construct `\.{@<section name@>}' can appear
%any number of times in the \CEE/ part of a section:
%Subsequent appearances indicate that a named section is being
%``used'' rather than ``defined.'' In other words, the
%\CEE/ code for the named section, presumably defined elsewhere, should be
%spliced in at this point in the \CEE/ program.  Indeed,
%the main idea of \.{CTANGLE} is to make a \CEE/ program out of
%individual sections, named and unnamed.  The exact way in which this is done
%is this: First all the macro definitions
%indicated by `\.{@d}' are turned into \CEE/ preprocessor macro definitions
%and copied at the beginning.
%Then the \CEE/ parts of unnamed sections are copied down,
%in order; this constitutes the first-order
%approximation to the text of the program. (There should be at least
%one unnamed section, otherwise there will be no program.) Then all section
%names that appear in the first-order approximation are replaced by the \CEE/
%parts of the corresponding sections, and this substitution process
%continues until no section names remain. All comments are removed, because
%the \CEE/ program is intended only for the eyes of the \CEE/ compiler.
마디 이름을 구성하는 `\.{@<마디 이름@>}'은 다른 마디들의 \CEE/ 부분에 몇 번이고 나타날 수
있다. 최초에 마디의 \CEE/ 부분을 정의하기 위해 쓰인 `\.{@<마디 이름@>}'를 제외한 나머지의
쓰임은 그 마디가 ``정의''된다는 의미가 아니라 ``사용''된다는 의미이다. 다른 말로 하면,
이미 다른 곳에서 정의된 마디의 \CEE/ 코드는 현재 `\.{@<마디 이름@>}'이 나타난 \CEE/
프로그램의 바로 그 지점에 삽입되어야 한다는 것이다. \.{CTANGLE}이 하는 주된 일은 각각의
이름을 가진 마디와 이름이 없는 마디들로부터 \CEE/ 프로그램을 만들어 낸다는 것이다. 그 방법은
다음과 같다. 먼저 `\.{@d}'로 표시되는 모든 매크로 정의들이 \CEE/ 언어의 전처리기 매크로
정의들로 변환되면서 파일의 맨 처음으로 복사된다. 그리고 나서 이름 없는 마디의 \CEE/ 부분들이
그다음에 차례로 복사된다. \.{CTANGLE}은 여러 번의 패스를 통하여 \.{CWEB} 파일로부터 \CEE/
프로그램을 만들어 내는데, 지금까지의 절차로 말미암아 1차 패스의 결과물을 얻게 된다.
(\.{CWEB} 파일에는 이름 없는 마디가 적어도 하나는 있어야 한다. 그렇지 않으면 \CEE/
프로그램이 만들어지지 않는다.) 그리고 나서, 1차 패스 결과물 내의 모든 마디 이름들은 해당하는
마디의 \CEE/ 부분으로 치환되고, 이 치환 작업은 이름을 가진 마디가 하나도 남지 않을 때까지
계속된다. 또한, \CEE/ 부분 내의 모든 주석문들은 제거된다. \.{CTANGLE}이 만들어 내는
\CEE/ 프로그램은 오로지 \CEE/ 컴파일러용으로 출력물을 만들기 때문이다.

%If the same name has been given to more than one section, the \CEE/ text
%for that name is obtained by putting together all of the \CEE/ parts in
%the corresponding sections. This feature is useful, for example, in a
%section named `Global variables', since one can then
%declare global variables in whatever sections those variables are
%introduced. When several sections have the same name, \.{CWEAVE} assigns the
%first section number as the number corresponding to that name, and it
%inserts a note at the bottom of that section telling the reader to `See
%also sections so-and-so'; this footnote gives the numbers of all the other
%sections having the same name as the present one. The \CEE/ text
%corresponding to a section is usually formatted by \.{CWEAVE} so that the
%output has an equivalence sign in place of the equals sign in the \.{CWEB}
%file; i.e., the output says `$\langle\,$section
%name$\,\rangle\equiv\null$\CEE/ text'. However, in the case of the second
%and subsequent appearances of a section with the same name, this `$\equiv$'
%sign is replaced by `$\mathrel+\equiv$', as an indication that the
%following \CEE/ text is being appended to the \CEE/ text of another section.
만일 같은 마디 이름이 다른 여러 마디의 이름으로 사용된다면, 그 이름을 가진 마디의
\CEE/ 부분은 이 마디 이름을 사용하는 마디의 \CEE/ 부분들의 합으로 이루어진다. \.{CWEB}의
이 기능은 매우 편리한 기능이다. \CEE/ 언어에서 전역 변수는 대개 프로그램의 첫 부분에 그
프로그램에서 사용되는 모든 전역 변수들을 한 곳에서 정의하거나 선언한다. 그런데 모든 변수들을
한 곳에서 정의하기보다는 그 변수가 사용되는 시점에서 그 변수에 대한 설명과 함께 정의를 하면
그 변수에 대한 의미가 확실해지고 프로그램을 읽어나가기가 더울 수월해질 것이다. 하지만, \CEE/
언어는 이러한 기능을 제공하지 않는다. 이때, 방금 전에 설명한 \.{CWEB}의 기능을 사용할 수
있다. 예를 들어 `전역 변수들'이라는 이름을 가진 마디가 있다고 하자. 이 마디가 여러 마디의 이름
으로 사용된다면, 전역 변수들은 여러 마디에 걸쳐서 정의되지만, 결국은 이 모든 전역 변수들이
처음에 정의한 `전역 변수들' 이란 이름을 가진 마디에 모이게 된다. 여러 개의 마디가 같은 이름을
가질 때마다, \.{CWEAVE}는 그 이름의 마디가 처음 정의되는 곳에 그 마디의 번호를 할당하고,
그 마디 밑에 각주로써 이 마디가 다른 마디에서 사용된다는 것을 알려주기 위해서
`이러저러한 마디들도 보라.'라고 써넣는다. 이 각주에는 이 마디를 \CEE/ 부분으로
사용하는 모든 마디들의 번호가 나타나게 된다. 마디의 \CEE/ 텍스트는 \.{CWEB} 파일 내에서
등호(=)기호로 나타나는데, 이는 \.{CWEAVE}에 의해서 `$\equiv$' 기호로 바뀐다.
즉, `$\langle\,$마디 이름$\,\rangle\equiv\null$\CEE/ 텍스트'로 바뀌게 된다.
그러나 동일한 이름의 마디가 여러 번 나타날 때, 두 번째 부터는 현재 나타나는 마디는
다른 마디의 \CEE/ 텍스트에 첨가된다는 의미로 `$\equiv$' 기호가 `$\mathrel+\equiv$'로
바뀐다.

%As \.{CTANGLE} enters and leaves sections, it inserts preprocessor
%\.{\#line} commands into the \CEE/ output file.  This means that
%when the compiler gives you error messages, or when you debug your program,
%the messages refer to line numbers in the \.{CWEB} file, and not in the
%\CEE/ file.  In most cases you can therefore
%forget about the \CEE/ file altogether.
\.{CTANGLE}은 마디를 처리할 때마다, \.{CTANGLE}의 출력물인 \CEE/ 파일에 전처리기
명령어인 \.{\#line} 문을 써넣는다. 이는 \CEE/ 파일을 컴파일 할때, 컴파일러가 에러 메시지를
내거나, 혹은 디버깅 할때, \CEE/ 파일이 아니라 \.{CWEB} 파일의 라인 넘버가 나오므로 원래
작성한 파일을 가지고 작업을 할 수 있도록 하기 위함이다. 따라서 대개 \.{CTANGLE}이 만들어
낸 \CEE/ 파일을 직접 다루는 일은 없다. 

%\section What \.{CWEAVE} Does.
%The general idea of \.{CWEAVE} is to make a \.{.tex} file from the \.{CWEB}
%file in the following way: The first line of the \.{.tex} file
%tells \TEX/ to input a file with macros that
%define \.{CWEB}'s documentation conventions. The next lines of the file
%will be copied from whatever \TEX/ text is in limbo before the first
%section.  Then comes the output for each section in turn, possibly
%interspersed with end-of-page marks.  Finally, \.{CWEAVE} will generate a
%cross-reference index that lists each section number in which each \CEE/
%identifier appears, and it will also generate an alphabetized list
%of the section names, as well as a table of contents that
%shows the page and section numbers for each ``starred'' section.
\section \.{CWEAVE}가 하는 일.
\.{CWEAVE}가 하는 일은 \.{CWEB} 파일로부터 \.{.tex} 파일을 만들어내는 것인데, 그 방법은
다음과 같다. 우선 \.{.tex}의 맨 첫 줄은 다른 \.{.tex} 파일들이 그렇듯이, \.{CWEB}의
문서화를 위한 규칙들을 만들어 놓은 매크로 파일을 포함하는 것이다. 그리고 다음 줄부터 \.{CWEB}
파일의 ``인림보''에 정의된 \TEX/ 텍스트들이 그대로 복사된다. 그 다음부터 각 마디에 대한 출력
결과가 차례로 나온다. 이 과정에서 각 페이지에 마디들이 적절히 배치된다. 마지막으로,
\.{CWEAVE}는 색인 페이지를 만드는데, 이 색인 페이지에는 각각의 \CEE/ 프로그램에 사용된
변수명 이라든가 여러 식별자 들이 나오고, 이 파일을 구성하는 모든 마디들이 마디 이름의
알파벳순으로 나온다. 또한, 색인 페이지뿐만 아니라 별표 마디와 그의 마디 번호를 나타내는
목차도 만든다.

%What is a ``starred'' section, you ask? A section that begins with `\.{@*}'
%instead of `\.{@\ }' is slightly special in that it denotes a new major
%group of sections. The `\.{@*}' should be followed by the title of this
%group, followed by a period. Such sections will always start on a new page
%in the \TEX/ output, and the group title will appear as a running headline
%on all subsequent pages until the next starred section. The title will also
%appear in the table of contents, and in boldface type at the beginning of
%its section. Caution:  Do not use \TEX/ control sequences in such titles,
%unless you know that the \.{cwebmac} macros will do the right thing with
%them. The reason is that these titles are converted to uppercase when
%they appear as running heads, and they are converted to boldface when they
%appear at the beginning of their sections, and they are also written out to
%a table-of-contents file used for temporary storage while \TEX/ is
%working; whatever control sequences you use must be meaningful in all
%three of these modes.
``별표'' 마디가 무엇인가 하고 의문을 가졌을 것이다. 앞에서 살펴보았듯이 보통의 마디는
`\.{@\ }'로 시작하는데, 별표 마디는 `\.{@*}'로 시작한다. 다른 보통의 마디와 크게 다른 점은
없다. 단순히 여러 마디로 구성되는 그룹을 대표하는 의미가 있다. `\.{@*}' 다음에는 반드시
그룹의 제목이 나오고 그 뒤에 마침표가 나와야 한다. 이러한 별표 마디는 \TEX/ 출력물에서 보면
반드시 새로운 페이지로 시작되고, 제목은 다음 별표 마디가 나오기 전까지 그 페이지와 그 뒤의
모든 페이지의 면주에 사용된다. 그리고 제목은 목차에도 나오고, 마디의 시작부에 굵은 글꼴로
출력된다.
%예를 들어, 어떤 프로그램에서 스택을 이용하고 그 프로그램은 스택을 직접 구현해야 한다고
%하자. 그러면 프로그램 어딘가에 스택을 구현하기 위한 많은 마디가 있을 것이다. 이때 이러한
%마디들을 그룹으로 묶어서 ``스택구현''이라는 제목의 별표 마디를 만들 수 있는 것이다.
주의: 이러한 제목에 \TEX/의 명령어는 사용하지 말아야 한다. 만약에 사용한다면, 출력물이 
원하는 모양이 나오지 않을 수도 있다. 왜냐하면, 제목들이 각 페이지의 면주에 사용 될 때는
제목의 각 글자는 대문자로 바뀌고, 마디의 처음에 나올 때는 굵은 글꼴로 바뀌고, 또한 목차에도
사용되기 때문이다. 이 세 가지 경우 모두 문제를 일으키지 않을 명령어라면 사용해도 된다.

%The \TEX/ output produced by \.{CWEAVE} for each section consists of
%the following: First comes the section number (e.g., `\.{\\M123.}'
%at the beginning of section 123, except that `\.{\\N}' appears in place of
%`\.{\\M}' at the beginning of a starred section). Then comes the
%\TEX/ part of the section, copied almost verbatim except as noted
%below. Then comes the middle part and the \CEE/ part, formatted
%so that there will be a little extra space between them if both are
%nonempty. The middle and \CEE/ parts are obtained by inserting
%a bunch of funny-looking \TEX/ macros into the \CEE/ program; these
%macros handle typographic details about fonts and proper math spacing,
%as well as line breaks and indentation.
각 마디에 대해서 \.{CWEAVE}가 만들어 내는 \TEX/ 출력물은 다음과 같이 구성된다. 먼저 마디
번호가 나온다. (예를 들어, 123번 마디의 시작에 `\.{\\M123.}'가 온다. 만약 이 마디가 별표
마디라면, `\.{\\M}' 대신에 `\.{\\N}'이 쓰인다.) 그리고 나서 그 마디의 \TEX/ 부분이
\.{CWEB} 파일에 입력해 넣은 그대로 복사되어 온다. 그 뒤로 중간 부분, \CEE/ 부분이 그들
사이에 약간의 간격을 두고 온다. 중간부와 \CEE/ 부분은 \TEX/ 부분 처럼 그냥 그대로 복사되어
오는 것은 아니고, 멋진 문서를 만들어 내기 위해서 매크로 파일에 정의된 대로 변환된다.

%\section C Code in \TEX/ Text and Vice Versa.
%When you are typing \TEX/ text, you will probably want to make frequent
%reference to variables and other quantities in your \CEE/ code, and you
%will want those variables to have the same typographic treatment
%when they appear in your text as when they appear in your
%program.  Therefore the \.{CWEB} language allows you to get the effect of
%\CEE/ editing within \TEX/ text, if you place `\.|' marks before and
%after the \CEE/ material. For example, suppose you want to say something
%like this:
%$$\hbox{ If \\{pa} is declared as `\&{int} ${}{*}\\{pa}$',
%the assignment $\\{pa}\K{\AND}\|a[\T{0}]$ makes \\{pa}
%point to the zeroth element of \|a.}$$
%The \TEX/ text would look like this in your \.{CWEB} file:
%$$\lpile{\.{If |pa| is declared as `|int *pa|', the}\cr
%The \TEX/ text would look like this in your \.{CWEB} file:
%$$\lpile{\.{If |pa| is declared as `|int *pa|', the}\cr
%\.{assignment |pa=\&a[0]| makes |pa| point
%to the zeroth element of |a|.}\cr}$$
%And \.{CWEAVE} translates this into something you are glad you didn't have
%to type:
%$$\lpile{\.{If \\\\\{pa\} is declared as
%  `\\\&\{int\} \$\{\}\{*\}\\\\\{pa\}\$',}\cr
%\.{the assignment \$\\\\\{pa\}\\K\{\\AND\}\\|a[\\T\{0\}]\$}\cr
%\.{makes \\\\\{pa\} point to the zeroth element of \\|a.}\cr}$$
\section \TEX/ 텍스트 내의 C 코드와 그의 역.
\TEX/ 텍스트를 입력하다 보면, \CEE/ 코드에서 사용된 변수들이나 다른 \CEE/의 요소들을 \TEX/
텍스트 내에서 사용할 일이 자주 생긴다. 이 경우에 그 변수들이나 요소들을 \CEE/ 코드에서
사용되던 그대로의 글꼴이나 모양을 유지해야 한다. 그래서 \.{CWEB} 언어는 그러한 기능을
제공하는데, 표현하고자 하는 \CEE/ 코드를 `\.|'로 둘러싸면 된다. 예를 들어 다음과 같은 \TEX/
문장을 얻고자 한다고 해보자.
$$\hbox{ 변수 \\{pa}를 `\&{int} ${}{*}\\{pa}$'로 선언하면,
식 $\\{pa}\K{\AND}\|a[\T{0}]$는 \\{pa}가 \|a의 0번째 원소를 가리킨다.}$$
위 문장을 얻기 위해서 \.{CWEB} 파일에 아래와 같은 \TEX/ 텍스트를 입력해야 할 것이다. 
$$\lpile{\.{변수 |pa|를 `|int *pa|'로 선언하면,}\cr
\.{식 |pa=\&a[0]|는 |pa|가 |a|의 0번째 원소를 가리킨다.}\cr}$$
그리고 \.{CWEAVE}는 위의 문장을 마치 암호처럼 보이는 아래와 같은 문장으로 변환한다.
$$\lpile{\.{변수 \\\\\{pa\}가
  `\\\&\{int\} \$\{\}\{*\}\\\\\{pa\}\$'로 선언될 때,}\cr
\.{식 \$\\\\\{pa\}\\K\{\\AND\}\\|a[\\T\{0\}]\$는 %}\cr
\\\\\{pa\}가 \\|a의 0번째 원소를 가리킨다.}\cr}$$
%Incidentally, the cross-reference index that \.{CWEAVE} would make, in
%the presence of a comment like this, would include
%the current section number as one of the index entries for \\{pa},
%even though \\{pa} might not appear in the \CEE/ part of
%this section. Thus, the index covers references to identifiers in
%the explanatory comments as well as in the program itself; you will
%soon learn to appreciate this feature. However, the identifiers
%\&{int} and \|a\ would not be indexed,
%because \.{CWEAVE} does not make index entries for reserved words or
%single-letter identifiers. Such identifiers are felt to be so ubiquitous
%that it would be pointless to mention every place where they occur.
덧붙여 말하자면, 위와 같은 문장이 주어졌을때, 비록 이 마디의 \CEE/ 부분에 \\{pa}가
없을지라도, \.{CWEAVE}가 생성하는 색인은 \\{pa}의 색인으로 현재 마디 번호를 포함한다.
색인은 프로그램에서 사용된 변수들뿐만 아니라, 보통의 설명을 하는 문장에 사용된 변수들까지도
모두 포함한다. 이 기능을 매우 유용한 기능이다. 그러나 \&{int}나 \|a 와 같은 것들은
색인에 포함되지 않는데, 이는 \.{CWEAVE}가 \CEE/ 언어의 예약어나 한 글자짜리 변수나
식별자들은 색인에 포함하지 않는다는 규칙 때문이다. 그러한 변수나 식별자들은 너무나도 흔하고
널리 사용되는 것이라 그들이 사용되는 모든 마디 번호를 색인 포함하는 것은 자칫하면
초점을 흐릴 수 있다.
  
%Although a section begins with \TEX/ text and ends with \CEE/ text, we
%have noted that the dividing line isn't sharp, since \CEE/ text can be
%included in \TEX/ text if it is enclosed in `\pb'.  Conversely, \TEX/ text
%appears frequently within \CEE/ text, because everything in
%comments (i.e., between \.{/*} and \.{*/}, or following \.{//})
%is treated as \TEX/ text.
%Likewise, the text of a section name consists of \TEX/ text, but
%the construct \.{@<section name@>} as a whole is expected to be found
%in \CEE/ text; thus, one typically goes back and forth
%between the \CEE/ and \TEX/ environments in a natural way, as in these
%examples:
%$$
%\displaylines{
%\hbox{\.{if} \.{(x==0)} \.{@<Empty} \.{the} \.{|buffer|} \.{array@>}} \cr
%\hbox{\.{...} \.{using}  \.{the} \.{algorithm}
%\.{in} \.{|@<Empty} \.{the} \.{|buffer|} \.{array@>|.}} }
%$$
비록 마디가 \TEX/ 텍스트로 시작해서 \CEE/ 텍스트로 끝난다고 할지라도, \CEE/ 텍스트가
`\pb'로 둘러싸여 있으면, 그 \CEE/ 텍스트가 \TEX/ 텍스트 내에 포함될 수 있기 때문에,
각 텍스트를 구분하는 곳이 명확하지 않을 수 있다. 역으로, \CEE/의 주석문 (즉, \.{/*} 와
\.{*/} 사이에 있는 것들, 또는 \.{//} 다음에 나오는 것들)에 있는 모든 것들은 사실상 \TEX/
텍스트로 취급되기 때문에, \CEE/ 텍스트 내에도 \TEX/ 텍스트가 매우 빈번히 나타난다고 할 수
있다. 마찬가지로, 마디 이름은 \TEX/ 텍스트이지만, \.{@<마디 이름@>}이 자체가 \CEE/ 텍스트
내에서도 발견될 수 있는 것이다. 따라서 아래의 예와 같이 \CEE/와 \TEX/ 환경을 자연스럽게
왔다갔다할 수 있다.
$$
\displaylines{
\hbox{\.{if} \.{(x==0)} \.{@<배열} \.{|buffer|를} \.{초기화하라@>}} \cr
\hbox{\.{|@<배열} \.{|buffer|를} \.{초기화하라@>|에}
  \.{사용된}  \.{알고리듬을} \.{이용해서} \.{...} }}
$$ 
%The first of these excerpts
%would be found in the \CEE/ part of a section, into which the code
%from the section
%named ``Empty the \\{buffer} array'' is being spliced. The second excerpt
%would be found in the \TEX/ part of the section, and the named section
%is being ``cited'', rather than defined or used.
%(Note the `\pb' surrounding the section name in this case.)
위에서 첫 번째 예는 어떤 마디의 \CEE/ 부분에서 볼 수 있는 것인데, ``배열 \\{buffer}를
초기화하라''라는 이름을 가진 마디에 사용된 코드가 이 부분에 삽입된다. 두 번째 예는 어떤
마디의 \TEX/ 부분에서 발견 할 수 있는 것인데, 그 이름의 마디가 사용이나 정의되는 것이 아니라
``인용''되는 것이다. (이 경우에 마디 이름을 감싸고 있는 `\pb'를 빠뜨리지 않도록 주의한다.)

%\section Macros.
%The control code \.{@d} followed by
%$$\\{identifier}\.{ }\hbox{\CEE/ text}\qquad\hbox{or by}\qquad
%\\{identifier}\.(\\{par}_1,\ldots,\\{par}_n\.{) }\hbox{\CEE/ text}$$
%(where there is no blank between the
%\\{identifier} and the parentheses in the second case) is
%transformed by \.{CTANGLE} into a preprocessor command, starting with
%\.{\#define}, which is printed at the top of the \CEE/ output file
%as explained earlier.
\section 매크로.
\.{CWEB} 명령어, \.{@d} 뒤에는 다음과 같은 것들이 나오는데,
$$\\{\hfontsize{at 9pt}식별자}\.{ }\hbox{\CEE/ 텍스트}\qquad\hbox{또는}\qquad
\\{\hfontsize{at 9pt}식별자}\.(\\{par}_1,\ldots,\\{par}_n\.{) }\hbox{\CEE/ 텍스트}$$
(위의 두 번째의 경우에서 \\{\hfontsize{at 9pt}식별자}와 괄호 사이에는 빈칸이 없다.)
이때, \.{@d}는 \.{CTANGLE}에 의해서 \.{\#define}으로 시작하는 전처리기 명령어로 변환되고,
앞서 설명한 대로, \.{CTANGLE}에 의해서 만들어지는 \CEE/ 파일의 맨 위에 출력된다.

%A `\.{@d}' macro definition can go on for several lines, and the
%newlines don't have to be protected by backslashes, since \.{CTANGLE}
%itself inserts the backslashes.   If
%for any reason you need a \.{\#define} command at a specific spot in
%your \CEE/ file, you can treat it as \CEE/ code, instead of as a
%\.{CWEB} macro; but then you do have to protect newlines yourself.
명령어 `\.{@d}'로 시작하는 매크로 정의는 여러 줄에 걸쳐서 나올 수가 있는데, \.{CTANGLE}이
알아서 각 줄의 맨 뒤에 백슬래쉬를 넣어주므로, \CEE/ 언어에서 하듯이 각 줄의 끝을 백슬래쉬로
마감할 필요가 없다. 만일, 어떠한 이유로 해서 마디의 중간부에서 `\.{@d}' 매크로 정의를 하지
않고 \CEE/ 부분에서 직접 \.{\#define}을 사용해서 매크로 정의를 할 경우에, 그 정의는
\.{CWEB}의 매크로가 아니라, \CEE/ 코드로 취급되므로, 그 경우는 반드시 백슬래쉬로 처리를
해주어야 한다.
 
%\section Strings and constants.
%If you want a string to appear in the \CEE/ file, delimited by pairs of
%\.' or \." marks as usual, you can type it exactly so in the \.{CWEB} file,
%except that the character `\.@' should be typed `\.{@@}' (it becomes a
%control code, the only one that can appear in strings; see below).
%Strings should end on the same line as they begin, unless there's a
%backslash at the end of lines within them.
\section 문자열과 상수.
\CEE/ 파일 내에서 문자열을 나타내고자 하면, 보통 \.'나 \."를 이용하게 된다. \.{CWEB}
파일에서도 마찬가지로 그렇게 하면 된다. 단, 문자 `\.@'는 `\.{@@}'로 해야만 제대로 볼 수
있다. (사실 \.{@@}는 \.{CWEB}의 명령어인데, 문자열 내에서 사용할 수 있는 유일한 명령어이다.
\.{CWEB}의 명령어에 대해서는 아래에서 자세히 살펴볼 것이다.) 문자열은 반드시 한 줄에 있어야
한다. 그렇지 않으면, 줄이 바뀌는 곳에 백슬래쉬를 넣어주어야 한다.

%\TEX/ and \CEE/ have different ways to refer to octal and hex constants,
%because \TEX/ is oriented to technical writing while \CEE/ is oriented to
%computer processing.  In \TEX/ you
%make a constant octal or hexadecimal by prepending \.' or \.",
%respectively, to it; in \CEE/ the constant should be preceded by \.0
%or \.{0x}.  In \.{CWEB} it seems reasonable to let each convention hold
%in its respective realm; so in \CEE/ text you get $40_8$ by typing
%`\.{040}', which \.{CTANGLE} faithfully copies into the \CEE/ file (for
%the compiler's benefit) and which \.{CWEAVE} prints as $\T{\~40}$.
%Similarly, \.{CWEAVE} prints the hexadecimal \CEE/ constant `\.{0x20}'
%as \T{\^20}. The use of italic font for octal digits and typewriter font
%for hexadecimal digits makes the meaning of such constants clearer in
%a document. For consistency, then, you
%should type `\.{|040|}'  or `\.{|0x20|}'
%in the \TEX/ part of the section.
\TEX/은 문서를 만들기 위한 언어이고 \CEE/는 프로그래밍 언어이기 때문에 그 둘이 8진수와
16진수 상수를 다루는 방법에는 약간의 차이가 있다. \TEX/은 \.'와 \."를 이용해서 각각 8진수와
16진수를 표시한다. \CEE/는 \.0과 \.{0x}를 이용한다. \.{CWEB}은 \TEX/과 \CEE/를
모두 포함하고 있기 때문에 각자의 영역에서 각자의 표기법을 사용할 수 있도록 해준다. 그래서
\.{CWEB}의 \CEE/ 텍스트에서 $40_8$을 얻기 위해서 `\.{040}'라고 하면, \.{CTANGLE}은 
그대로 \CEE/ 파일에 복사를 하고, \.{CWEAVE}은 $\T{\~40}$이라고 출력한다. 비슷한 방법으로,
\.{CWEAVE}는 16진수 \CEE/ 상수 `\.{0x20}'을 $\T{\^200}$이라고 출력한다. \.{CWEAVE}은
8진수는 이탤릭체를 사용하고 16진수는 타자체를 사용해서 나타내기 때문에 문서에서 그 둘을 확실히
구분할 수 있다. 일관성을 위해서, 어떤 마디의 \TEX/부에서 8진수와 16진수 상수를 나타낼 때는
`\.{|040|}' 또는 `\.{|0x20|}' 와 같이 입력해야 한다.

%\section Control codes.
%A \.{CWEB} {\sl control code\/}
%is a two-character combination of which the first is `\.@'.
%We've already seen the meaning of several control codes; it's time to
%list them more methodically.
\section 제어 부호.
\.{CWEB}의 {\sl\hfontsize{at 9pt}제어 부호}는 첫 번째 문자가 `\.@'인 두 글자의
결합체이다. 이미 앞에서 여러 제어 부호를 살펴봤는데, 이제 \.{CWEB}의 모든 제어 부호를
좀 더 순차적으로 나열해 볼 시간이 됐다.

%In the following list,
%the letters in brackets after a control code indicate in what contexts that
%code is allowed.  $L$ indicates that the code is allowed in limbo; $T$
%(for \TEX/), $M$ (for middle), and $C$ (for \CEE/) mean that the code is
%allowed in each of the three parts of a section, at top level---that
%is, outside such constructs as `\pb' and section names.  An arrow $\to$
%means that the control code terminates the present part of the \.{CWEB}
%file, and inaugurates the part indicated by the letter following the
%arrow.  Thus $[LTMC\to T]$ next to \.{@\ } indicates that this control
%code can occur in limbo, or in any of the three parts of a section, and
%that it starts the (possibly empty) \TEX/ part of the following section.
앞으로 나열하는 리스트에서 제어 부호 다음에 나오는 대괄호 안의 글자는 그 제어 부호가 어떤
환경에서 사용될 수 있는가를 설명한다. $L$은 명령어가 인림보에서 사용될 수 있다는 것을 뜻하고,
$T$는 \TEX/ 부분에서, $M$는 중간 부분에서, $C$는 \CEE/ 부분과 같은 최상위 수준에서
사용될 수 있다는 것을 나타낸다. 최상위 수준이라 함은, `\pb'나 마디 이름과 같은 환경 이외의
것을 말한다. 화살표 $\to$는 제어 부호를 사용함으로 해서 현재 환경을 마감하고 화살표 다음에
나오는 글자가 뜻하는 새로운 환경이 시작된다는 것을 의미한다. 따라서 제어 부호 \.{@\ } 다음에
나오는 $[LTMC\to T]$가 뜻하는 바는 이 제어 부호는 인림보와 마디의 \TEX/ 부분, 중간 부분,
\CEE/ 부분에서 사용할 수 있고, 이 제어 부호를 사용하면, 새로운 마디의 \TEX/ 부분이
시작된다는 의미이다.

%Two other abbreviations can occur in these brackets: The letter $r$ stands for
%{\it restricted context}, that is, material inside \CEE/ comments, section
%names, \CEE/ strings and control texts (defined below); the letter
%$c$ stands for {\it inner \CEE/ context}, that is, \CEE/ material
%inside `\pb' (including `\pb's inside comments, but not those
%occurring in other restricted contexts).  An asterisk $*$ following
%the brackets means
%that the context from this control code to the matching \.{@>} is
%restricted.
대괄호 안에는 위에서 설명한 것들 외에 두 가지 다른 문자가 나올 수 있는데, 글자 $r$은
{\it\hfontsize{at 9pt}제한된 환경}을 나타내는데, \CEE/ 주석문 안의 환경, 마디 이름,
\CEE/ 문자열, 아래에서 설명될 명령 텍스트 등이 이 환경에 속한다. 글자 $c$는 내부 \CEE/
환경({\it inner \CEE/ context})을 나타내는데, 이에 속하는 환경은 `\pb'안에 있는 \CEE/
텍스트가 그것이다. 이때 `\pb'는 주석문 안에 사용된 `\pb'는 해당하지만, 다른 제한된 환경에서
나타나는 `\pb'는 제외된다. 대괄호 다음에 나오는 $*$는 현재 설명하는 제어 부호는 \.{@>}와
짝을 이루고 그러한 명령어들로 이루어지는 환경이 제한된 환경이라는 뜻이다.
 
%Control codes involving letters are case-insensitive; thus \.{@d} and
%\.{@D} are equivalent. Only the lowercase versions are mentioned
%specifically below.
\.{CWEB} 제어 부호를 구성하는 두 문자중 `\.{@}'를 제외한 나머지 한 문자가 기호가 아닌
알파벳인 명령어들은 \.{CWEB}이 대소문자를 구분하지 않기 때문에 \.{@d}와 \.{@D}는 같은
명령어가 된다. 이 글에서는 소문자만 다룰 것이다.
 
\gdef\@#1[#2] {\penalty-50\yskip\hangindent 2em\noindent\.{@#1\unskip
  \spacefactor1000{ }}$[#2]$\quad}
\def\more{\hangindent 2em \hangafter0}
\def\subsec{\penalty-300\medskip\noindent}

%\@{@} [LTMCrc] A double \.@ denotes the single character `\.@'. This is
%the only control code that is legal everywhere.
%Note that you must use this convention if you are giving an internet
%email address in a \.{CWEB} file (e.g., \.{levy@@math.berkeley.edu}).
\@{@} [LTMCrc] 기호 \.@를 두 번 중복해서 사용하면, 문서에는 `\.@' 하나만 나타난다.
이 제어 부호는 모든 환경에서 사용할 수 있는 유일한 것이며, 주로 \.{CWEB} 파일 내에서 이메일
주소를 나타낼 때 사용된다. (예, \.{levy@@math.berkeley.edu}).

%\subsec
%Here are the codes that introduce the \TEX/ part of a section.
\subsec
다음 제어 부호를 사용하면 마디의 \TEX/ 부분이 시작된다.

%\@\ [LTMC\to T] This denotes the beginning of a new (unstarred)
%section. A tab mark or form feed or
%end-of-line character is equivalent to a space when it follows an \.@
%sign (and in most other cases).
\@\ [LTMC\to T] 이 제어 부호를 사용하면 별표가 아닌 새로운 마디가 시작된다.
빈칸 대신에 탭, 용지 먹임, 줄의 끝 문자 등이 \.@ 다음에
올 수 있고, 그것들은 빈칸과 같은 효과를 나타낸다.

%\@* [LTMC\to T] This denotes the beginning of a new starred
%section, i.e., a section that begins a new major group. The title of the new
%group should appear after the \.{@*}, followed by a period. As explained
%above, \TEX/ control sequences should be avoided in such titles unless
%they are quite simple. When \.{CWEAVE} and \.{CTANGLE} read a \.{@*}, they
%print an asterisk on the terminal
%followed by the current section number, so that the user
%can see some indication of progress. The very first section should be starred.
\@* [LTMC\to T] 이 제어 부호를 사용하면 새로운 별표 마디가 시작된다. 별표 마디는 앞에서
설명했듯이, 여러 마디로 이루어지는 그룹을 대표하는 마디이다. 이때 별표 마디는 제목을 갖는데, 그
제목은 \.{@*} 다음에 나오고 제목 다음에는 반드시 마침표가 나와야 한다. 될 수 있으면 이 제목에
\TEX/의 명령어들은 사용하지 않아야 한다. 그리고 \.{CWEAVE}와 \.{CTANGLE}은 이 명령어를
만나면, 모니터에 $*$와 현재 처리하고 있는 마디 번호를 출력해서 사용자가 모니터를 통하여 어떤
마디가 처리되고 있는지를 알 수 있다. \.{CWEB} 파일에서 가장 먼저 나오는 마디, 즉 1번 마디는
반드시 별표 마디이어야 한다.

%\more You can specify the ``depth'' of a starred section by typing \.* or a
%decimal number after the \.{@*}; this indicates the relative ranking
%of the current group of sections in the program hierarchy. Top-level
%portions of the program, introduced by \.{@**}, get their names typeset
%in boldface type in the table of contents; they are said to have
%depth~$-1$. Otherwise the depth is a nonnegative number, which governs
%the amount of indentation on the contents page. Such indentation helps
%clarify the structure of a long program. The depth is assumed to be 0
%if it is not specified explicitly; when your program is short, you
%might as well leave all depths zero.  A starred section always begins
%a new page in the output, unless the depth is greater than~1.
\more
명령어 \.{@*} 다음에 $*$ 혹은 숫자를 써서 별표 마디의 ``깊이''를 지정할 수 있다. 이 깊이는
전체 프로그램의 구조에 있어서 현재 별표 마디의 상대적 순위를 나타낸다. 프로그램의 최상위 수준을
\.{@**}로 나타내면 그 마디의 제목은 목차에 굵은 글꼴로 표시되고 그 마디의 깊이는 $-1$이 된다.
이를 제외한 다른 마디들의 깊이는 대개 양수인데, 그 숫자는 목차에서 들여쓰기로 양을 나타낸다.
들여쓰기가 많이 되면 될수록 그 마디는 하위 마디라는 뜻이다. 이렇게 마디에 깊이를 두어 목차에서
들여쓰기를 하면, 매우 긴 프로그램의 구조를 보다 명확하게 파악할 수 있다. 프로그램이 짧을 때는
모든 깊이를 0으로 하면 된다. 별표 마디는 언제나 새로운 페이지로 나타나는데, 별표 마디의 깊이를
일부러 1부다 큰 수로 하면 새로운 페이지가 아닌 현재 진행 중인 페이지에 나타난다.
 
%\subsec
%The middle part of each section consists of any number of
%macro definitions (beginning with \.{@d}) and format definitions (beginning
%with \.{@f} or \.{@s}), intermixed in any order.
\subsec
각 마디의 중간 부분에는 여러 개의 매크로 정의, 포멧 정의들이 순서 없이 올 수 있다. 매크로
정의는 \.{@d}로 시작하고 포멧 정의는 \.{@f} 또는 \.{@s}로 시작한다.

%\@{d} [TM\to M] Macro definitions begin with \.{@d}, followed by
%an identifier and optional parameters and \CEE/ text as explained earlier.
\@{d} [TM\to M] 매크로 정의는 \.{@d}로 시작하고 그다음에 식별자가 오고 그다음에 파라미터는
옵션이고 마지막으로 \CEE/ 텍스트가 나온다. 

%\@{f} [TM\to M] Format definitions begin with \.{@f}; they cause
%\.{CWEAVE} to treat identifiers in a special way when they appear in
%\CEE/ text. The general form of a format definition is `\.{@f} \|l
%\|r', followed by an optional comment enclosed between
%\.{/*} and \.{*/}, where \|l and \|r
%are identifiers; \.{CWEAVE} will subsequently treat identifier \|l as it
%currently treats \|r. This feature allows a \.{CWEB} programmer to invent
%new reserved words and/or to unreserve some of \CEE/'s reserved
%identifiers. For example, the common words `error' and `line'
%have been given a special meaning in the \CEE/ preprocessor,
%so \.{CWEAVE} is set up to format them specially; if you want a variable
%named \\{error} or \\{line}, you should say
%$$\.{@f error normal}\qquad\qquad\.{@f line normal}$$
%somewhere in your program.
\@{f} [TM\to M] 포멧 정의는 \.{@f}로 시작한다. 이 명령어는 식별자가 \CEE/ 텍스트에서
사용될 때 특별한 모양으로 출력되도록 \.{CWEAVE}를 조종하는 일을 한다. 포멧 정의의 일반적인
형태는 `\.{@f} \|l \|r'이고, 뒤이어 \.{/*} 와 \.{*/}의 주석문이 올 수 있다.
여기서 \|l과 \|r은 식별자이다. \.{CWEAVE}가 위와 같은 명령어를 만나면, \|l을 \|r
다루듯이 한다. 이 명령어는 \.{CWEB} 프로그래머가 자신만의 데이타 타입을 만들었거나 기존의
\CEE/ 예약어들의 그 기능을 멈추게 하는 등 유용하게 사용될 수 있다.
\.{CWEB}은 `\&{int}'처럼 \CEE/의 예약어를 굵은 글꼴로 표시한다. 따라서 
프로그래머는 자신이 만들어낸 데이타 타입도 \CEE/의 예약어처럼
취급하라고 \.{CWEAVE}에게 알려줄 때, 바로 이 명령어를 사용할 수
있다. 또 `error'나 `line' 같은 단어는 \CEE/의 전처리기에서 특별한
의미가 있으므로, \.{CWEAVE}는 그 단어들을 \CEE/의 예약어와
같은 방식으로 다룰 텐데, \\{error}나 \\{line}을 변수로 사용하고
싶으면, 아래와 같은 방식으로 \.{CWEAVE}에게 알려줘야 한다.
$$\.{@f error normal}\qquad\qquad\.{@f line normal}$$

%\more If \|r is the special identifier `\\{TeX}', identifier \|l
%will be formatted as a \TEX/ control sequence; for example,
%`\.{@f foo TeX}' in the \.{CWEB} file will cause identifier \\{foo} to
%be output as \.{\\foo} by \.{CWEAVE}. The programmer should define
%\.{\\foo} to have whatever custom format is desired, assuming \TEX/
%math mode. (Each underline
%character is converted to \.{x} when making the \TEX/ control sequence,
%and each dollar sign is converted to~\.X;
%thus \\{foo\_bar} becomes \.{\\fooxbar}. Other characters, including digits,
%are left untranslated, so \TEX/ will consider them as macro parameters,
%not as part of the control sequence itself. For example,
%$$\.{\\def\\x\#1\{x\_\{\#1\}\} @f x1 TeX @f x2 TeX}$$
%will format \.{x1} and \.{x2} not as \\{x1} and \\{x2} but as $x_1$ and $x_2$.)
\more 만약에 \|r이 특별한 식별자 `\\{TeX}'이라면, 식별자 \|l%
은 \TEX/의 명령어처럼 취급된다. 예를 들면, \.{CWEB} 파일 내에
`\.{@f foo TeX}'가 쓰이면, 식별자 \\{foo}는 \.{CWEAVE}에
의해서 \TEX/ 파일에 \.{\\foo}로 출력된다. 따라서
프로그래머는 \.{\\foo}가 \TEX/의 수학 모드에서 사용된다고 가정하고
그것의 의미를 정의해야 한다. (위의 명령어를 통해서 \TEX/ 명령어를 만들
때, 밑줄은 \.{x}로 변환되고, 달러 기호는 \.X로
변환되므로 \\{foo\_bar}는 \.{\\fooxbar}로 변환된다. 숫자와 그
밖의 다른 문자들은 아무런 변환 없이 그대로 출력된다. 그래서 그
문자들을 명령어의 일부가 아니라 명령어의 파라미터들로
간주한다. 예를 들어, 아래와 같은 문장은
$$\.{\\def\\x\#1\{x\_\{\#1\}\} @f x1 TeX @f x2 TeX}$$
\.{x1}과 \.{x2}를 \\{x1}과 \\{x2}가 아닌 $x_1$과 $x_2$로 만든다.)

%\more If \|r is the special identifier `\\{make\_pair}', identifier \|l will
%be treated as a \CPLUSPLUS/ function template. For example, after
%\.{@f}~\.{convert}~\.{make\_pair} one can say `\.{convert<int>(2.5)}' without
%having \.< and \.> misunderstood as less-than and greater-than signs.
\more 만약에 \|r이 특별한 식별자 `\\{make\_pair}'라면,
식별자 \|l은 \CPLUSPLUS/ 함수 템플릿 처럼
취급된다. 예를 들어, \.{@f}~\.{convert}~\.{make\_pair} 다음에
`\.{convert<int>(2.5)}'라고 쓸 수 있는데, 이렇게 하면, \.<와 \.>가 수식의
부등호로 오인하는 일 없이 사용할 수 있다.

%\more \.{CWEAVE} knows that identifiers being
%defined with a \&{typedef} should become reserved words; thus you
%don't need format definitions very often.
\more 한 가지 다행인 것은 \.{CWEAVE}는 \&{typedef}로 정의된 식별자는
예약어로 간주하기 때문에 \.{CWEB} 프로그래밍하는데 있어서
포멧 정의를 쓸 일은 그다지 많지 않다.  

%\@{s} [TM\to M;\;L] Same as \.{@f}, but \.{CWEAVE} does not show the format
%definition in the output, and the optional \CEE/ comment is not
%allowed. This is used mostly in \.{@i} files.
\@{s} [TM\to M;\;L] 포멧 정의 명령어 \.{@f}와 같은 기능을 하지만, 이 명령어를
사용하면 \.{CWEAVE}는 정의를 \TEX/ 파일에 출력하지 않는다. 마찬가지로 뒤에 \CEE/의
주석문이 올 수 있다. 주로 \.{@i} 파일들 내에서 사용된다.

%\subsec
%Next come the codes that govern the \CEE/ part of a section.
\subsec
다음으로, 마디의 \CEE/언어부에서 사용되는 명령어들을 설명한다.

%\@{c @p} [TM\to C] The \CEE/ part of an unnamed section begins with \.{@c}
%(or with \.{@p} for ``program''; both control codes do the same thing).
%This causes \.{CTANGLE} to append the following \CEE/ code
%to the first-order program text, as explained on page~\tangref.
%Note that \.{CWEAVE} does not print a `\.{@c}' in the \TEX/
%output, so if you are creating a \.{CWEB} file based on a \TEX/-printed
%\.{CWEB} documentation you have to remember to insert \.{@c} in the
%appropriate places of the unnamed sections.
\@{c @p} [TM\to C] 이름이 없는 마디의 \CEE/언어부는 \.{@c}로
시작한다. (또는 \.{@p}로 시작하는데, 이는 ``프로그램''을 의미하고 이
둘은 같은 의미이다.) \.{CTANGLE}은 이 명령어 다음에 나오는 \CEE/
텍스트들로 \tangref 쪽에서 설명한 대로, 1차 패스 프로그램 텍스트를
만든다. \.{CWEAVE}는 \TEX/ 파일에 `\.{@c}'를 출력하지
않는다. 그래서 \TEX/ 기능을 위주로 하는 \.{CWEB}문서를 기반으로
하는 \.{CWEB} 파일을 만들고자 한다면, 이름 없는 마디의 적당한
위치에 \.{@c}를 넣어야 하는 것을 잊지 말아야 한다.

%\@< [TM\to C;\; C;\; c] $*$ This control code introduces a
%section name (or unambiguous prefix, as discussed above), which
%consists of \TEX/ text and extends to the matching \.{@>}.
%The whole construct \.{@<...@>} is conceptually a \CEE/ element.
%The behavior is different depending on the context:
\@< [TM\to C;\; C;\; c] $*$ 이 명령어 다음에는 마디 이름 (또는 앞에서
설명한 대로 다른 것과 구별할 수 있는 최소한의 접두어)이 온다. 마디
이름은 \TEX/ 텍스트가 오고 \.{@>}로 끝을 맺는다. 전체 \.{@<...@>}는
개념적으로는 \CEE/ 요소이다. 이 명령어가 하는 일은 주어진 환경에 따라
달라진다. 

%\more A \.{@<} appearing in contexts $T$ and $M$ attaches the
%following section name to the current section, and inaugurates the
%\CEE/ part of the section.  The closing \.{@>} should be followed by
%\.{=} or \.{+=}.
\more \.{@<}가 $T$와 $M$ 환경에서 사용될 때, 이 명령어는 뒤따라 나오는
마디 이름을 현재 마디에 덧붙이고, 그 마디의 \CEE/언어부를 시작하도록 한다.
\.{@>} 다음에는 반드시 \.{=} 또는 \.{+=}이 나와야 한다.  

%\more
%In context $C$, \.{@<} indicates that the named
%section is being used---its \CEE/ definition is spliced in by
%\.{CTANGLE}, as explained on page~\tangref.
%As an error-detection measure,
%\.{CTANGLE} and \.{CWEAVE} complain if such a section name is followed
%by \.=, because most likely this is meant as the definition of a new
%section, and so should be preceded by \.{@\ }.  If you really want to
%say $\langle\,$foo$\,\rangle=\\{bar}$, where $\langle\,$foo$\,\rangle$
%is being used and not defined, put a newline before the \.=.
\more
$C$ 환경에서, \.{@<}는 이름 있는 마디가 사용된다는 것을
가리킨다.---\tangref쪽에 설명된 대로, \.{CTANGLE}에 의해서
그것의 \CEE/ 코드가 삽입된다. 에러를 발견하기 위한
수단으로, 그러한 마디 이름 다음에 \.=가
나오면, \.{CTANGLE}과 \.{CWEAVE}는 에러 메시지를 낸다. 왜냐하면 \.=이
나온다는 것은 새로운 마디가 정의된다는 의미이기
때문이다. `\.{@\ @<...@>=}'는
가능하다. $\langle\,$foo$\,\rangle=\\{bar}$와 같은 문장을 반드시
사용해야 한다면, \.= 전에서 한 줄 띄워야 한다. 여기서
$\langle\,$foo$\,\rangle$는 정의되는 것이 아니라 사용되는 것이다.

%\more
%Finally, in inner \CEE/ context (that is, within `\pb' in the \TEX/ part
%of a section or in a comment), \.{@<...@>}
%means that the named section is being
%cited.  Such an occurrence is ignored by \.{CTANGLE}. Note that
%even here we think of the section name as being a \CEE/ element, hence the \pb.
\more
마지막으로, 내부 \CEE/ 환경 (즉, 마디의 \TEX/부에 있는 `\pb' 내의
환경이나, 주석문) 에서 \.{@<...@>}는 이름 가진 마디가 인용된다는 것을
의미한다. \.{CTANGLE}는 이와 같은 것들을 무시한다.

%\@( [TM\to C;\;C;\;c] $*$ A section name can begin with \.{@(}.
%Everything works just as for \.{@<}, except that the \CEE/ code
%of the section named \.{@(foo@>} is written by \.{CTANGLE}
%to file \.{foo}. In this way you can get multiple-file output from
%a single \.{CWEB} file. (The \.{@d} definitions are not output
%to such files, only to the master \.{.c} file.) One use of this feature
%is to produce header files for other program modules that will be loaded
%with the present one. Another use is to produce a test routine that
%goes with your program. By keeping the sources for a program and its
%header and test routine together, you are more likely to keep
%all three consistent with each other. Notice that the output of a named
%section can be incorporated in several different output files, because
%you can mention \.{@<foo@>} in both \.{@(bar1@>} and \.{@(bar2@>}.
\@( [TM\to C;\;C;\;c] $*$ 마디 이름은 \.{@(}로도 시작할 수 있다. 이
명령어가 하는 일은 한 가지만 빼고 \.{@<}가 하는 일과 같다.
\.{CTANGLE}은 \.{@(foo@>}라는 이름을 가진 마디의 \CEE/ 코드들을
\.{foo}라는 이름의 파일로 출력한다는 것이다. 이 방법을 통해서 하나의
\.{CWEB}에서 여러 개의 파일을 얻을 수 있다. (주의할 것은 \.{@d}
정의들은 그러한 파일로 출력돼지 않고, 오로지 \.{CWEB} 파일이
\.{CTANGLE}에 의해서 변환되는 \.{.c} 파일에만 출력된다는 것이다.) 이
방법이 유용하게 사용되는 한가지 경우는 바로 현재 작성 중인 \.{CWEB} 파일과
함께 로드될 다른 프로그램 모듈들을 위한 헤더 파일들을 만드는 경우이다.
또 다른 경우는 주어진 프로그램의 테스트 코드를 만들어내는
경우이다. 하나의 \.{CWEB} 파일로 주요 소스 파일을 물론 헤더 파일들과 테스트
코드를 관리할 수 있다는 것은 매우 유용하고 편리한 기능이다.  
\.{@(bar1@>}와 \.{@(bar2@>}가 동시에 \.{@<foo@>}를 사용할 수 있기 때문에,
이름을 가진 하나의 마디가 다른 여러 개의 출력 파일들에 속할 수 있다는 것을
명심해야 한다.   

%\@{h} [Cc] Causes \.{CTANGLE}
%to insert at the current spot the \.{\#define} statements
%from the middle parts of all sections,
%and {\it not\/} to write them at the beginning of the \CEE/ file.
%Useful when you want the macro definitions to come after the include files,
%say. (Ignored by \.{CTANGLE} inside `\pb'.)
\@{h} [Cc] 이 명령어를 사용하면 \.{CTANGLE}은 모든 마디의 중간부에서
정의된 매크로들을 이 명령어가 사용된 지점에 \.{\#define} 문장들로
변환해서 \CEE/ 파일의 맨 처음에 기록된다. 이
명령어는 매크로 정의들을 헤더 파일들 다음에 위치시키고 싶을 때 헤더
파일들 다음에 이 명령어를 사용하면 된다

%\subsec
%The next several control codes introduce ``control
%texts,'' which end with the next `\.{@>}'.  The closing `\.{@>}' must be on
%the same line of the \.{CWEB} file as the line where the control text began.
%The context from each of these control codes to the matching \.{@>} is
%restricted.
\subsec
다음에 설명할 명령어들은 ``명령 텍스트''라는 것을
소개하는데, 명령 텍스트는 
명령어와 `\.{@>}' 사이에 있는 텍스트를 말한다. 기억할 것은
`\.{@>}'은 반드시 명령어와 같은 줄에 있어야 한다는 것이다. 
명령 텍스트의 환경은 제한된 환경이 된다. 

%\@\^ [TMCc] $*$ The control text that follows, up to the next
%`\.{@>}', will be entered into the index together with the identifiers of
%the \CEE/ program; this text will appear in roman type. For example, to
%put the phrase ``system dependencies'' into the index that is output by
%\.{CWEAVE}, type
%`\.{@\^system dependencies@>}' in each section
%that you want to index as system dependent.
\@\^ [TMCc] $*$ 이 명령어의 명령 텍스트는 \CEE/
프로그램에서 사용되는 식별자들처럼 색인 리스트에 들어가게 한다. 이 텍스트는 색인에서
로마체로 나타난다. 예를 들어, ``system dependencies''를 색인에
넣으려면, `\.{@\^system dependencies@>}' 라고 입력하면 된다. 

%\@. [TMCc] $*$ The control text that follows will be entered into the index
%in \.{typewriter} \.{type}.
\@. [TMCc] $*$ 이 명령어의 명령 텍스트는 색인에 \.{타자 글꼴}로 나타난다.

%\@: [TMCc] $*$ The control text that follows will be entered into the index
%in a format controlled by the \TEX/ macro `\.{\\9}', which you
%should define as desired.
\@: [TMCc] $*$ 이 명령어의 명령 텍스트는 색인에 \TEX/ 매크로
`\.{\\9}'가 정하는 형태대로 나타나는데, 매크로 \.{\\9}는 반드시 사전에
정의되어 있어야 한다.

%\@{t} [MCc] $*$ The control text that follows will
%be put into a \TEX/ \.{\\hbox} and formatted along with the neighboring
%\CEE/ program. This text is ignored by \.{CTANGLE}, but it can be used
%for various purposes within \.{CWEAVE}. For example, you can make comments
%that mix \CEE/ and classical mathematics, as in `$\\{size}<2^{15}$', by
%typing `\.{|size < 2@t\$\^\{15\}\$@>|}'.
\@{t} [MCc] $*$ 이 명령어의 명령 텍스트는 \TEX/의 \.{\\hbox}에
들어가서 그와 이웃하는 \CEE/ 프로그램과 어울려 쓰인다. \.{CTANGLE}은
이 명령 텍스트를 완전히 무시하지만, \.{CWEAVE}은 다양한 방법으로 이
명령어를 유용하게 사용한다. 예를 들어, 이 명령어를 이용해서 \CEE/ 코드와 수식이
어우러 진 `$\\{size}<2^{15}$'와 같은 멋진 주석문을 만들 수 있는데, 이
주석문은 `\.{|size < 2@t\$\^\{15\}\$@>|}'를 입력하면 얻을 수 있다.

%\@= [MCc] $*$ The control text that follows will
%be passed verbatim to the \CEE/ program.
\@= [MCc] $*$ 이 명령어의 명령 텍스트는 \CEE/ 프로그램에 글자 그대로 복사된다.  

%\@{q} [LTMCc] $*$ The control text that follows will
%be totally ignored---it's a comment for readers of the \.{CWEB} file only.
%A file intended to be included in
%limbo, with \.{@i}, can identify itself with \.{@q} comments.
%Another use is to balance unbalanced parentheses in C strings,
%so that your text editor's parenthesis matcher doesn't go into a tailspin.
\@{q} [LTMCc] $*$ 명령 텍스트는 오로지
\.{CWEB} 파일을 읽는 사람에게만 해당하는 주석이어서 \.{CTANGLE}과
\.{CWEAVE}은 이 명령 텍스트를 완전히 무시한다.
이 명령어는 \.{@i}를 이용해서 인림보에 포함될 파일들을 주로 \.{@q}를
이용해서 주석 처리하는 경우에 사용된다. 다른 한가지 예는 \CEE/
문자열에서 균형이 맞지 않는 괄호의 균형을 맞출 때 쓰인다. 이 경우 괄호를
매치시키는 기능을 제공하는 편집기가 괄호의 균형이 맞지 않을 때 이상하게
동작하는 것을 막을 수 있다. 
 
%\@! [TMCc] $*$
%The section number in an index entry will be underlined if `\.{@!}'
%immediately precedes the identifier or control text being indexed. This
%convention is used to distinguish the sections where an identifier is
%defined, or where it is explained in some special way, from the sections
%where it is used. A~reserved word or an identifier of length one will not
%be indexed except for underlined entries. An `\.{@!}' is implicitly inserted
%by \.{CWEAVE} when an identifier is being defined or declared in \CEE/
%code; for example, the definition
%$$\hbox{\&{int} \\{array}[\\{max\_dim}], \\{count}${}=\\{old\_count};$}$$
%makes the names \\{array} and \\{count} get an underlined entry in the
%index.  Statement labels, function definitions like
%\\{main}(\&{int}~\\{argc},\,\&{char}~$*$\\{argv}[\,]),
%and \&{typedef} definitions also
%imply underlining. An old-style
%function definition (without prototyping) doesn't define its arguments;
%the arguments will, however, be considered to be defined
%(i.e., their index entries will be underlined) if their types are
%declared before the body of the function in the usual way
%(e.g., `\&{int}~\\{argc}; \&{char}~${*}\\{argv}[\,]$; $\{\,\ldots\,\}$').
%Thus \.{@!} is not needed very often, except in unusual constructions
%or in cases like
%$$\.{enum boolean \{@!false, @!true\};}$$
%here \.{@!} gives the best results because individual constants enumerated
%by \.{enum} are not automatically underlined in the index at their
%point of definition.
\@! [TMCc] $*$
식별자나 명령 텍스트 바로 앞에 `\.{@!}'를 붙이면, 그 식별자나 명령
텍스트가 색인에 들어갈 때, 그것들이 사용된 마디 번호에 밑줄이 그어진다.
이 명령어는 식별자가 정의된 마디인지, 사용된 마디인지를 구별할 경우에
사용된다. 한 글자 짜리 예약어나 식별자는 색인에 들어가지 않는데, 한 글자
짜리라고 하더라도 밑줄이 그어진 식별자는 색인에 들어간다.
어떤 식별자가 \CEE/ 코드에서 정의되거나 선언될 때, \.{CWEAVE}는
자동으로 `\.{@!}'를 붙여준다. 예를 들어, 다음과 같은 정의를 하면,
$$\hbox{\&{int} \\{array}[\\{max\_dim}],
\\{count}${}=\\{old\_count};$}$$
변수명 \\{array}와 \\{count}는 자동적으로 밑줄이 그어져서 색인에
들어간다. 
\\{main}(\&{int}~\\{argc},\,\&{char}~$*$\\{argv}[\,])와 같은 함수
정의, \&{typedef} 정의들도 \.{CWEAVE}가 알아서 자동으로 밑줄을
그어준다. 프로토타입이 없는 구식 함수 정의는 함수의 인자들을 정의하지
않지만 만약에 그 타입들이 함수의 몸체 전에 `\&{int}~\\{argc};
\&{char}~${*}\\{argv}[\,]$; $\{\,\ldots\,\}$' 처럼 보통의 방식으로
정의되었다면, 색인에 밑줄이 그어져서 들어간다.
따라서, 흔히 사용되는 식별자의 정의나,
$$\.{enum boolean \{@!false, @!true\};}$$
위와 같은 경우를 제외하면, \.{@!}를 자주 사용할 일을 없을
것이다. 위 경우 \.{@!}를 사용하는 것을 매우 권장하는데, 이는 열거형
자료형 정의는 색인에 들어갈 때 기본적으로 밑줄이 그어지지 않기 때문이다.

%\subsec
%We now turn to control codes that affect only the operation of
%\.{CTANGLE}.
\subsec
다음으로 \.{CTANGLE}만이 사용하는 명령어를 살펴보자.

%\@' [MCc] This control code is dangerous because it has quite different
%meanings in \.{CWEB} and the original \.{WEB}. In \.{CWEB} it produces the
%decimal constant corresponding to the ASCII code for a string of length~1
%(e.g., \.{@'a'} is \.{CTANGLE}d into \.{97} and \.{@'\\t'} into
%\.9). You might want to use this if you need to work in ASCII on a
%non-ASCII machine; but in most cases the \CEE/ conventions of
%\.{<ctype.h>} are adequate for character-set-independent programming.
\@' [MCc] 이 명령어는 \.{CWEB}과 원래의 \.{WEB}에서 전혀 다르게
동작하기 때문에 이 명령어 사용에는 주의를 기울여야
한다. \.{CWEB}에서는, 문자열의 길이가 1인 문자열의 ASCII 코드에
해당하는 10진수를 만들어낸다. (예, \.{@'a'}는 \.{97}로, \.{@'\\t'}는
\.9로 \.{CTANGLE}된다.) 이 명령어는 비 ASCII 컴퓨터에서 ASCII로
작동하는 코드를 만들고자 할 때 사용될 것이다. 그러나 대부분은
\.{<ctype.h>}에서 문자셋에 무관하게 프로그래밍하도록 돼 있어서 이
명령어를 사용할 일은 드물다.

%\@\& [MCc] The \.{@\&} operation causes whatever is on its left to be
%adjacent to whatever is on its right, in the \CEE/ output. No spaces or
%line breaks will separate these two items.
\@\& [MCc] 이 명령어는 이 명령어의 좌우에 있는 것을 붙여서 하나의
\CEE/ 코드를 만들어낸다. 어떠한 공백들이나 줄 바꿈들도 이
명령어로 결합한 것을 갈라놓지 못한다.

%\@{l} [L] \.{CWEB} programmers have the option of using any 8-bit character
%code from the often-forbidden range 128--255 within \TEX/ text; such
%characters are also permitted in strings and even in identifiers of the
%\CEE/ program.
%Under various extensions of the basic
%ASCII standard, the higher 8-bit codes correspond
%to accented letters, letters from non-Latin alphabets,
%and so on. When such characters occur in identifiers, \.{CTANGLE} must replace
%them by standard ASCII alphanumeric characters or
%\.{\_}, in order to generate legal \CEE/ code.  It does this by means
%of a transliteration table, which by default associates the string
%\.{Xab} to the character with ASCII code \T{\^}$ab$ (where $a$ and $b$ are
%hexadecimal digits, and $a\ge8$).  By placing the
%construction \.{@l\ ab\ newstring} in limbo, you are telling
%\.{CTANGLE} to replace this character by \.{newstring} instead.
%For example, the ISO Latin-1 code for the letter `\"u' is \T{\^FC}
%(or \.{'\char`\\374'}),
%and \.{CTANGLE} will normally change this code to the
%three-character sequence \.{XFC} if it
%appears in an identifier. If you say \.{@l} \.{fc} \.{ue}, the code will
%be transliterated into \.{ue} instead.
\@{l} [L] \.{CWEB} 프로그래머들은 \TEX/ 텍스트 안에서 눈에 보이지 않는
범위인 125--255 중의 임의의 8-비트 문자 코드를 사용할 수 있는 옵션을
가지고 있다. 그러한 눈에 보이지 않는 코드를 문자열에서 사용할 수 있고,
심지어는 \CEE/ 프로그램의 식별자로도 사용할 수 있다. 기본 아스키 표준의
다양한 확장 하에서 위 범위의 8-비트 코드들은 엑센트 글자, 비-라틴 알파벳 글자들 등과 대응된다.
그러한 문자들이
식별자로 쓰이면, 올바른 \CEE/ 코드를 생성하기 위해서 \.{CTANGLE}은
그들의 표준 아스키 숫자 혹은 알파벳 코드나 \.{\_}로 변환해야 한다. 이러한
변환은 변환표를 통해서 이루어진다. 변환표는
기본적으로 문자열 \.{Xab}를 아스키 코드 \T{\^}$ab$에
결합시킨다. (여기서 $a$와 $b$는 16진수 숫자이고, $a\ge8$이다.) 
인림보에서 \.{@l\ ab\ newstring} 처럼 정의하면, \.{CTANGLE}에게 이
문자를 \.{newstring}으로 바꾸라고 명령을 하게 되는 것이다. 예를 들어, 
글자 `\"u'의 ISO Latin-1 코드는 \T{\^FC} (또는 \.{'\char`\\374'})
이고, \.{CTANGLE}은 이 코드가 식별자로 쓰였다면, 세 글자짜리 \.{XFC}로
바꾼다. 만일 \.{@l} \.{fc} \.{ue}라고 하면, 그 코드는 대신에 \.{ue}로
변환된다. 

%\more
%\.{CWEAVE} passes 8-bit characters straight through to \TEX/ without
%transliteration; therefore \TEX/ must be prepared to receive them.
%If you are formatting all your nonstandard identifiers as ``custom''
%control sequences, you should
%make \TEX/ treat all their characters as letters. Otherwise you should either
%make your 8-bit codes ``active'' in \TEX/, or load fonts that
%contain the special characters you need in the correct positions.
%(The font selected by \TEX/ control sequence \.{\\it} is used for
%identifiers.)
%Look for special macro packages designed for \.{CWEB} users in your language;
%or, if you are brave, write one yourself.
\more
\.{CWEAVE}는 8-비트 문자들을 아무런 변환 없이 그대로 \TEX/에게 보낸다.
그러므로 \TEX/은 반드시 그것들을 받을 준비가 되어있어야 한다.
만일 당신의 모든 비표준 식별자를 자신 고유의 명령어로 다루고자 한다면,
\TEX/으로 하여금 그 모든 비표준 문자를 글자로 인식하게 하여야 한다.
그렇지 않으면, 당신만의 8-비트 코드들을 \TEX/의 ``액티브'' 코드로
만들던가, 당신이 필요로 하는 특별 문자를 가지고 있는 글꼴을 읽어들여야 한다.
(\TEX/ 명령어 \.{\\it}로 선택된 글꼴은 식별자를 나타내는 글꼴로
사용된다.) 당신이 사용하는 언어로 작성된 \.{CWEB} 사용자를 위한 매크로
패키지가 있는지 확인해 보라. 만약, 그러한 매크로 패키지가 없고, 당신이
용감하다면 직접 만들어 보라.

%\subsec
%The next eight control codes (namely `\.{@,}', `\.{@/}', `\.{@|}', `\.{@\#}',
%`\.{@+}', `\.{@;}', `\.{@[}', and `\.{@]}') have no effect on the \CEE/
%program output by \.{CTANGLE}; they merely help to improve the readability
%of the \TEX/-formatted \CEE/ that is output by \.{CWEAVE}, in unusual
%circumstances. \.{CWEAVE}'s built-in formatting method is fairly good
%when dealing with syntactically correct \CEE/ text, but
%it is incapable of handling all possible cases, because it must deal with
%fragments of text involving macros and section names; these fragments do
%not necessarily obey \CEE/'s syntax. Although \.{CWEB} allows you to
%override the automatic formatting, your best strategy is not to worry
%about such things until you have seen what \.{CWEAVE} produces automatically,
%since you will probably need to make only a few corrections when you are
%touching up your documentation.
\subsec
다음에 설명할 명령어 여덟 개(이름하여, `\.{@,}', `\.{@/}', `\.{@|}', `\.{@\#}',
`\.{@+}', `\.{@;}', `\.{@[}', `\.{@]}')는 \.{CTANGLE}이 만들어내는
\CEE/ 프로그램에는 아무런 영향도 미치지 못하는 것들이다. 이 명령어들은
특수한 환경하에서 단순히 \.{CWEAVE}가 만들어낸 \CEE/ 코드 문서를 보기
좋게 하는데 쓰이는 명령어들이다.
\.{CWEAVE}에 내장된 문서 포멧 도구는 매우 좋아서 문법적으로 올바른
\CEE/ 텍스트를 잘 다룰 수 있다. 하지만, 모든 경우를 다 커버하는 것은
아니다. 왜냐하면, 매크로나 마디 이름에 포함된 텍스트의 일부분도 다룰 수
있어야 하는데, 그 일부분은 \CEE/의 문법을 따르지 않을 수도 있기 때문이다.
비록 \.{CWEB}이 자동 포멧을 당신의 구미에 맞게 수정할 수 있는 도구를
제공하긴 하지만, 수정하지 말고 일단 \.{CWEAVE}가 자동으로 처리하는
것을 자세히 지켜보라. 그러면 대부분은 \.{CWEAVE}가 놀랄 정도로
당신이 원하는 대로 올바르게 처리할 것이므로, 당신이 직접 수정해서
새로운 포멧 도구를 만들 일은 거의 없을 것이다.

%\@, [MCc] This control code inserts a thin space in \.{CWEAVE}'s output.
%Sometimes you need this extra space if you are using
%macros in an unusual way, e.g., if two identifiers are adjacent.
\@, [MCc] 이 명령어는 \.{CWEAVE}가 만들어낸 파일 안에 작은 간격을 넣는다.
이 간격은 예를 들어, 두 식별자가 서로 붙어 있는 등 일반적이지 않은
방식으로 매크로들을 사용할 때 필요하다. 

%\@/ [MC] This control code causes a line break to occur within a \CEE/
%program formatted by \.{CWEAVE}. Line breaks
%are chosen automatically by \TEX/ according to a scheme that works 99\%\
%of the time, but sometimes you will prefer to force a line break so that
%the program is segmented according to logical rather than visual
%criteria. If a comment follows, say `\.{@/@,}' to break the line
%before the comment.
\@/ [MC] 이 명령어는 \.{CWEAVE}가 만들어낸 \CEE/ 코드 문서에서 강제로
줄 바꿈을 할 경우에 사용한다. 줄 바꿈은 99\% \TEX/이 알아서 올바르게
자동으로 해줄 것이나, 당신은 때대로 프로그램의 논리상 줄 바꿈을 하면
프로그램을 더욱 이해하기 쉬운 곳이 있을 것이다. 그때에 이 명령어를
사용하면 된다. 만약에 줄 바꿈 하고자 하는 지점 바로 뒤에 주석문이
나온다면, 주석문 전에서 줄 바꿈을 하기 위해서는 `\.{@/@,}'를 입력해야 한다.

%\@| [MC] This control code specifies an optional line break in the midst of
%an expression. For example, if you have
%a long expression on the right-hand side of an assignment
%statement, you can use `\.{@|}' to specify breakpoints more logical than
%the ones that \TEX/ might choose on visual grounds.
\@| [MC] 이 명령어는 표현식 한가운데서 선택적으로 줄 바꿈을 할 때
사용한다. 예를 들어, 대입문의 우변이 매우 긴 경우에 이 명령어 \.{@|}를
써서 줄바꿈을 하여 \TEX/이 규칙에 의해서 만들어내는 것보다 보기 좋고
프로그램의 논리 흐름으로는 더 좋은 문서를 만들 수 있다.

%\@\# [MC] This control code forces a line break, like \.{@/} does,
%and it also causes a little extra white space to appear between the lines at
%this break. You might use it, for example,
%between groups of macro definitions that are logically separate but within
%the same section. \.{CWEB} automatically inserts this extra space
%between functions, between external declarations and functions, and
%between declarations and statements within a function.
\@\# [MC] 이 명령어는 \.{@/} 처럼 강제적으로 줄 바꿈할 때
사용한다. 한 가지 다른 점은 줄 바꿈한 후에 그 지점에 약간의 간격을 둔다는
것이다. 예를 들어, 프로그램 구조상 의미 있는 것들을 그룹으로 하여
하나의 코드 무리를 만들 수 있는데, 각 그룹 사이에 약간의 간격이
있다면, 프로그램 소스를 보기도 좋고 이해하기도 쉬울 것이다. 관례상 전역
변수 선언부와 함수 선언부 사이에, 선언과 실행문 사이에 조금이
필요한데, 이때 이 명령어를 쓰면 된다.

%\@+ [MC] This control code cancels a line break that might otherwise be
%inserted by \.{CWEAVE}, e.g., before the word `\&{else}', if you want to
%put a short if--else construction on a single line.
%If you say `\.{\{@+}' at the beginning of a compound statement
%that is the body of a function, the first declaration or
%statement of the function will appear on the same line as the
%left brace, and it will be indented by the same amount as the
%second declaration or statement on the next line.
\@+ [MC] 이 명령어는 특별한 주문이 없는 한, \.{CWEAVE}가 알아서
자동으로 줄 바꿈하는 자리, 예를 들어, `\&{else}' 바로 앞 지점 같은
자리에 줄 바꿈이 일어나지 못하도록 한다. if--else 구문을 한 줄에
사용하고 싶은 경우에 사용하면 된다. 만약에 함수의 몸체가 시작되는
부분에 `\.{\{@+}'라고 입력하면, 함수 몸체의 첫 번째 문장은 이 왼쪽
괄호와 같은 줄에 생기고, 들여쓰기 정도는 다음 줄의 문장과 같은 만큼
들여쓰기 된다.

%\@; [MC] This control code is treated like a semicolon, for formatting
%purposes, except that it is invisible. You can use it, for example, after
%a section name or macro when the \CEE/ text represented by that section or macro
%is a compound statement or ends
%with a semicolon. Consider constructions like
%$$\lpile{\.{if (condition) macro @;}\cr
%\.{else break;}\cr}$$
%where \\{macro} is defined to be a compound statement (enclosed in braces).
%This is a well-known infelicity of \CEE/ syntax.
\@; [MC] 이 명령어는 문서를 보기 좋게 하는 목적으로 세미 콜론과 같이
취급되는데, 다른 점은 세미 콜론이 눈에 보이지 않는다는 것이다. 예를
들어, 당신은 이 명령어를 \CEE/ 코드 안에서 사용되는 마디 이름 뒤나 매크로
다음에 사용할 수 있다. 이때 마디나 매크로는 여러 줄로 구성된 것이면 이
명령어를 사용하는 의미가 더욱 확실하겠다. 다음과 같은 문장을 생각해 보자.
$$\lpile{\.{if (condition) macro @;}\cr
\.{else break;}\cr}$$
위에서 \\{macro}는 괄호로 둘러싸인 여러 문장으로 된 매크로라고 하자. 여러
문장으로 된 매크로가 사용될 때, 세미콜론을 붙이면 \CEE/ 문법에 맞지
않는데, 이 명령어를 이용하면 그것을 해결할 수 있다.

%\@{[} [MC] See \.{@]}.
\@{[} [MC] \.{@]}를 보라.

%\@] [MC] Place \.{@[...@]} brackets around program text that \.{CWEAVE} is
%supposed to format as an expression, if it doesn't already do so. (This
%occasionally applies to unusual macro arguments.) Also
%inser `\.{@[@]}' between a simple type name and a left parenthesis
%when declaring a pointer to a function, as in
%$$\.{int @[@] (*f)();}$$
%otherwise \.{CWEAVE} will confuse the first part of that declaration with
%the \CPLUSPLUS/ expression `\&{int}($*f$)'. Another example, for people
%who want to use low-level \.{\#define} commands in the midst of \CEE/ code
%and the definition begins with a cast:
%$$\.{\#define foo @[(int)(bar)@]}$$
\@] [MC] \.{CWEAVE}가 표현 식으로 사용할 프로그램 텍스트
주위에 \.{@[...@]}로 둘러싸라. (이 명령어는 경우에 따라서 비정상적인
매크로 인자에 적용되기도 한다.) 또는 아래와 같이, `\.{@[@]}'를 함수의
포인터를 선언하는 경우에 자료형 이름과 왼쪽 괄호 사이에 넣어라.
$$\.{int @[@] (*f)();}$$
만약 그렇게 하지 않으면, \.{CWEAVE}는 위 선언문을 \CPLUSPLUS/ 표현식
`\&{int}($*f$)'의 첫 번째 부분과 혼동할 것이다. 또 다른 예는, \CEE/ 코드
중간에 저수준 명령 \.{\#define}을 사용하고자 할 때, 그리고 정의가
캐스트로 시작할 때 이 명령어를 사용하라.
$$\.{\#define foo @[(int)(bar)@]}$$

%\subsec
%The remaining control codes govern the input that \.{CWEB} sees.
\subsec
남은 명령어들은 \.{CWEB} 입력을 처리하는 명령어들이다.

%\@{x @y @z}[\\{change\_file}]
%\.{CWEAVE} and \.{CTANGLE} are designed to work with two input files,
%called \\{web\_file} and \\{change\_file}, where \\{change\_file} contains
%data that overrides selected portions of \\{web\_file}. The resulting merged
%text is actually what has been called the \.{CWEB} file elsewhere in this
%report.
\@{x @y @z}[\\{change\_file}]
\.{CWEAVE}와 \.{CTANGLE}은 \\{web\_file}과 \\{change\_file}이라고
불리는 두 종류의 입력 파일을 다룰 수 있도록 디자인되었다. 여기서
\\{change\_file}은 \\{web\_file}에서 변경할 부분을 대체할 자료들이 있다.
이 두 파일을 합하여 생긴 파일이 사실상 \.{CWEB}
파일이라고 불리는 것이다.

%\more Here's how it works: The change file consists of zero or more
%``changes,''
%where a change has the form `\.{@x}$\langle$old lines$\rangle$\.{@y}$\langle$%
%new lines$\rangle$\.{@z}'. The special control codes \.{@x}, \.{@y}, \.{@z},
%which are allowed only in change files, must appear at the beginning of
%a line; the remainder of such a line is ignored.
%The $\langle$old lines$\rangle$ represent material that exactly matches
%consecutive lines of the \\{web\_file}; the $\langle$new lines$\rangle$
%represent zero or more lines that are supposed to replace the old. Whenever
%the first ``old line'' of a change is found to match a line in the
%\\{web\_file}, all the other lines in that change must match too.
\more 이 명령어가 동작하는 방식은 다음과 같다. 수정 파일은 0개 이상의 수정 항목을 가지고
있는데, 수정 항목은
`\.{@x}$\langle$old lines$\rangle$\.{@y}$\langle$new
lines$\rangle$\.{@z}'와 같은 형태를 가지고 있다. 특수 명령어 \.{@x},
\.{@y}, \.{@z}는 오로지 수정 파일에서만 쓰일 수 있는데, 반드시 줄의
첫 번째 칼럼부터 시작되어야 하고, 그 줄에서 위 형식이 갖추어지고 난 나머지는 무시된다.
$\langle$old lines$\rangle$은 \\{web\_file} 내에 있는 줄들과 정확히
일치해야 하고, $\langle$new lines$\rangle$은 0개 이상의 줄로
구성되는데, 앞의 old를 대체할 줄들로 구성된다. 수정 항목의 ``old
line''의 첫 번째 줄과 정확히 일치하는 \\{web\_file}의 줄이 발견될 때마다,
수정 항목의 나머지 다른 줄들도 반드시 일치하여야 한다.

%\more Between changes, before the first change, and after the last change,
%the change file can have any number of lines that do not begin with
%`\.{@x}', `\.{@y}', or~`\.{@z}'. Such lines are bypassed and not used for
%matching purposes.
\more 수정 파일은 수정 항목들 사이, 첫 번째 수정 항목 앞, 마지막 수정 항목 뒤에,
`\.{@x}', `\.{@y}', 또는~`\.{@z}'로 시작하지 않는 임의의
개수의 줄을 가질 수 있다. 그러한 줄들은 설명이나 주석을 담고 있어서 
그냥 무시되는 줄이지, 일치하는 목적으로 사용되는 줄들이 아니다.

%\more This dual-input feature is useful when working with a master \.{CWEB}
%file that has been received from elsewhere (e.g., \.{tangle.w} or
%\.{weave.w} or \.{tex.web}), when changes are desirable to customize the
%program for your local computer system. You will be able to debug your
%system-dependent changes without clobbering the master web file; and once
%your changes are working, you will be able to incorporate them readily
%into new releases of the master web file that you might receive from time
%to time.
\more 이러한 두 종류의 파일을 입력으로 취하는 특징은 다음과 같은
경우에 사용하면 적당하다. 우선 \.{tangle.w} 또는 \.{weave.w} 또는
\.{tex.web}와 같은 마스터 \.{CWEB} 파일이 있고, 이 파일
자체를 수정하고 않고, 현재 로컬 시스템에서 \.{CWEB} 파일로 만들어진
실행 파일이 동작하도록 하기 위해서 \.{CWEB} 파일의 일부가 변경될
필요가 있을 때, 그 변경 부분만을 골라내어 수정 파일을 만드는 경우이다.
이렇게 하면, 로컬 시스템에 맞게 변경된 부분을 디버깅할 때, 마스터
웹 파일을 건드리지 않고 작업할 수도 있다. 그래서 일단, 수정 파일이
제대로 동작한다는 것이 확인되면, 자주 새롭게 배포되는 마스터 웹 파일에
간단하게 이 수정 파일을 접목시킬 수 있다.

%\@{i} [\\{web\_file}]
%Furthermore the \\{web\_file} itself can be a combination of
%several files.  When either \.{CWEAVE} or \.{CTANGLE} is reading a file and
%encounters the control code \.{@i} at the beginning of a line, it
%interrupts normal reading and starts looking at the file named after the
%\.{@i}, much as the \CEE/ preprocessor does when it encounters an \.{\#include}
%line.  After the included file has been entirely read, the program
%goes back to the next line
%of the original file.  The file name following \.{@i} can be
%surrounded by \." characters, but such delimiters are
%optional. Include files can nest.
\@{i} [\\{web\_file}] 
더구나 \\{web\_file} 자체는 여러 파일을 조합해서 구성할 수 있다.
\.{CWEAVE}나 \.{CTANGLE}이 파일을 읽다가 어떤 줄의 처음에 명령어
\.{@i}를 만나면, 현재 읽고 있는 것을 잠깐 멈추고 \.{@i} 다음에 나오는
파일 이름의 파일을 읽어 들인다. 이는 마치 \CEE/ 프로그램에서
\.{\#include}를 만났을 때 \CEE/ 전처리기가 취하는 행동과 같다. 이
명령어로 포함하는 파일을 모두 읽어들이면, 프로그램은 원래 읽어들이던
파일의 그 장소로 옮겨가서 계속 파일 내용을 읽어들인다. \.{@i} 다음에
나오는 파일 이름은 \." 문자로 둘러쌀 수 있는데, 반드시 그렇게 해야 하는
것은 아니다. 포함되는 파일에 \.{@i}가 쓰여서 또 포함할 수 있다. 즉
중첩해서 파일을 포함하는 것이 가능하다.

%\more
%Change files can have lines starting with \.{@i}. In this way
%you can replace one included file with another. Conceptually, the
%replacement mechanism described above does its work first, and
%its output is then checked for \.{@i} lines. If \.{@i} \.{foo}
%occurs between \.{@y} and \.{@z} in a change file, individual lines
%of file \.{foo} and files it includes are not changeable; but changes
%can be made to lines from files that were included by unchanged input.
\more
수정 파일은 \.{@i}로 시작하는 줄들을 가질 수 있다. 이러한 방법으로 포함할 파일을 다른 파일로
바꿀 수 있다. 개념적으로 보면, 위에서 설명한 치환 체계는 우선 자기에게 주어진 일을 처리하고,
그 결과는 \.{@i}를 위해 검사된다. 만약 어떤 수정 파일의 \.{@y}와 \.{@z} 사이에 \.{@i}
\.{foo}이 있다면, \.{foo} 파일의 개별적인 줄들과 그것이 포함하는 파일들은 변경할 수 없다.
그러나 파일의 줄들을 변경할 수 있다. 

%\more On UNIX systems (and others that support environment variables),
%if the environment variable \.{CWEBINPUTS} is set, or if the compiler flag
%of the same name was defined at compile time,
%\.{CWEB} will look for include files in the directory thus named, if
%it cannot find them in the current directory.
\more \UNIX/ 시스템 상에서 (그리고 다른 환경 변수를 제공하는 다른 시스템 상에서), 환경 변수
\.{CWEBINPUTS}이 정해졌거나, 컴파일시에 컴파일 플래그로 \.{CWEBINPUTS}이
정의되었다고, 현재 디렉토리에 포함하려는 파일이 없다면, \.{CWEB}은 포함할 파일을 위 변수에
정의된 디렉토리에서 찾으려고 할 것이다.

%\section Additional features and caveats.
%\noindent
%1. In certain installations of \.{CWEB} that
%{\def\\#1#2{`{\tentex\char'#1#2}'}%
%have an extended character set, the characters
%\\13, \\01, \\31, \\32, \\34, \\35,
%\\36, \\37, \\04, \\20, and \\21}
%can be typed as abbreviations for `\.{++}', `\.{--}', `\.{->}',
%`\.{!=}', `\.{<=}', `\.{>=}', `\.{==}', `\.{\v\v}', `\.{\&\&}',
%`\.{<<}', and `\.{>>}',
%respectively.
\section 추가적인 특징과 주의 사항들.

1. 확장 문자셋이 사용 가능한 시스템에 \.{CWEB}이 설치된 경우에 다음과 같은 문자들은
`\.{++}', `\.{--}', `\.{->}',
`\.{!=}', `\.{<=}', `\.{>=}', `\.{==}', `\.{\v\v}', `\.{\&\&}',
`\.{<<}', `\.{>>}' 각각 다음 문자들로 축약될 수 있다.
{\def\\#1#2{`{\tentex\char'#1#2}'}%
\\13, \\01, \\31, \\32, \\34, \\35,
\\36, \\37, \\04, \\20, \\21}.

%2. If you have an extended character set, you can use it with only minimal
%restrictions, as discussed under the rules for \.{@l} above. But you should
%stick to standard ASCII characters if you want to write programs that will
%be useful to all the poor souls out there who don't have extended
%character sets.
2. 만약 확장 문자셋을 사용할 수 있다면, 앞서 설명한 \.{@l} 명령어 규칙대로 단지 최소의
제약조건들만으로 지키고 그 확장 문자셋을 사용할 수 있다. 그러나 프로그램을 작성하고자 한다면,
확장 문자셋을 사용하지 못하는 불쌍한 영혼들을 위해서 기본 아스키 문자들을 주로 사용하기 바란다.

%3. The \TEX/ file output by \.{CWEAVE} is broken into lines having at most
%80 characters each. When
%\TEX/ text is being copied, the existing line breaks are copied as well.
%If you aren't doing anything too tricky, \.{CWEAVE} will recognize when
%a \TEX/ comment is being split across two or more lines, and it will
%append `\.\%' to the beginning of such continued comments.
3.\.{CWEAVE}가 만들어낸 \TEX/ 파일은 한 줄에 최대 80개의 문자가 들어있는 줄들로 나뉜다.
\TEX/ 텍스트를 복사하면, 기존의 줄 바꿈 표시도 복사된다. 파일을 작성할 때 너무 이상한 짓만
하지 않는다면, \.{CWEAVE}는 \TEX/ 주석문이 두 줄 이상에 걸쳐서 나눠질 때 그것을
인식해서 새 줄이 시작할 때마다 `\.\%'를 입력할 것이다.

%4. \CEE/ text is translated by a ``bottom up'' procedure that
%identifies each token as a ``part of speech'' and combines parts of speech
%into larger and larger phrases as much as possible according to a special
%grammar that is explained in the documentation of \.{CWEAVE}. 
%It is easy to
%learn the translation scheme for simple constructions like single
%identifiers and short expressions, just by looking at a few examples of
%what \.{CWEAVE} does, but the general mechanism is somewhat complex because
%it must handle much more than \CEE/ itself. 
%Furthermore the output
%contains embedded codes that cause \TEX/ to indent and break lines as
%necessary, depending on the fonts used and the desired page width. For
%best results it is wise to avoid enclosing long \CEE/ texts in \pb, since the
%indentation and line breaking codes are omitted when the \pb\ text is
%translated from \CEE/ to \TEX/. Stick to simple expressions or
%statements.  If a \CEE/ preprocessor command is enclosed in \pb,
%the \.\# that introduces it must be at the beginning of a line,
%or \.{CWEAVE} won't print it correctly.
4. \CEE/ 텍스트는 ``상향식'' 절차로 \TEX/ 텍스트로 변환된다. 상향식 절차는
\CEE/의 각각의 토큰을 ``구문''으로 여기고 \.{CWEAVE}의 문서에 설명된
특별 규칙에 따라서 그 구문들을 결합하여 점점 더 포함하는 큰 절로 만든다. 간단한 식별자나 짧은
표현 식을 위한 변환 규칙을 배우는 것은 그저 \.{CWEAVE}가 동작하는 몇 가지 예제만 보는 것으로
충분하다. 그러나 \.{CWEAVE}가 단지 \CEE/ 텍스트만을 다루는 것이 아니기 때문에 일반적인
규칙은 다소 복잡하다. 더구나 출력물은 사용된 글꼴과 원하는 페이지 너비를 기반으로 해서, 필요에
따라 \TEX/이 들여쓰기를 하고 줄 바꿈을 하도록 하는 코드를 포함한다. 최상의 출력물을 얻고자
한다면, 긴 \CEE/ 텍스트를 \pb로 둘러싸는 일은 삼가야겠다. 왜냐하면 \pb 로 \CEE/ 텍스트를
둘러싸면 그 텍스트는 \TEX/ 텍스트로 변환되는데, 이때 \CEE/ 텍스트 고유의 들여쓰기나 줄 바꿈
같은 것들이 완전히 제거되기 때문이다. \pb 로 둘러싸는 \CEE/ 텍스트는 간단한 표현 식이나
문장만을 고려하는 것이 좋다. 만일 \CEE/의 전처리기 명령을 \pb 로 둘러싼다면, \.\#는 줄의 맨
앞에 와야 한다. 그렇지 않으면, \.{CWEAVE}가 제대로 된 출력을 내보내지 않을 것이다.

%5. Comments are not permitted in \pb\ text. After a `\.|'
%signals the change from \TEX/ text to \CEE/ text, the next `\.|' that is
%not part of a string or control text or section name ends the \CEE/ text.
5. 주석은 \pb\ 텍스트 안에서 사용할 수 없다. `\.|'가 \TEX/ 텍스트를 \CEE/ 텍스트로
바꾸라는 신호를 준 후에, 문자열이나 명령 텍스트나 마디 이름의 일부가 아닌 그다음 나오는
`\.|'가 \CEE/ 텍스트를 종료시킨다.

%6. A comment must have properly nested occurrences of left and right
%braces, otherwise \.{CWEAVE} will complain. But it
%does try to balance the braces, so that \TEX/ won't foul up too much.
6. 주석문은 반드시 왼쪽과 오른쪽 괄호가 적절히 매치되어야 한다. 그렇지 않으면, \.{CWEAVE}가
에러 메시지를 뿜어낼 것이다. 그러나 \.{CWEAVE}는 \TEX/을 너무 심한 혼란에 빠뜨리지 않게
하기 위해서 괄호의 균형을 맞추려고 시도한다.

%7. When you're debugging a program and decide to omit some of your
%\CEE/ code, do NOT simply ``comment it out.'' Such comments are not
%in the spirit of \.{CWEB} documentation; they will appear to readers
%as if they were explanations of the uncommented-out instructions.
%Furthermore, comments of a program must be valid \TEX/ text; hence
%\.{CWEAVE} will get confused if you enclose \CEE/ statements in
%\.{/*...*/} instead of in \.{/*|...|*/}.  If you must comment out
%\CEE/ code, you can surround it with preprocessor commands
%like \.{\#if 0==1} and \.{\#endif}.
7. \.{CWEB} 프로그램을 디버깅하려고 하는데, \CEE/ 코드 중 일부를 제외하고 하기로 했다고
하자. 이때 단순히 그 제외하고자 하는 코드를 절대로 주석 처리하면 안 된다. 그러한 상황에서의
주석 처리는 \.{CWEB}이 문서를 만들어내는 정신에 어긋나는 행동이다. 왜냐하면, 그러한 주석들은
마치 프로그램의 설명 중 일부인 양, 프로그램을 읽는 사람들에게 보이기 때문이다. 게다가,
프로그램의 주석문은 반드시 올바른 \TEX/ 텍스트 이어야 한다. 따라서 만약 주석문 처리를
\.{/*|...|*/} 처럼 하지 않고, \.{/*...*/} 처럼 한다면, \.{CWEAVE}는 혼란에 빠질
것이다. 어떤 \CEE/ 코드를 반드시 주석 처리해야 한다면, \.{\#if 0==1} 와 \.{\#endif}와
같은 전처리기 명령을 이용해서 해야 한다.

%8. The \.{@f} feature allows you to define one identifier to act like
%another, and these format definitions are carried out sequentially.
%In general, a given identifier has only one printed format
%throughout the entire document, and this format is used even before
%the \.{@f} that defines it. The reason is that \.{CWEAVE} operates in two
%passes; it processes \.{@f}'s and cross-references on the first pass and
%it does the output on the second.  (However, identifiers that
%implicitly get a boldface format, thanks to a \.{typedef} declaration,
%don't obey this rule; they are printed differently before and after
%the relevant \.{typedef}.  This is unfortunate, but hard to fix. You can
%get around the restriction by saying, say, `\.{@s} \.{foo} \.{int}',
%before or after the \.{typedef}.)
8. 명령어 \.{@f}는 한 식별자를 \TEX/이 다른 식별자 다루듯이 다루게 할 때 사용하는
명령어인데, 이러한 \.{@f}를 이용한 포멧 정의가  여러 번 나올 때, 그 명령이 실행되는 순서는
포멧 정의가 정의된 대로 차례대로 실행된다. 일반적으로, 주어진 한 식별자는 문서 전체를 통해서
한가지 모양으로 출력된다. 그리고 그 모양은 그 식별자에 \.{@f} 명령어가 적용되기 전까지 계속
유지된다. 그 이유는 \.{CWEAVE}는 주어진 파일을 두 번의 패스를 통해서 \TEX/ 파일을
만들어내기 때문이다. 우선 \.{CWEAVE}는 첫 번째 패스 동안에 \.{@f}로 된 포멧 정의들과
교차 참조들을 처리하고, 그 결과를 두 번째 패스로 보낸다. (그러나 \.{typedef} 정의에 의해서
굵은 글꼴의 모양을 갖게 된 식별자들에는 이 규칙들이 적용되지 않는다. 그 식별자들은
\.{typedef} 정의 전과 후의 모양이 다르다. 이러한 상황이 그리 좋은 것만은 아니지만, 그렇다고
쉽게 고칠 수 있는 것도 아니다. 이러한 제약 상황을 해결하는 방법은 \.{typedef} 정의를 하기 전
또는 후에 `\.{@s} \.{foo} \.{int}'라고 입력하는 것이다.)

%9. Sometimes it is desirable to insert spacing into formatted \CEE/ code that
%is more general than the thin space provided by `\.{@,}'. The \.{@t} feature
%can be used for this purpose; e.g., `\.{@t\\hskip 1in@>}' will
%leave one inch of blank space. Furthermore, `\.{@t\\4@>}' can be
%used to backspace by one unit of indentation, since the control sequence
%\.{\\4} is defined in \.{cwebmac} to be such a backspace. (This
%control sequence is used, for example, at the beginning of lines that
%contain labeled statements, so that the label will stick out a little at
%the left.) You can also use `\.{@t\}\\3\{-5@>}' to force a break
%in the middle of an expression.
9. \TEX/으로 조판된 \CEE/ 코드에 약간의 간격을 넣는 것이 `\.{@,}'가 만들어내는 좁은
간격보다 더 자연스러울 때가 종종 있다.  \.{@t}의 특징이 이러한 목적으로 사용될 수 있다.
예를 들어, \TEX/ 명령어 \.{\\4}는 \.{cwebmac}에 들여쓰기 간격만큼 왼쪽으로 들여 쓰라는
명령이므로, `\.{@t\\hskip 1in@>}'는 1인치의 공간을 마련할 것이다. `\.{@t\\4@>}'는
들여쓰기 간격만큼 왼쪽으로 들여 쓰는 명령이다. (이 명령어는 예를 들어, 레이블이 있는 줄의
시작부에서 레이블과 줄의 시작부와 약간의 간격을 두기 위해서 사용된다.) 그리고 또한 표현 식의
중간에서 강제적으로 간격을 두기 위해서 `\.{@t\}\\3\{-5@>}'를 사용할 수도 있다.

%10. Each identifier in \.{CWEB} has a single formatting convention. Therefore
%you shouldn't use the same identifier to denote, say, both a type name and
%part of a \.{struct}, even though \CEE/ does allow this.
10. \.{CWEB}에서 사용된 각각의 식별자는 그 고유의 표현 방식이 있다. 그러므로 자료형 이름과
\.{struct}의 일부분을 표시하기 위해서 비록 \CEE/가 그것을 허용한다고 하더라도, 같은 식별자를
사용해서는 안 된다. 

%\section Running the programs.
%The \UNIX/ command line for \.{CTANGLE} is
%$$\.{ctangle [options] web\_file[.w] [\{change\_file[.ch]|-\} [out\_file]]}$$
%and the same conventions apply to \.{CWEAVE}. If `\.-' or no change file is
%specified, the change file is null. The extensions \.{.w} and \.{.ch}
%are appended only if the given file names contain no dot. If the
%web file defined in this way cannot be found, the extension \.{.web}
%will be tried. For example, `\.{cweave} \.{cob}' will try to read
%\.{cob.w}; failing that, it will try \.{cob.web} before giving up.
%If no output file name is specified, the name of the \CEE/ file output by
%\.{CTANGLE} is obtained by appending the extension \.{.c};
%the name of the \TEX/ file output by \.{CWEAVE} gets the extension \.{.tex}.
%Index files output by \.{CWEAVE} replace \.{.tex} by \.{.idx} and \.{.scn}.
\section 프로그램 실행하기.
\UNIX/에서 \.{CTANGLE}은 다음과 같은 명령으로 실행할 수 있다.
$$\.{ctangle [options] web\_file[.w] [\{change\_file[.ch]|-\} [out\_file]]}$$
\.{CWEAVE}도 위와 같은 방법으로 실행할 수 있다. 만약 `\.-' 나 수정
파일(change file)을 정하지 않으면, 수정 파일은 프로그램상에서 널(null)이 된다.
확장자 \.{.w}와 \.{.ch}는 주어진 파일 이름에 점이 없을 때에만, 자동으로 붙는데, 찾고자
하는 웹 소스 파일을 찾을 수 없을 때는 확장자 \.{.web}를 붙여서 다시 시도한다. 예를 들어,
명령어 `\.{cweave} \.{cob}'는 먼저 \.{cob.w} 파일을 읽어들이려 시도할 것이고, 그 시도가
실패하면, 포기하기 전에 \.{cob.web}를 읽어들이려 시도한다. 결과 파일 이름을 정하지 않으면,
\.{CTANGLE}은 주어진 파일 이름에 \.{.c}를 붙이고, \.{CWEAVE}는 \.{.tex}을 붙여서 파일을
생성한다. \.{CWEAVE}가 부수적으로 만드는, 색인 관련 파일들은 확장자 \.{.idx}와 \.{.scn}가
붙어서 자동으로 생성된다.

%Programmers who like terseness might choose to set up their
% operating shell so that `\.{wv}' expands to
%`\.{cweave -bhp}'; this will suppress most terminal output from \.{CWEAVE}
%except for error messages.
간단 명료한 것을 좋아하는 프로그래머는 자신이 사용하는 OS의 쉘을 조정해서 `\.{wv}'를
`\.{cweave -bhp}'로 확장할 수도 있는데, 그 방법은 에러 메시지를 제외하고는 \.{CWEAVE}이
모니터로 내뿜는 모든 메시지를 무시한다.

%Options are introduced either by a \.- sign, to turn an option off,
%or by a \.+ sign to turn one on. For example, `\.{-fb}' turns off
%options \.f and \.b; `\.{+s}' turns on option \.s. Options can be
%specified before the file names, after the file names, or both. The following
%options are currently implemented:
옵션은 \.- 나 \.+ 기호와 함께 나오는데, \.-는 옵션을 끄는 역할을 하고,
\.+는 켜는 일을 한다. 예를 들어, 옵션 `\.{-fb}'는 \.f와 \.b 옵션의 역할을
못하도록 끄고, `\.{+s}'는 \.s옵션을 켠다. 옵션들은 파일 이름 앞에 올
수도 있고, 뒤에 올 수도 있고, 앞 뒤에 올 수도 있다. 아래에 설명하는
옵션들은 현재 프로그램에서 구현된 것들이다.

\yskip
\def\option#1 {\textindent{\.#1}\hangindent2\parindent}

%\option b Print a banner line at the beginning of execution. (On
%by default.)
\option b 프로그램이 실행될 때, 배너를 출력한다. (기본으로 실행되는 옵션)

%\option e Enclose \CEE/ material formatted by \.{CWEAVE} in
%brackets \.{\\PB\{...\}}, so that special hooks can be used.
%(Off by default; has no effect on \.{CTANGLE}.)
\option e \.{CWEAVE}가 \CEE/ 코드를 \.{\\PB\{...\}}로 감싼다. 그래서
그 코드에 특별한 일을 할 수 있다.
(기본으로 꺼져있는 옵션이고, \.{CTANGLE}에는 아무런 영향을 미치지 않는다.)

%\option f Force line breaks after each \CEE/ statement formatted
%by \.{CWEAVE}. (On by default; \.{-f} saves paper but looks less \CEE/-like
%to some people.) (Has no effect on \.{CTANGLE}.)
\option f \.{CWEAVE}가 \CEE/ 문장들을 다룰 때, 그 문장들 다음에
강제적으로 줄 바꿈을 한다. (기본 실행 옵션이다. \.{-f} 옵션은 줄 바꿈을
하지 않으므로 종이를 절약할 수 있으나, \CEE/ 코드처럼 보이지 않을 수
있다. \.{CTANGLE}에는 전혀 영향을 미치지 않는 옵션이다.)

%\option h Print a happy message at the conclusion of a successful
%run. (On by default.)
\option h 프로그램이 실행을 정상적으로 끝마쳤을때, 제대로 실행했다는
행복한 메시지를 뿌려준다. (기본 실행 옵션이다.)

%\option p Give progress reports as the program runs. (On by default.)
\option p 프로그램이 실행할 때 진행상황을 보여준다. (기본 실행 옵션이다.)

%\option s Show statistics about memory usage after the program
%runs to completion. (Off by default.)
%If you
%have large \.{CWEB} files or sections, you may need to see
%how close you come to exceeding the capacity of \.{CTANGLE} and/or \.{CWEAVE}.
\option s 프로그램이 실행을 정상적으로 끝마쳤을 때, 프로그램의 메모리 사용에 대한
통계를 보여준다. (기본 실행 옵션이 아니다.) 만약 프로그래밍하는
\.{CWEB} 파일이 매우 크거나 마디의 수가 많다면, \.{CTANGLE} 이나/또는
\.{CWEAVE}의 용량을 얼마나 초과하는지를 살펴볼 필요가 있을 것이다.

%\option x Include indexes and a table of contents in the \TEX/ file
%output by \.{CWEAVE}. (On by default.) (Has no effect on \.{CTANGLE}.)
\option x \.{CWEAVE}가 만들어낸 \TEX/ 파일이 색인이나 목차 정보를
포함하도록 한다. (기본 실행 옵션이고, \.{CTANGLE}에는 전혀 영향을
끼치지 않는다.)

%\section Further details about formatting.
%You may not like the way \.{CWEAVE} handles certain
%situations. If you're desperate, you can customize \.{CWEAVE}
%by changing its grammar.  This means changing the source code,
%a task that you might find amusing. A table of grammar rules
%appears in the \.{CWEAVE} source listing, and you can make a separate
%copy of that table by copying the file \.{prod.w} found in the \.{CWEB}
%sources and saying `\.{cweave}~\.{-x}~\.{prod}', followed by
%`\.{tex}~\.{prod}'.
\section 포멧팅에 관한 더 알아야 할 세부 사항.
당신은 \.{CWEAVE}가 처리하는 방식이 맘에 들지 않을 때가 있을 것이다. 그 상황을 참지
못하겠다면, \.{CWEAVE}가 동작하는 방식이 기록된 문법 표를 수정해서 \.{CWEAVE}를 자신이
원하는 대로 동작하도록 만들 수 있다. 그렇게 하려면, \.{CWEAVE} 프로그램의 소스를 직접
수정해야 하는데, 이 일이 경우에 따라서는 즐거운 일이 될 수도 있다. 어쨌든 \.{CWEAVE}의
동작 원리가 기록된 문법 표는 \.{CWEAVE} 프로그램 소스 코드에 기록되어 있는데, 이를 별도의
파일로 만들어서 보고 싶으면, \.{CWEB}의 소스 파일 중에서 \.{prod.w}라는 파일이 있는데,
`\.{cweave}~\.{-x}~\.{prod}', 곧이어 `\.{tex}~\.{prod}'라는 명령들을 통해서 만들 수
있다.  
%You can see exactly
%how \.{CWEAVE} is parsing your \CEE/ code by preceding
%it with the line `\.{@ @c @2}'. (The control code `\.{@2}'
%turns on a ``peeping'' mode, and `\.{@0}' turns it off.)
%For example, if you run \.{CWEAVE} on the file
\.{CWEAVE}가 어떻게 \CEE/ 코드를 구문분석해서 아름다운 문서를 만드는지를 자세히 알고 싶으면,
그 \CEE/ 코드 앞에 `\.{@ @c @2}'라고 입력하면 된다. (명령어 \.{@2}는 \.{CWEAVE}가 하는
일을 자세히 볼 수 있도록 스위치를 켜는 역할을 하고 그 반대로 스위치를 끄는 명령어는
`\.{@0}'이다.) 예를 들어, 아래와 같은 코드를 \.{CWEAVE}로 돌리면,
\medskip
\begingroup
\verbatim
@ @c @2
main (argc,argv)
char **argv;
{ for (;argc>0;argc--) printf("%s\n",argv[argc-1]); }
!endgroup
\endgroup
\medskip\noindent
%you get the following gibberish on your screen:
모니터에 다음과 같은 암호같은 코드들이 나온다. 
\medskip
\begingroup
\verbatim
[...]
4:*exp ( +exp+ )...
11:*exp +exp+ int...
5:*+exp+ int +unorbinop+...
[...]
60: +fn_decl+*+{+ -stmt- +}-
55:*+fn_decl+ -stmt-
52:*+function-
[...]
!endgroup
\endgroup
\medskip
%The first line says that grammar rule 4 has just been applied, and \.{CWEAVE}
%currently has in its memory a sequence of chunks of \TEX/ code (called
%``scraps'') that are respectively
%of type \\{exp} (for expression), open-parenthesis,
%\\{exp} again, close-parenthesis, and further scraps that haven't yet
%been considered by the parser.  (The \.+ and \.- signs stipulate that
%\TEX/ should be in or out of math mode at the scrap boundaries. The \.* shows
%the parser's current position.)
%Then rule 11 is applied, and
%the sequence $(\,exp\,)$ becomes an \\{exp} and so on.  In the
%end the whole \CEE/ text has become one big scrap of type \\{function}.
첫 번째 줄은 \.{CWEAVE}가 \CEE/ 코드를 구문 분석하는 문법 중에 4번째 규칙을 적용한다는 것을
의미하고, 현재 이것을 처리하는 순간의 \.{CWEAVE}가 사용하는 메모리는
``스크랩(scraps)''이라고 불리는 일련의 \TEX/ 코드들을 가지고 있는데, 그 일련의 \TEX/
코드들은 각각 \\{exp} (표현식), 여는 괄호, 다시 \\{exp}, 닫는 괄호, 그리고 나머지 아직
파서가 처리하지 않고 남아 있는 스크랩들이라는 것을 나타낸다. (\.+ 와 \.- 기호는 \TEX/이
스크랩으로 나누는 경계에 있을 때 수학 모드 안에 또는 밖에 있어야 한다는 것을 분명히 하는데
사용되고, \.*는 현재 파서의 위치를 나타낸다.) 그리고 나서 11번째 규칙이 적용되고,
$(\,exp\,)$는 \\{exp}이 되는 등, 연속적으로 적용이 되고 있다는 것을 보여준다. 이러한 과정을
통해서 위의 코드는 결국은 함수({\it function\/})형이라는 커다란 스크랩이 되는 것이다.

%Sometimes things don't work as smoothly, and you get a bunch of
%lines lumped together.  This means that \.{CWEAVE} could not
%digest something in your \CEE/ code.  For instance, suppose
%`\.{@<Argument definitions@>}' had appeared instead of
%`\.{char **argv;}' in the program above. Then \.{CWEAVE} would have
%been somewhat mystified, since it thinks that section names
%are just \\{exp}s.  Thus it would tell \TEX/ to format
%`\X2:Argument declarations\X' on the same line as
%`$\\{main}(\\{argc},\39\\{argv}{}$)'.
%In this case you should help \.{CWEAVE} by putting `\.{@/}' after
%`\.{main(argc,argv)}'.
\.{CWEAVE}의 문법은, 입력된 \CEE/ 코드를 완전히 이해하여 언제나 올바르게 적용되지는 않아서,
입력한 \CEE/ 코드가 모두 한 줄에 다 나와 버리는 경우가 생긴다. 예를 들어, 위의 프로그램에서
`\.{@<Argument definitions@>}'가 `\.{char **argv;}' 대신에 나왔다고 해보자. 그러면
\.{CWEAVE}는 마디 이름을 그냥 단순히 \\{exp}라고 판단하기 때문에 다소 혼란에 빠진다.
그래서 \TEX/에게 `$\\{main}(\\{argc},\39\\{argv}{}$)'와 같은 줄에 
`\X2:Argument declarations\X'라는 형태로 출력할 것이다. 이 경우
\.{CWEAVE}가 올바르게 동작할 수 있도록 
`\.{main(argc,argv)}' 다음에 `\.{@/}'를 써주어야 한다.

%\.{CWEAVE} automatically inserts a bit of extra space between declarations
%and the first apparent statement of a block. One way to defeat this
%spacing locally is
%$$\vbox{\halign{#\hfil\cr
%\.{int x;@+@t\}\\6\{@>}\cr
%\.{@<Other locals@>@;@\#}\cr}}$$
%the `\.{@\#}' will put extra space after `$\langle\,$Other locals$\,\rangle$'.
\.{CWEAVE}는 자동으로 선언문들과 그다음의 첫 번째 실행문 사이에 약간의 공간을 두는데,
\.{CWEAVE} 명령을 실행할 때, 옵션으로 \.{-o}를 주어서 전체적으로 공간을 두지 말라고 할 수도
있다. 이렇게 전체적으로 공간을 없애는 것이 아니라 필요한 곳에서만 공간을 없앨 수도 있는데, 그
방법은 아래와 같다.
$$\vbox{\halign{#\hfil\cr
\.{int x;@+@t\}\\6\{@>}\cr
\.{@<Other locals@>@;@\#}\cr}}$$
`\.{@\#}'는 `$\langle\,$Other locals$\,\rangle$' 다음에 약간의 공간을
만든다. 

%\section Hypertext and hyperdocumentation.
%Many people have of course noticed analogies between \.{CWEB} and the
%World Wide Web. The \.{CWEB} macros are in fact set up so that the output of
%\.{CWEAVE} can be converted easily into Portable Document Format,
%with clickable hyperlinks that can be read with Adobe's Acrobat Reader,
%using a widely available open-source program called \.{dvipdfm} developed
%by Mark~A. Wicks. After using \.{CWEAVE} to convert \.{cob.w} into
%\.{cob.tex}, you can prepare and view a hypertext version of the program
%by giving the commands
\section 하이퍼텍스트와 하이퍼문서화.
당연히 많은 사람들이 \.{CWEB}이 월드와이드웹과 유사한 점이 많다는 것을 눈치 챘을 것이다.
사실 \.{CWEB}의 매크로들은 \.{CWEAVE}의 결과물인 \TEX/ 파일을 {\mc PDF} 파일
(Portable Document Format)로 쉽게 바꿀 수 있는 기능들을 가지고 있다. 그래서 어도비의
아크로뱃 리더를 이용하면 \.{CWEB} 문서 내에서 하이퍼링크를 클릭하여 여기저기를 자유롭게
오고 갈 수 있다. 이 특징은  Mark~ A. Wicks가 개발한 \.{dvipdfm}이라는 오픈 소스
프로그램으로 가능하게 되었다. \.{CWEAVE}를 이용해서 \.{cob.w}를 \.{cob.tex}로 변환한
후에 보통의 방식으로 \TEX/ 컴파일을 하는 것이 아니라, 아래와 같은 명령들을 이용하면
하이퍼텍스트 문서를 만들 수 있다.
$$\vbox{\halign{\.{#}\hfil\cr
tex "\\let\\pdf+ \\input cob"\cr
dvipdfm cob\cr
acroread cob.pdf\cr}}$$
%instead of invoking \TeX\ in the normal way. (Thanks to Hans Hagen,
%C\'esar Augusto Rorato Crusius, and Julian Gilbey
%for the macros that make this work.)
%Alternatively, thanks to H\`an Th\^e\kern-.3em\raise.3ex\hbox{\'{}} Th\`anh
%and Andreas Scherer, you can generate \.{cob.pdf} in one step by simply
%saying `\.{pdftex}~\.{cob}'.
(위의 명령들을 가능하도록 매크로를 만든 한스 하겐, C\'esar Augusto Rorato
Crusius, Julian Gilbey 에게 감사한다.) 위 방법 말고 더 쉬운 방법이 있는데, 단순히
`\.{pdftex}~\.{cob}'라고 하면 곧바로 \.{cob.pdf}를 얻을 수 있다. 이에 대해서는 
한데탄과 Andreas Scherer에게 감사한다.

%A more elaborate system called \.{CTWILL}, which extends the usual cross
%references of \.{CWEAVE} by preparing links from the uses of identifiers
%to their definitions, is also available---provided that you are willing
%to work a bit harder in cases where an identifier is multiply defined.
%\.{CTWILL} is intended primarily for hardcopy output, but its principles
%could be used for hypertext as well.
%See Chapter 11 of {\sl Digital Typography\/} by D.~E. Knuth (1999), and
%the program sources at \.{ftp://ftp.cs.stanford.edu/pub/ctwill}.
\.{CWEAVE}와 비슷한 기능을 하지만, 그보다는 좀 더 공들인 \.{CTWILL}라 불리는 프로그램이
있는데, 이 프로그램은 \.{CWEAVE}의 일반적인 교차 참조 기능을 좀 더 확장한 프로그램으로,
변수가 사용된 지점에 연결된 링크를 통해서 곧바로 그 변수가 정의된 곳으로 갈 수 있다.
\.{CTWILL}은 책이나 문서 같은 하드 카피를 만들기 위해 작성된 프로그램이지만 이 프로그램도
하이퍼텍스트 기능을 갖출 수 있다. 도널드 크누스의 책, {\sl Digital Typography\/} (1999)
11장을 살펴보라.  그리고 \.{CTWILL} 프로그램의 소스들을
\.{ftp://ftp.cs.stanford.edu/pub/ctwill} 에서 찾을 수 있다.

\section 부 록.
부록~A는 \.{CWEB} 언어로 작성한 실제 프로그램의 예로써, \.{CWEB} 시스템의 그 자체의
소스에서 일부를 발췌하였다. 이 부록에 실린 내용을 주의 깊게 살펴보면, \.{CWEB} 프로그램에
대한 기본 개념과 감을 잡을 수 있다.

%Appendix B is the file that sets \TEX/ up to accept
%the output of \.{CWEAVE}, and Appendix~C discusses how to use some of those
%macros to vary the output formats.
부록 B는 \.{CWEAVE}의 결과물인 \TEX/ 파일을 처리하기 위해서 컴파일러 \TEX/을 설정하는
매크로 파일을 설명하고, 부록~C는 다양한 출력물을 얻기 위해서 그 매크로들을 사용하는 방법을
설명한다. 

%A ``long'' version of this manual, which can be produced from the \.{CWEB}
%sources via the \UNIX/ command \.{make} \.{fullmanual}, also contains
%appendices D, E, and~F, which exhibit the complete source code for
%\.{CTANGLE} and \.{CWEAVE}.
사실 이 사용자 설명서는 완벽한 버전이 아니다. \.{CWEB} 시스템의 사용자
설명서의 완벽한 버전은 \UNIX/ 명령어 \.{make} \.{fullmanual}로
얻을 수 있고, 그 완벽한 버전에는 이 설명서에는 없는 부록 D, E, F가 더
있으며, 그 부록은 \.{CWEB} 시스템의 소스 코드를 담고 있는 프로그램
파일인 \.{COMMON}, \.{CTANGLE}, \.{CWEAVE}를 가지고 있다.

\vfil\eject\titletrue

\def\headlinefont{\rm\hfont{nanummjm}{at 9pt}}
\def\runninghead{\headlinefont 부록 A --- {\tt CWEB} 파일 포멧}
\section 부록 A: \.{CWEB} 프로그램에서 발췌한 예제.
%This appendix consists
%of four listings.  The first shows the \.{CWEB} input that
%generated sections 12--15 of the file \.{common.w}, which contains
%routines common to \.{CWEAVE} and \.{CTANGLE}.
%Note that some of the lines are indented to show the program structure;
%the indentation is ignored by \.{CWEAVE} and \.{CTANGLE}, but users find
%that \.{CWEB} files are quite readable if they have some such indentation.
부록 {\mc A}는 네 부분으로 구성되어 있다. 첫 번째는 \.{CWEAVE}와
\.{CTANGLE}이 공통으로 사용하는 루틴들을 담고 있는 \.{common.w} 파일의
12--15 마디들을 만드는 소스 코드이다. 아래 소스 리스트를 보면,
사실은 필요없는 프로그램의 들여쓰기 등이 되어 있는 것을 알 수
있다. 들여쓰기 같은 것은 사용자가 어떻게 하던
상관없이 \.{CWEAVE}와 \.{CTANGLE} 프로그램이 \CEE/ 코드를 이해하고
있어서 알아서 자동으로 들여쓰기 등 보기 좋은 모양으로 원래의 소스
리스트를 편집하여 문서로 만들어준다. 하지만, 지금처럼 \.{CWEB} 소스
파일을 직접 살펴 볼 경우도 있으므로 이와 같은 들여쓰기를 할 필요도 있다.

%The second and third listings
%show corresponding parts
%of the \CEE/ code output by \.{CTANGLE} and of
%the corresponding \TEX/ code output by \.{CWEAVE}, when run on \.{common.w}.
%The fourth listing shows how that output looks when printed out.
두 번째와 세 번째는 첫 번째 부분에 해당하는 \.{CTANGLE}이 만들어낸 \CEE/ 코드 출력물과
\.{CWEAVE}가 만들어낸 \TEX/ 코드 출력물이다. 네 번째는 최종 출력 문서이다.

\vskip 10pt
\begingroup \def\tt{\eighttt} \baselineskip9pt \hfont{nanumgtm}{at 8pt}
\verbatim
@ 프로시저 |prime_the_change_buffer|는 계속되는 매칭 과정에서 다음
매칭 동작을 위해서 |change_buffer|를 세팅한다. 수정 파일에 있는 빈 줄은 매칭에
쓰이지 않기 때문에, |(change_limit==change_buffer && !!changing)|와
같은 검사를 해야한다. 즉, 수정 파일을 다 읽어들였는지 확인해야
한다. 이 프로시저는 |changing|이 1일 때만 호출된다. 따라서 에러
메시지들은 정확하게 보고될 것이다.

@c
void
prime_the_change_buffer()
{
  change_limit=change_buffer; /* 이 값은 수정 파일이 끝났는지를 체크할 때 사용된다. */
  @<수정 파일에서 주석문을 무시하고 건너뛰어라; 파일의 끝이면 |return|@>;
  @<빈 줄이 아닐 때까지 건너뛰어라; 파일의 끝이면 |return|@>;
  @<|buffer|와 |limit|의 내용을 |change_buffer|와 |change_limit|으로 옮겨라@>;
}

@ 수정 파일에서 \.{@@x}로 시작하는 줄들을 찾는 과정에서, \.{@@y},
\.{@@z}, \.{@@i}로 시작하지 않으면서 \.{@@}로 시작하는 줄이 나와도 된다.
(만약 그러한 명령어로 시작된다면, 수정 파일이 잘못되었다는 것을 뜻한다.)

@<수정 파일에서 주석문을 무시하고 건너뛰어라...@>=
while(1) {
  change_line++;
  if (!!input_ln(change_file)) return;
  if (limit<buffer+2) continue;
  if (buffer[0]!!='@@') continue;
  if (xisupper(buffer[1])) buffer[1]=tolower(buffer[1]);
  if (buffer[1]=='x') break;
  if (buffer[1]=='y' || buffer[1]=='z' || buffer[1]=='i') {
    loc=buffer+2;
    err_print("!! Missing @@x in change file");
@.Missing @@x...@>
  }
}

@ 여기서 \.{@@x} 다음에 나오는 줄들을 찾는다.

@<빈 줄이 아닐 때까지 건너뛰어라...@>=
do {
  change_line++;
  if (!!input_ln(change_file)) {
    err_print("!! Change file ended after @@x");
@.Change file ended...@>
    return;
  }
} while (limit==buffer);

@ @<|buffer|와 |limit|의 내용을 |change_buffer|와 |change_limit|으로 옮겨라@>=
{
  change_limit=change_buffer-buffer+limit;
  strncpy(change_buffer,buffer,limit-buffer+1);
}

!endgroup
\endgroup
\vfill\eject

\def\runninghead{\headlinefont 부록 
A --- {\tt CTANGLE}의 변환 결과물 \CEE/ 코드}

%Here's the portion of the \CEE/ code generated by \.{CTANGLE} that corresponds
%to the source on the preceding page.  Notice that sections~13, 14 and~15
%have been tangled into section~12.
이 페이지는 앞 페이지의 소스에 해당하는, \.{CTANGLE}이 만들어낸 \CEE/
코드를 담고 있다. 13, 14, 15번 마디가 12번 마디 안으로
들어간 것(tangled)을 확인하라.

\vskip10pt
\begingroup \def\tt{\eighttt} \baselineskip9pt
\verbatim
/*:9*//*12:*/
#line 247 "common.w"

void
prime_the_change_buffer()
{
change_limit= change_buffer;
/*13:*/
#line 261 "common.w"

while(1){
change_line++;
if(!!input_ln(change_file))return;
if(limit<buffer+2)continue;
if(buffer[0]!!='@')continue;
if(xisupper(buffer[1]))buffer[1]= tolower(buffer[1]);
if(buffer[1]=='x')break;
if(buffer[1]=='y'||buffer[1]=='z'||buffer[1]=='i'){
loc= buffer+2;
err_print("!! Missing @x in change file");

}
}

/*:13*/
#line 252 "common.w"
;
/*14:*/
#line 278 "common.w"

do{
change_line++;
if(!!input_ln(change_file)){
err_print("!! Change file ended after @x");

return;
}
}while(limit==buffer);

/*:14*/
#line 253 "common.w"
;
/*15:*/
#line 288 "common.w"

{
change_limit= change_buffer-buffer+limit;
strncpy(change_buffer,buffer,limit-buffer+1);
}

/*:15*/
#line 254 "common.w"
;
}

/*:12*//*16:*/
!endgroup
\endgroup
\vfill\eject

\def\runninghead{\headlinefont 부록 
A --- {\tt CWEAVE}의 변환 결과 \TEX/ 코드}

\.{common.tex} 파일에서 앞의 페이지들에 해당하는 부분을 발췌한 것이다.

\vskip8pt
\begingroup \def\tt{\eighttt} \baselineskip9pt \hfont{nanumgtm}{at 8pt}
\verbatim
\M{12}프로시저 \PB{\\{prime\_the\_change\_buffer}}는 계속되는 매칭 과정에서 다음 
매칭 동작을 위해서 \PB{\\{change\_buffer}}를 세팅한다. 수정 파일에 있는 빈 줄은 매칭에
쓰이지 않기 때문에, \PB{$(\\{change\_limit}\E\\{change\_buffer}\W\R\R%
\\{changing})$}와 같은 검사를 해야한다. 즉, 수정 파일을 다 읽어들였는지 확인해야 한다. 
이 프로시저는 \PB{\\{changing}}이 1일 때만 호출된다. 
따라서 에러 메시지들은 정확하게 보고될 것이다.

\Y\B\&{void} \\{prime\_the\_change\_buffer}(\,)\1\1\2\2\6
${}\{{}$\1\6
${}\\{change\_limit}\K\\{change\_buffer}{}$;\C{ 이 값은 수정 파일이
끝났는지를 체크할 때 사용된다. }\6
\X13:수정 파일에서 주석문을 무시하고 건너뛰어라; 파일의 끝이면 \PB{\&{return}}\X;\6
\X14:빈 줄이 아닐 때까지 건너뛰어라; 파일의 끝이면 \PB{\&{return}}\X;\6
\X15:\PB{\\{buffer}}와 \PB{\\{limit}}의 내용을 \PB{\\{change\_buffer}}와 %
\PB{\\{change\_limit}}으로 옮겨라\X;\6
\4${}\}{}$\2\par
\fi

\M{13}수정 파일에서 \.{@x}로 시작하는 줄들을 찾는 과정에서, \.{@y}, \.{@z}, \.{@i}로 
시작하지 않으면서 \.{@}로 시작하는 줄이 나와도 된다. (만약 그러한 명령어로 시작된다면, 
수정 파일이 잘못되었다는 것을 뜻한다.)

\Y\B\4\X13:수정 파일에서 주석문을 무시하고 건너뛰어라; 
파일의 끝이면 \PB{\&{return}}\X${}\E{}$\6
\&{while} (\T{1})\5
${}\{{}$\1\6
${}\\{change\_line}\PP;{}$\6
\&{if} ${}(\R\R\\{input\_ln}(\\{change\_file})){}$\1\5
\&{return};\2\6
\&{if} ${}(\\{limit}<\\{buffer}+\T{2}){}$\1\5
\&{continue};\2\6
\&{if} ${}(\\{buffer}[\T{0}]\R\I\.{'@'}){}$\1\5
\&{continue};\2\6
\&{if} (\\{xisupper}(\\{buffer}[\T{1}]))\1\5
${}\\{buffer}[\T{1}]\K\\{tolower}(\\{buffer}[\T{1}]);{}$\2\6
\&{if} ${}(\\{buffer}[\T{1}]\E\.{'x'}){}$\1\5
\&{break};\2\6
\&{if} ${}(\\{buffer}[\T{1}]\E\.{'y'}\V\\{buffer}[\T{1}]\E\.{'z'}\V\\{buffer}[%
\T{1}]\E\.{'i'}){}$\5
${}\{{}$\1\6
${}\\{loc}\K\\{buffer}+\T{2};{}$\6
\\{err\_print}(\.{"!!\ Missing\ @x\ in\ ch}\)\.{ange\ file"});\6
\4${}\}{}$\2\6
\4${}\}{}$\2\par
\U12.\fi

\M{14}여기서 \.{@x} 다음에 나오는 줄들을 찾는다.

\Y\B\4\X14:빈 줄이 아닐 때까지 건너뛰어라; 파일의 끝이면 \PB{%
\&{return}}\X${}\E{}$\6
\&{do}\5
${}\{{}$\1\6
${}\\{change\_line}\PP;{}$\6
\&{if} ${}(\R\R\\{input\_ln}(\\{change\_file})){}$\5
${}\{{}$\1\6
\\{err\_print}(\.{"!!\ Change\ file\ ende}\)\.{d\ after\ @x"});\6
\&{return};\6
\4${}\}{}$\2\6
\4${}\}{}$\2\5
\&{while} ${}(\\{limit}\E\\{buffer}){}$;\par
\U12.\fi

\M{15}\B\X15:\PB{\\{buffer}}와 \PB{\\{limit}}의 내용을 \PB{\\{change%
\_buffer}}와 \PB{\\{change\_limit}}으로 옮겨라\X${}\E{}$\6
${}\{{}$\1\6
${}\\{change\_limit}\K\\{change\_buffer}-\\{buffer}+\\{limit};{}$\6
${}\\{strncpy}(\\{change\_buffer},\39\\{buffer},\39\\{limit}-\\{buffer}+%
\T{1});{}$\6
\4${}\}{}$\2\par\U12.\fi
!endgroup
\endgroup
\vfil\eject

\def\inote#1#2#3.{\Y\noindent{\hangindent2em%
  \baselineskip10pt\eightrm\hfont{nanummjm}{at 7pt}%
    #1~\ifacro{\pdfnote#3.}\else#3\fi#2.\par}}
\def\U{\inote{이 코드는}{번 섹션에서 사용됩니다}}

\def\runninghead{\headlinefont 부록 A --- 최종 문서}

앞 페이지에 해당하는 부분을 최종 문서로 만들어 낸 결과이다.

{
\baselineskip13pt \hfont{nanummjm}{at 9pt}
\M{12}프로시저 \PB{\\{prime\_the\_change\_buffer}}는 계속되는 매칭 과정에서 다음 
매칭 동작을 위해서 \PB{\\{change\_buffer}}를 세팅한다. 수정 파일에 있는 빈 줄은 매칭에
쓰이지 않기 때문에, \PB{$(\\{change\_limit}\E\\{change\_buffer}\W\R\R%
\\{changing})$}와 같은 검사를 해야한다. 즉, 수정 파일을 다 읽어들였는지 확인해야 한다. 
이 프로시저는 \PB{\\{changing}}이 1일 때만 호출된다. 
따라서 에러 메시지들은 정확하게 보고될 것이다.

\Y\B\&{void} \\{prime\_the\_change\_buffer}(\,)\1\1\2\2\6
${}\{{}$\1\6
${}\\{change\_limit}\K\\{change\_buffer}{}$;\C{ 이 값은 수정 파일이
끝났는지를 체크할 때 사용된다. }\6
\X13:수정 파일에서 주석문을 무시하고 건너뛰어라; 파일의 끝이면 \PB{\&{return}}\X;\6
\X14:빈 줄이 아닐 때까지 건너뛰어라; 파일의 끝이면 \PB{\&{return}}\X;\6
\X15:\PB{\\{buffer}}와 \PB{\\{limit}}의 내용을 \PB{\\{change\_buffer}}와 %
\PB{\\{change\_limit}}으로 옮겨라\X;\6
\4${}\}{}$\2\par
\fi

\M{13}수정 파일에서 \.{@x}로 시작하는 줄들을 찾는 과정에서, \.{@y}, \.{@z}, \.{@i}로 
시작하지 않으면서 \.{@}로 시작하는 줄이 나와도 된다. (만약 그러한 명령어로 시작된다면, 
수정 파일이 잘못되었다는 것을 뜻한다.)

\Y\B\4\X13:수정 파일에서 주석문을 무시하고 건너뛰어라; 
파일의 끝이면 \PB{\&{return}}\X${}\E{}$\6
\&{while} (\T{1})\5
${}\{{}$\1\6
${}\\{change\_line}\PP;{}$\6
\&{if} ${}(\R\R\\{input\_ln}(\\{change\_file})){}$\1\5
\&{return};\2\6
\&{if} ${}(\\{limit}<\\{buffer}+\T{2}){}$\1\5
\&{continue};\2\6
\&{if} ${}(\\{buffer}[\T{0}]\R\I\.{'@'}){}$\1\5
\&{continue};\2\6
\&{if} (\\{xisupper}(\\{buffer}[\T{1}]))\1\5
${}\\{buffer}[\T{1}]\K\\{tolower}(\\{buffer}[\T{1}]);{}$\2\6
\&{if} ${}(\\{buffer}[\T{1}]\E\.{'x'}){}$\1\5
\&{break};\2\6
\&{if} ${}(\\{buffer}[\T{1}]\E\.{'y'}\V\\{buffer}[\T{1}]\E\.{'z'}\V\\{buffer}[%
\T{1}]\E\.{'i'}){}$\5
${}\{{}$\1\6
${}\\{loc}\K\\{buffer}+\T{2};{}$\6
\\{err\_print}(\.{"!!\ Missing\ @x\ in\ ch}\)\.{ange\ file"});\6
\4${}\}{}$\2\6
\4${}\}{}$\2\par
\U12.\fi

\M{14}여기서 \.{@x} 다음에 나오는 줄들을 찾는다.

\Y\B\4\X14:빈 줄이 아닐 때까지 건너뛰어라; 파일의 끝이면 \PB{%
\&{return}}\X${}\E{}$\6
\&{do}\5
${}\{{}$\1\6
${}\\{change\_line}\PP;{}$\6
\&{if} ${}(\R\R\\{input\_ln}(\\{change\_file})){}$\5
${}\{{}$\1\6
\\{err\_print}(\.{"!!\ Change\ file\ ende}\)\.{d\ after\ @x"});\6
\&{return};\6
\4${}\}{}$\2\6
\4${}\}{}$\2\5
\&{while} ${}(\\{limit}\E\\{buffer}){}$;\par
\U12.\fi

\M{15}\B\X15:\PB{\\{buffer}}와 \PB{\\{limit}}의 내용을 \PB{\\{change%
\_buffer}}와 \PB{\\{change\_limit}}으로 옮겨라\X${}\E{}$\6
${}\{{}$\1\6
${}\\{change\_limit}\K\\{change\_buffer}-\\{buffer}+\\{limit};{}$\6
${}\\{strncpy}(\\{change\_buffer},\39\\{buffer},\39\\{limit}-\\{buffer}+%
\T{1});{}$\6
\4${}\}{}$\2\par\U12.\fi

}

\vfil\eject\titletrue
\rightskip=0pt % get out of C mode (cf. \B)
\sfcode`;=1500 \pretolerance 200 \hyphenpenalty 50 \exhyphenpenalty 50

\def\runninghead{\headlinefont 부록 B --- 문서 설정을 위한 매크로}
\section 부록 B: \.{cwebmac.tex} 파일.
%This is the file that extends ``plain \TEX/'' format in order to support the
%features needed by the output of \.{CWEAVE}.
부록 {\mc B}는 ``플레인 \TEX/'' 포멧을 확장하여 \.{CWEAVE}가 만들어
낸 출력물을 멋진 문서로 만들기 위한 매크로들을 담고 있는 포멧 파일을 보여준다.

\vskip6pt
\begingroup \def\tt{\eighttt} \baselineskip9pt
\def\printmacs{\begingroup
  \def\do##1{\catcode`##1=12 } \dospecials
  \parskip 0pt \parindent 0pt
  \catcode`\ =13 \catcode`\^^M=13
  \tt \verbatimdefs \input cwebmac \endgroup}
\printmacs
\endgroup
\vfill\eject

\def\runninghead{\headlinefont 부록 C --- 문서 작성에 알아둘 것들}
%\section Appendix C: How to use \.{CWEB} macros.
%The macros in \.{cwebmac} make it possible to produce a variety of formats
%without editing the output of \.{CWEAVE}, and the purpose of this appendix
%is to explain some of the possibilities.
\section 부록 C: \.{CWEB} 매크로 사용법.
\.{cwebmac} 파일에 정의되어 있는 매크로들 이용하면 \.{CWEAVE}의 결과물인 \TEX/ 코드들을
직접 수정하지 않고도 여러 다양한 형태의 문서를 만들 수 있다. 부록~C 에서는 그러한 매크로들의
사용법을 설명한다.
\def\point#1.{\yskip\indent#1.\quad\ignorespaces}

%\point 1. Four fonts have been declared in addition to the standard fonts of
%\.{PLAIN} format: You can say `\.{\{\\mc UNIX\}}' to get {\mc UNIX} in
%medium-size caps; you can say `\.{\{\\sc STUFF\}}' to get {\sc STUFF}
%in small caps; and you can select the largish fonts \.{\\titlefont}
%and \.{\\ttitlefont} in the title of your document, where \.{\\ttitlefont}
%is a typewriter style of type. There are macros \.{\\UNIX/} and \.{\\CEE/}
%to refer to \UNIX/ and \CEE/ with medium-size caps.
\point 1. \.{CWEB} 매크로는 \.{PLAIN} 포멧의 기본 글꼴 외에 네 가지의 글꼴을
더 제공한다. 중간 크기의 알파벳 대문자체인 {\mc UNIX}를 얻고자 한다면,
`\.{\{\\mc UNIX\}}' 라고 하면 되고, 작은 크기의 알파벳 대문자체인 {\sc STUFF}를 얻고
싶으면 `\.{\{\\sc STUFF\}}' 라고 하면 된다. 그리고 문서의 제목 같은 다소 큰 글꼴을
얻기 위해서는 \.{\\titlefont} 나 \.{\\ttitlefont} 를 이용하면 되는데, 후자는 타자 글꼴로
나타낼 때 사용한다. 그 외에 \.{\\UNIX/} 와 \.{\\CEE/} 같은 매크로가 있는데, 이들은 중간
크기의 대문자체인 \UNIX/와 \CEE/를 나타낸다.

%\point 2. When you mention an identifier in \TEX/ text, you normally call
%it `\.{|identifier|}'. But you can also say `\.{\\\\\{identifier\}}'. The
%output will look the same in both cases, but the second alternative
%doesn't put \\{identifier} into the index, since
%it bypasses \.{CWEAVE}'s translation from \CEE/ mode. In the second
%case you must put a backslash before each underline character
%in the identifier.
\point 2. \TEX/ 텍스트에서 \CEE/ 프로그램에서 사용되는 식별자를  언급할 때는 앞에서 배운
바와 같이 `\.{|identifier|}'를 이용한다. 그런데 그 방법 말고 `\.{\\\\\{identifier\}}'
로 하는 방법이 더 있다. 이 두 경우 출력은 같다. 그런데 후자의 경우는 \\{identifier}를 
색인에 넣지 않는다. 왜냐하면, 위와 같은 표기는 \.{CWEAVE}가 \CEE/ 모드를 다룰 때 고려해야
할 대상에서 제외되기 때문이다. 또한, 후자는 식별자에 밑줄이 사용될 때는 반드시
그 앞에 백슬래쉬를 넣어야 한다. 이는 전자와 같이 할 때는 \.{CWEAVE}가 다 알아서 해주기
때문에 필요없는 작업이다.

%\point 3. To get typewriter-like type, as when referring to `\.{CWEB}', you
%can use the `\.{\\.}' macro (e.g., `\.{\\.\{CWEB\}}'). In the argument to
%this macro you should insert an additional backslash before the symbols
%listed as `special string characters' in the index to \.{CWEAVE}, i.e.,
%before backslashes and dollar signs and the like.
%A `\.{\\\ }' here will result in the visible space symbol; to get an
%invisible space following a control sequence you can say `\.{\{\ \}}'.
%If the string is long, you can break it up into substrings that
%are separated by `\.{\\)}'\,; the latter gives a discretionary backslash
%if \TEX/ has to break a line here.
\point 3. `\.{CWEB}'과 같은 타자 글꼴을 이용하고자 한다면, 매크로 `\.{\\.}'를 이용하면
된다. (예를 들어, `\.{\\.\{CWEB\}}'). 이 매크로의 인자로 \.{CWEAVE}이 `특수 문자열'
이라고 알고 있는 리스트에 속하는 기호들 앞에는 백슬래시를 하나 더 넣어야 한다. 즉, 그 리스트에
속하는 기호로는 백슬래쉬나 달러 기호와 같은 기호가 있다. `\.{\\\ }'는 눈에 보이는 공백 문자를
나타낸다. \TEX/ 명령어 다음에 나오는 공백 문자를 앞의 경우와 달리 눈에 보이지 않게 하려면,
`\.{\{\ \}}' 라고 해야 한다. 만일 표현하고자 하는 문자열이 길다면, `\.{\\)}'\, 를
이용해서 그 문자열을 둘로 나눌 수 있는데, 이때 \TEX/이 그 문자열에서 줄 바꿈을 하고자 할 경우,
후자는 임의의 백슬래쉬를 나타낸다.

%\point 4. The three control sequences \.{\\pagewidth}, \.{\\pageheight},
%and \.{\\fullpageheight} can be redefined in the limbo section at the
%beginning of your \.{CWEB} file, to change the dimensions of each page.
%The default settings
%$$\lpile{\.{\\pagewidth=6.5in}\cr
%  \.{\\pageheight=8.7in}\cr
%  \.{\\fullpageheight=9in}\cr}$$
%were used to prepare this manual; \.{\\fullpageheight} is
%\.{\\pageheight} plus room for the additional heading and page numbers at
%the top of each page. If you change any of these quantities, you should
%call the macro \.{\\setpage} immediately after making the change.
\point 4. 출력 페이지의 크기를 변경하고자 한다면, 작성하는 \.{CWEB} 파일의 인림보에 다음과
같은 명령어들을 재정의 하면된다. \.{\\pagewidth}, \.{\\pageheight}, 
\.{\\fullpageheight}. 기본 값은 다음과 같다. 즉, 지금 보고 있는 바로 이 문서의 규격에
해당하는 값이다.
$$\lpile{\.{\\pagewidth=6.5in}\cr
  \.{\\pageheight=8.7in}\cr
  \.{\\fullpageheight=9in}\cr}$$
\smallskip\noindent
\.{\\fullpageheight}의 값은 \.{\\pageheight}의 값에 각 페이지의 상단에 있는 면주와
페이지 번호를 위한 공간을 더한 값이다. 위의 값들을 변경한 다음에는 곧바로 매크로
\.{\\setpage}를 반드시 호출해야 한다.

%\point 5. The \.{\\pageshift} macro defines an amount by which right-hand
%pages (i.e., odd-numbered pages) are shifted right with respect to
%left-hand (even-numbered) ones. By adjusting this amount you may be
%able to get two-sided output in which the page numbers line up on
%opposite sides of each sheet.
\point 5. 매크로 \.{\\pageshift}는 오른쪽 페이지들(즉, 페이지 번호가 홀수인 페이지들)이
왼쪽 페이지들(즉, 페이지 번호가 짝수인 페이지들)에 비해서 오른쪽으로 얼마만큼 이동되었는지를
나타내는 값이다. 이 값을 조정하면 양면 인쇄를 할 때에 보기 좋은 적절한 페이지들을 얻을 수 있다.

%\point 6. The \.{\\title} macro will appear at the top of each page
%in small caps; it is the job name unless redefined.
\point 6. 매크로 \.{\\title}은 이 문서의 제목을 나타내는데, 매 페이지 상단에 작은
대문자체로 나타난다. 그리고 이 매크로를 재정의 하지 않으면, 작성하고 있는 \.{CWEB} 파일의
이름이 나온다.

%\point 7. The first page usually is assigned page
%number 1. To start on page 16, with contents
%on page 15, say this: `\.{\\def\\contentspagenumber\{15\}}
%\.{\\pageno=\\contentspagenumber} \.{\\advance\\pageno by 1}'.
\point 7. 대개 첫 번째 페이지는 페이지 번호는 1로 시작한다. 그런데 첫 번째 페이지의 번호를
16으로 하고, 목차 페이지의 번호를 15로 하고 싶으면,
`\.{\\def\\contentspagenumber\{15\}}
\.{\\pageno=\\contentspagenumber} \.{\\advance\\pageno by 1}'
와 같이 하면 된다.

%\point 8. The macro \.{\\iftitle} will suppress the header line if it is
%defined by `\.{\\titletrue}'. The normal value is \.{\\titlefalse}
%except for the table of contents; thus, the contents
%page is usually unnumbered.
%
%Two macros are provided to give flexibility to the table of
%contents: \.{\\topofcontents} is invoked just before the contents
%info is read, and \.{\\botofcontents} is invoked just after.
%Here's a typical definition:
%$$\lpile{\.{\\def\\topofcontents\{\\null\\vfill}\cr
%  \.{ { }\\titlefalse \% include headline on the contents page}\cr
%  \.{ { }\\def\\rheader\{\\mainfont The \{\\tt CWEAVE\}{ }processor\\hfil\}}\cr
%  \.{ { }\\centerline\{\\titlefont The \{\\ttitlefont CWEAVE\}{ }processor\}}\cr
%  \.{ { }\\vskip 15pt \\centerline\{(Version 3.64)\}{ }\\vfill\}}\cr}$$
%Redefining \.{\\rheader}, which is the headline for right-hand pages,
%suffices in this case to put the desired information at the top of the
%contents page.
\point 8. 다른 문서를 보더라도 일반적으로, 제목 페이지에는 면주가 없다. 그러한 목적으로
\.{CWEB} 문서에서 어떤 페이지를 제목 페이지라고 지정하는 방법은 매크로 \.{\\iftitle}을
`\.{\\titletrue}'로 정의하는 것이다. \.{CWEB}의 모든 페이지의 \.{\\iftitle}의 값은
기본적으로 \.{\\titlefalse} 이다. 당연히 페이지 번호가 붙지 않는 타이틀 페이지는 예외로
\.{\\titletrue} 이다.

제목 페이지는 \.{\\topofcontents}, \.{\\botofcontents}라는 두 개의 매크로를
재정의해서 원하는 대로 수정할 수 있다. 매크로 \.{\\topofcontents}는 목차 정보가 로드되기
바로 직전에 호출되고, 매크로 \.{\\botofcontents}는 바로 다음에 호출된다.
위 두 매크로의 전형적 예는 다음과 같다.
$$\lpile{\.{\\def\\topofcontents\{\\null\\vfill}\cr
 \.{ { }\\titlefalse \% include headline on the contents page}\cr
 \.{ { }\\def\\rheader\{\\mainfont The \{\\tt CWEAVE\}{ }processor\\hfil\}}\cr
 \.{ { }\\centerline\{\\titlefont The \{\\ttitlefont CWEAVE\}{ }processor\}}\cr
 \.{ { }\\vskip 15pt \\centerline\{(Version 3.64)\}{ }\\vfill\}}\cr}$$
위와 같이 목차 페이지 상단에 원하는 정보를 나타낼 때는 오른쪽 페이지들의 면주를 결정하는
\.{\\rheader} 매크로를 재정의 하면 된다.

%\point 9. Data for the table of contents is written to a file that
%is read after the indexes have been \TEX/ed; there's one line of data
%for every starred section. The file \.{common.toc} might look like this:
%$$\lpile{\.{\\ZZ \{Introduction\}\{0\}\{1\}\{28\}\{\}}\cr
%  \.{\\ZZ \{The character set\}\{2\}\{5\}\{29\}\{\}}\cr}$$
%and so on. The \.{\\topofcontents} macro could
%redefine \.{\\ZZ} so that the information appears in any desired format.
%(See also point~19 below.)
\point 9. 목차를 만들기 위한 자료는 색인들이 \TEX/으로 처리된 후에 읽히는 파일에 기록된다.
그 파일의 한 줄 한 줄은 각각 목차에 표시될 별표 마디의 정보를 가지고 있다.
파일 \.{common.toc}는 예를 들어 다음과 같은 것들을 담고 있다.
$$\lpile{\.{\\ZZ \{Introduction\}\{0\}\{1\}\{28\}\{\}}\cr
  \.{\\ZZ \{The character set\}\{2\}\{5\}\{29\}\{\}}\cr}$$
매크로 \.{\\topofcontents}는 \.{\\ZZ}를 원하는 형태로 출력하기 위해서 재정의 할 수 있다.
(아래의 항목~19를 살펴보라.)

%\point 10. Sometimes it is necessary or desirable to divide the output of
%\.{CWEAVE} into subfiles that can be processed separately. For example,
%the listing of \TEX/ runs to more than 500 pages, and that is enough to
%exceed the capacity of many printing devices and/or their software.
%When an extremely large job isn't cut into smaller pieces, the entire
%process might be spoiled by a single error of some sort, making it
%necessary to start everything over.
\point 10. \.{CWEAVE}의 결과 파일인 \TEX/ 파일을, 개별적으로 처리 가능한 여러 개의 서브
파일로 나눠야 할 경우가 종종 있다. 예를 들어, \TEX/ 파일 소스가 500페이지가 넘어서,
컴퓨터의 메모리나 기타 시스템의 자원 제한 때문에 그것을 한 번에 컴파일하기에는 역부족인 경우가
있다. 크기가 엄청나게 큼에도 \TEX/ 파일을 나누지 않고 그냥 작업하다가는 \TEX/이 더는 도저히
작업을 진행할 수 없는 경우가 생긴다. 이러하였으면 처음으로 되돌아가서 그 파일을 여러 개의
작은 파일로 나누고 작업을 다시 시작해야 하는데, 이와 같은 삽질을 막기 위해서라도 파일을
나누어도 그 결과물은 하나의 파일에서 얻은 결과물과 같게 하는 방법을 알고 있어야 한다.

%Here's a safe way to break a woven file into three parts:
%Say the pieces are $\alpha$,
%$\beta$, and $\gamma$, where each piece begins with a starred section.
%All macros should be defined in the opening limbo section of $\alpha$,
%and copies of this \TEX/ code should be placed at the
%beginning of $\beta$ and of $\gamma$. In order to process the parts
%separately, we need to take care of two things: The starting page
%numbers of $\beta$ and $\gamma$ need to be set up properly, and
%the table of contents data from all three runs needs to be
%accumulated.
그 방법을 설명하기 위해 예를 들어 세 개의 파일 $\alpha$, $\beta$, $\gamma$로 나눈다고
해보자. 여기서 각 파일은 별표 마디로 시작한다. 문서의 형태와 모양 결정하는 모든 매크로는
$\alpha$의 인림보에 정의되어 있어야 하고, 그러한 \TEX/ 코드들은 $\beta$, $\gamma$ 파일을
첫 부분에 마찬가지로 복사되어 있어야 한다. 일단 각 파일을 개별적으로 처리하기 위해서 두 가지를
고려해야 한다. 우선은 $\beta$와 $\gamma$ 파일의 시작 페이지 번호를 적절히 정해야 하고,
그다음에 이 세 개의 파일로 최종적으로 만들어질 문서의 목차는 이 세 개의 파일이 가지고 있는
정보를 누적해서 모두 담고 있어야 한다.

%The \.{cwebmac} macros include two control sequences \.{\\contentsfile} and
%\.{\\readcontents} that facilitate the necessary processing.  We include
%`\.{\\def\\contentsfile\{cont1\}}' in the limbo section of $\alpha$, and
%we include `\.{\\def\\contentsfile\{cont2\}}' in the limbo section of
%$\beta$; this causes \TEX/ to write the contents data for $\alpha$ and $\beta$
%into \.{cont1.tex} and \.{cont2.tex}. Now in $\gamma$ we say
%$$\.{\\def\\readcontents\{\\input cont1 \\input cont2
%  \\input \\contentsfile\}}\,;$$
%this brings in the data from all three pieces, in the proper order.
\.{cwebmac} 파일에는 \.{\\contentsfile}과 \.{\\readcontents}라는 두 개의 매크로가
정의 되어 있는데, 이 매크로들은 위에서 바로 앞에 언급한 두 가지 고려사항을 해결하는데 필요한
매크로들이다. 우선 파일 $\alpha$의 in limbo에 `\.{\\def\\contentsfile\{cont1\}}'라고
입력하고, 파일 $\beta$의 인림보에는 `\.{\\def\\contentsfile\{cont2\}}'라고 입력한다.
이렇게 하면, \TEX/은 파일 $\alpha$와 $\beta$의 목차에 관한 내용을 파일 \.{cont1.tex}과
\.{cont2.tex}에 기록한다. 이제 마지막 파일인 $\gamma$ 파일에 다음과 같이 하면 그만이다.
$$\.{\\def\\readcontents\{\\input cont1 \\input cont2
  \\input \\contentsfile\}}\,;$$
위와 같이 하면, 세 파일의 모든 자료를 올바른 순서로 가져오게 된다.

%However, we still need to solve the page-numbering problem. One way to
%do it is to include the following in the limbo material for $\beta$:
%$$\lpile{\.{\\message\{Please type the last page number of part 1: \}}\cr
%  \.{\\read -1 to \\temp \\pageno=\\temp \\advance\\pageno by 1}\cr}$$
%Then you simply provide the necessary data when \TEX/ requests
%it; a similar construction is used at the beginning of $\gamma$.
하지만, 위와 같이 하더라도 아직도 페이지 번호 붙이기 문제는 해결되지 않는다. 이 문제를 해결하는
한 가지 방법은 아래와 같은 내용을 $\beta$ 파일의 인림보에 입력하는 것이다.
$$\lpile{\.{\\message\{Please type the last page number of part 1: \}}\cr
  \.{\\read -1 to \\temp \\pageno=\\temp \\advance\\pageno by 1}\cr}$$
그리고 나서 \TEX/ 컴파일을 할 때, \TEX/이 물어보는 질문에 간단히 대답만 하면 된다. 이와
비슷한 방법을 파일 $\gamma$의 시작부에도 하면 된다.

%This method can, of course, be used to divide a woven file into
%any number of pieces.
물론, 위와 같은 방법을 이용하면 하나의 파일을 세 개가 아닌 여러 개의 임의의 개수로 나눌 수
있다. 

%\point 11. Sometimes it is nice to include things in the index that are
%typeset in a special way. For example, we might want to have an
%index entry for `\TeX'. \.{CWEAVE} provides two simple ways to
%typeset an index entry (unless the entry is an identifier or a reserved word):
%`\.{@\^}' gives roman type, and `\.{@.}' gives typewriter type.
%But if we try to typeset `\TeX' in roman type by saying, e.g.,
%`\.{@\^\\TeX@>}', the backslash character gets in the way,
%and this entry wouldn't appear in the index with the T's.
\point 11. 때때로 색인에 들어가는 아이템을 일반적인 로마체나 타자글꼴로 표시하는 것이 아니라
특별한 방법으로 표시하는 것이 보기 좋을 때가 있다. 만약 색인에 넣고자 하는 아이템이 식별자나
예약어가 아니라면, 앞에서 \.{CWEB}의 명령어에 대해서 알아볼 때 설명한 명령어를 이용하면 된다.
`\.{@\^}'는 색인에 들어갈 아이템을 로마체로 만들고, `\.{@.}'는 타자체로 만든다. 예를 들어,
색인에 `\TEX/'을 넣고자 한다고 하자. `\TEX/'을 로마체로 색인에 넣고자 하여
`\.{@\^\\TeX@>}'라고 했다면, 백슬래쉬 문자가 방해가 되어 \TeX 은 색인 T에 나오지 않는다.

%The solution is to use the `\.{@:}' feature, declaring a macro that
%simply removes a sort key as follows:
%$$\.{\\def\\9\#1\{\}}$$
%Now you can say, e.g., `\.{@:TeX\}\{\\TeX@>}' in your \.{CWEB} file; \.{CWEAVE}
%puts it into the index alphabetically, based on the sort key, and
%produces the macro call `\.{\\9\{TeX\}\{\\TeX\}}' which will ensure that
%the sort key isn't printed.
해결책은 다음과 같은 방법으로 단순히 정렬키를 제거하는 매크로 `\.{@:}'의 특징을 이용하는
것이다.  
$$\.{\\def\\9\#1\{\}}$$
이제,  \.{CWEB} 파일 내에 `\.{@:TeX\}\{\\TeX@>}'라고 하면, \.{CWEAVE}는 이것을
정렬키에 기반을 둔 알파벳 순으로 색인에 넣을 것이다. 그리고 정렬키는 출력되지 않는 매크로 호출
`\.{\\9\{TeX\}\{\\TeX\}}'을 실행할 것이다.

%A similar idea can be used to insert hidden material into section
%names so that they are alphabetized in whatever way you might wish.
%Some people call these tricks ``special refinements''; others call
%them ``kludges.''
위와 비슷한 아이디어로 보이지 않는 객체를 마디 이름에 넣기 위해서 사용할 수 있다. 어떤 이들은
이러한 속임수를 ``special refinements''라고 부르고, 다른 사람들은 모두 ``kludges''라고
부른다.
 
%\point 12. The control sequence \.{\\secno} is set to the number of the
%section being typeset.
\point 12. \TEX/ 명령어 \.{\\secno}는 페이지 번호를 설정하는데 사용된다.

%\point 13. If you want to list only the sections that have changed,
%together with the index, put the command `\.{\\let\\maybe=\\iffalse}' in
%the limbo section before the first section of your \.{CWEB} file. It's
%customary to make this the first change in your change file.
\point 13. 변경된 마디들만을 색인과 함께 나열하고 싶으면, 인림보에
`\.{\\let\\maybe=\\iffalse}'와 같은 명령을 입력하라. 이와 같은 작업을
수정 파일의 서두에서 하는 것이 대부분이다.

%This feature has a \TeX nical limitation, however: You cannot use it
%together with control sequences like \.{\\proclaim} or \.{\\+} or
%\.{\\newcount} that plain \TeX\ has declared to be `\.{\\outer}',
%because \TeX\ refuses to skip silently over such control sequences.
%One way to work around this limitation is to say
%$$\.{\\fi \\let\\proclaim\\relax \\def\\proclaim\{...\} \\ifon}$$
%where \.{\\proclaim} is redefined to be the same as usual but without
%an \.{\\outer} qualification. (The \.{\\fi} here stops the conditional
%skipping, and the \.{\\ifon} turns it back on again.) Similarly,
%$$\.{\\fi \\newcount\\n \\ifon}$$
%is a safe way to use \.{\\newcount}. Plain \TeX\ already provides a
%non-outer macro \.{\\tabalign} that does the work of \.{\\+}; you can say
%$$\postdisplaypenalty=10000
%\.{\\fi \\let\\+\\tabalign \\ifon}$$
%if you prefer the shorter notation \.{\\+}.
그러나 이 특징은 \TeX nical한 한계를 가지고 있다. 위 
명령을 \.{\\proclaim} 또는 \.{\\+} 또는 
\.{\\newcount}와 같은 플레인 \TeX 에 `\.{\\outer}'로 정의된 명령어와는
함께 사용할 수 없다. 왜냐하면, \TeX 은 그러한 명령어들을 조용히
넘어가려고 하지 않기 때문이다. 이러한 한계를 이겨내는 방법은 다음과
같이 하는 것이다.
$$\.{\\fi \\let\\proclaim\\relax \\def\\proclaim\{...\} \\ifon}$$
여기서 \.{\\proclaim}은 원래의 그것과 같은 기능을 하지만 \.{\\outer}를
띄어버린 것으로 재정의된 것이다. (\.{\\fi}는 조건 분기를 멈추게
하는 것이고, \.{\\ifon}는 다시 그것을 가능하게 하는 것이다.) 비슷하게,
다음 문장은
$$\.{\\fi \\newcount\\n \\ifon}$$
\.{\\newcount}를 사용하는 안전한 방법이다. 플레인 \TeX 은 이미
비-outer 매크로  \.{\\tabalign}를 제공하는데, 이 매크로는 \.{\\+}와
같은 일을 하는 매크로이다. 그래서 아래와 같은 명령어를 이용하면,
$$\postdisplaypenalty=10000
\.{\\fi \\let\\+\\tabalign \\ifon}$$
\.{\\tabalign}보다 짧은 \.{\\+}를 이용할 수 있다.

%\point 14. To get output in languages other than English, redefine the
%macros \.{\\A}, \.{\\As}, \.{\\ATH}, \.{\\ET}, \.{\\ETs}, \.{\\Q},
%\.{\\Qs}, \.{\\U}, 
%\.{\\Us}, \.{\\ch}, \.{\\fin}, \.{\\con}, \.{\\today}, \.{\\datethis}, and
%\.{\\datecontentspage}. \.{CWEAVE} itself need not be changed.
\point 14. 영어가 아닌 다른 언어로 \.{CWEB} 프로그램을 작성할 때는
다음과 같은 매크로들을 재정의 하면 된다.
\.{\\A}, \.{\\ATH}, \.{\\As}, \.{\\ET}, \.{\\ETs}, \.{\\Q},
\.{\\Qs}, \.{\\U}, 
\.{\\Us}, \.{\\ch}, \.{\\fin}, \.{\\con}, \.{\\today}, \.{\\datethis},
\.{\\datecontentspage}. \.{CWEAVE}를 직접 수정할 필요가 없는 것이다.

%\point 15. Some output can be selectively suppressed with the macros
%\.{\\noatl}, \.{\\noinx}, \.{\\nosecs}, \.{\\nocon}.
\point
15. 매크로 \.{\\noatl}, \.{\\noinx}, \.{\\nosecs}, \.{\\nocon}들을
이용하면, 출력 문서에서 목차와 같은 일부분을 선택적으로 생략할 수
있다. 즉 문서의 크기가 매우 작아서 목차가 필요 없겠다고 판단되면
작성하는 \.{CWEB} 파일의 인림보에 \.{\\nocon}이라고 입력하면 된다.

%\point 16. All accents and special text symbols of plain \TEX/ format
%will work in \.{CWEB} documents just as they are described in
%Chapter~9 of {\sl The \TEX/book}, with one exception.
%The dot accent (normally \.{\\.}) must be typed \.{\\:} instead.
\point 16. 플레인 \TEX/ 포멧에 정의된 모든 악센트와 특수
기호들은 {\sl The \TEX/book\/}의 제9장에 설명된 대로 \.{CWEB}
문서에서도 그대로 동작한다. 단, 한가지 예외가 있다.
보통 \TEX/에서 \.{\\.}로 표시하는 닷 악센트(dot
accent)는 \.{CWEB}에서는 반드시 \.{\\:}로 입력해야 한다. \.{CWEB}은
이미 \.{\\.}을 다른 용도로 사용하고 있다.

%\point 17. Several commented-out lines in \.{cwebmac.tex} are suggestions
%that users may wish to adopt. For example, one such line inserts a blank page
%if you have a duplex printer. Appendices D, E, and F of the complete
%version of this manual are printed using a commented-out option that
%substitutes `$\gets$' for `$=$' in the program listings. Looking at those
%appendices might help you decide which format you like better.
\point 17. \.{cwebmac.tex} 파일을 잘 살펴보면 몇몇 매크로나
명령어들이 주석 처리되어 있는 것을 발견할 수 있는데, 당신의 취향에
따라서 그 주석들을 풀어서 그 매크로의 기능을 이용할 수 있다. 예를 들어, 주석 처리된
것들 중 어느 하나를 풀면 빈 페이지를 만들 수 있는데, 이는 duplex
프린터를 사용하고 있다면 유용한 기능이 된다. 이 사용자 설명서의 완전한
버전은 부록 D, E, F가 있는데, 그 부록의 소스 리스트에서는 대입
연산자로 `$=$' 대신에 `$\gets$'를 사용한다. 그 부록들을 보고 취향에
맞는 어떤 주석을 풀어야 할지 말지를 결정해서 사용하면 된다.

%\point 18. Andreas Scherer has contributed a macro called \.{\\pdfURL}
%with which one can say things like the following, anywhere in the \TeX\ parts
%or the \CEE/ comments of a \.{CWEB} file:
%$$\vbox{\halign{\.{#}\hfil\cr
%You can send email to
% \\pdfURL\{the author\}\{mailto:andreas.scherer@@pobox.com\}\cr
%or visit \\pdfURL\{his home page\}\{http://www.pobox.com/%
%       \\TILDE/scherer\}.\cr}}$$
%In a {\mc PDF} document, the first argument will appear in blue as
%clickable text; the Acrobat reader, if correctly configured, will then
%redirect those links to the user's browser and open either the
%email client or the HTML viewer. In a hardcopy document, both arguments
%will be printed ({\tt the second in parentheses and typewriter type}).
%Certain special characters in an Internet address need to be handled
%in a somewhat awkward way, so that \.{CWEAVE} and/or \TeX\
%will not confuse them with formatting controls: Use \.{@@} for \.@
%and \.{\\TILDE/} for \.\~ and \.{\\UNDER/} for \.\_.
\point 18. Andreas Scherer는 \.{\\pdfURL}라는 매크로를 작성해 주었는데, 이 매크로는
\.{CWEB} 파일의 \TEX/ 부분 또는 \CEE/ 주석문의 어느 곳에서도 다음과 같은 방법으로
사용할 수 있다.
$$\vbox{\halign{\.{#}\hfil\cr
\hfontsize{at 9pt}다음의 주소
 \\pdfURL\{the author\}\{mailto:andreas.scherer@@pobox.com\}로
이메일을 보내거나,\cr
\hfontsize{at 9pt}웹사이트 \\pdfURL\{his home page\}\{http://www.pobox.com/%
       \\TILDE/scherer\}를 방문하길 바란다.\cr}}$$
{\mc PDF} 문서에서, 이 매크로의 첫 번째 인자는 클릭할 수 있는 파란색 글씨로 나타난다.
이 매크로를 올바르게 사용했다면, Acrobat reader는 사용자의
웹브라우저나 이메일 클라이언트를 통해서 링크된 쪽으로 리다이렉트
시켜서 해당 페이지를 연다. {\mc PDF}가 아닌 하드카피에서 이 매크로를
사용하면, 문서에는 두 인자가 동시에 출력된다.
({\tt the second in parentheses and typewriter type}).
인터넷 주소에 사용되는 일부 특별한 문자는 다소 이상한 방법으로
사용되어야 하는데, 이는 \.{CWEAVE} 와/또는 \TEX/의 명령어와 혼동하지
않기 위함이다. \.@를 사용할 자리에는 \.{@@}를 쓰고, \.\~ 대신에 
\.{\\TILDE/}를 써야하고, \.\_ 대신에 \.{\\UNDER/}를 사용해야 한다.

%\point 19. {\mc PDF} documents contain bookmarks that list all the major group
%titles in the table of contents, some of which will be subsidiary to
%others if the depth feature of \.{@*} has been used.
%Such bookmark entries are also known as ``outlines.''
%Moreover, the final group title, `Names of the sections',
%can be opened up to list every section name; Acrobat users can therefore
%navigate easily to any desired section.
\point 19. {\mc PDF} 문서는 목차에 사용된 모든 별표 마디의 제목에 대한
책갈피를 가지고 있고, 명령어\.{@*}의 깊이 특성을 이용하면
앞에서 명령어를 공부할 때 살펴본 바와 같이, 어떤 것은 다른 것들의
소제목이 되기도 한다. 그러한 책갈피 요소들을 
``outlines''라고 한다. 책갈피의 마지막 그룹 제목은 `Names of the
sections'인데, 이 그룹은 클릭하면 모두 펼쳐져서 프로그램에 사용된 모든
마디 이름을 볼 수 있다. 따라서 Acrobat 사용자는 이 기능을 통해서
원하는 마디를 쉽게 찾아 볼 수 있다. 

%The macros of \.{cwebmac.tex} are careful to ``sanitize'' all the names
%that appear as bookmarks, by removing special characters and
%formatting codes that are inappropriate for the limited typographic
%capabilities of {\mc PDF} outlines. For example, one section of \.{CWEAVE}
%is named `Cases for \\{case\_like}', which is represented by the \TeX\ code
%`\.{Cases} \.{for} \.{\\PB\{\\\\\{case\\\_like\}\}}' in \.{cweave.tex}; its
%sanitized name is simply `\.{Cases} \.{for} \.{case\_like}'.
%(When \.{.pdf} files are produced, the fifth parameter of every \.{\\ZZ} in
%the \.{.toc} file is set to the sanitized form of the first parameter;
%see point~9 above and point~20 below.)
\.{cwebmac.tex}의 매크로들은 책갈피에
사용된 모든 이름들에서 {\mc PDF}의 개요의 제한된 기능 때문에 개요에 사용되기에는 부적합한
특수 문자나 명령어나 매크로 같은 쓸데 없는 코드를 조심스럽게 순화하기 위해서 ``제거한다.''
예를 들어, \.{CWEAVE}의 어떤 마디 이름이 `Cases for \\{case\_like}'인데, 이는 \TEX/
코드로 \.{cweave.tex} 파일에서 `\.{Cases} \.{for}
\.{\\PB\{\\\\\{case\\\_like\}\}}'로 나타나는데, 이 이름을 개요를 위해서 방해되는 것을
제거하여 순화하면 단순히 `\.{Cases} \.{for} \.{case\_like}'가 된다. 
(\.{.pdf} 파일들이 만들어질 때, \.{.toc} 파일 안에 있는 모든 \.{\\ZZ}
매크로의 다섯 번째 인자는 첫 번째 파라미터의 순화된 버전이다. 앞서 나온
항목~9와 뒤에 나올 항목~20을 참고하라.)

%In general, sanitization removes \TeX\ control sequences and braces,
%except for control sequences defined by \.{CWEB} itself. Such a translation
%works most of the time, but you can override the defaults and obtain any
%translation that you want by using \TeX nical tricks. For example, after
%$$\.{\\sanitizecommand\\foo\{bar\}}$$
%the control sequence \.{\\foo} will sanitize to `\.{bar}'. And after
%$$\.{\\def\\kluj\#1\\\\\{foo\}}$$
%the \TeX\ code `\.{\\kluj bar\\\\}' will print as `foo' but sanitize
%to `\.{bar}', because the control sequences \.{\\kluj} and
%\.{\\\\} are removed by sanitization.
일반적으로, 개요을 위한 순화는 \.{CWEB}가 직접 정의한 명령어들을 제외한 \TEX/의 명령어와
괄호들을 제거한다. 대부분 경우 이러한 제거 과정은 순조롭게 이루어지지만, \TeX nical한
속임수를 사용하면, 당신은 이러한 기본 제거 과정을 당신이 원하는 당신이 정한 규칙대로 제거할
수 있다. 예를 들어, 아래와 같은 문장이 적용되면,
$$\.{\\sanitizecommand\\foo\{bar\}}$$
명령어 \.{\\foo}는 `\.{bar}'로 바뀐다. 그리고 다음과 같은 문장이 적용되면, 
$$\.{\\def\\kluj\#1\\\\\{foo\}}$$
\TeX\ 코드 `\.{\\kluj bar\\\\}'는 `foo'로 출력되지만 책갈피에는 
`\.{bar}'로 출력된다. 왜냐하면, 명령어 \.{\\kluj}와 
\.{\\\\}는 제거 과정에서 사라지기 때문이다.

%\point 20. Furthermore, group titles can be converted to an arbitrary
%sanitized text while also changing their form in running headlines, by
%using \.{\\ifheader}. Consider, for example, a \.{CWEB} source file that
%begins with the two lines
%$$\lpile{\.{\\def\\klujj\#1\\\\\{\\ifheader FOO\\else foo\\fi\}}\cr
%\.{@*Chinese \\klujj bar\\.}\cr}$$
%This coding introduces a major group entitled `{\bf Chinese foo}', with
%running headline `{\eightrm CHINESE FOO}' and table-of-contents entry
%`Chinese foo'. The corresponding bookmark is, however, `\.{Chinese} \.{bar}'.
%And the corresponding \.{.toc} file entry is
%`\.{\\ZZ \{Chinese \\klujj bar\\\\\}\{1\}\{1\}\{1\}\{Chinese bar\}}'.
\point 20. 게다가, 그룹 제목들은 \.{\\ifheader} 매크로를 사용해서
면주에서도 바뀌어야 한다면 임의의 순화된 텍스트로 바뀔 수 있다.
예를 들어, 아래와 같은 두 줄로 시작하는 \.{CWEB} 소스 파일을 생각해보자.
$$\lpile{\.{\\def\\klujj\#1\\\\\{\\ifheader FOO\\else foo\\fi\}}\cr
\.{@*Chinese \\klujj bar\\.}\cr}$$
이 코드는 면주에는 `{\eightrm CHINESE FOO}'로 나타나고 목차에는
`Chinese foo'로 나타나는 그룹 제목이 `{\bf Chinese foo}'를 만든다.
그러나 이에 해당하는 책갈피는 `\.{Chinese} \.{bar}'가 된다. 그리고
해당하는 \.{.toc} 파일에 나오는 항목은 
`\.{\\ZZ \{Chinese \\klujj bar\\\\\}\{1\}\{1\}\{1\}\{Chinese bar\}}'이 된다.

\vfill\end

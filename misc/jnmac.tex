% Macros for jangNah

\input kotex.sty

\catcode`@=11 % borrow the private macros of PLAIN (with care)

\def\rmhangul{\hangulfont="Noto Serif CJK KR Regular" at 10pt}
\def\bfhangul{\hangulfont="Noto Serif CJK KR Bold" at 10pt}
\def\tthangul{\hangulfont="Noto Sans Mono CJK kr Regular" at 10pt}
\def\slhangul{\hangulfont=UnBatang at 10pt}
\def\ithangul{\hangulfont=UnBatang at 10pt}

\font\tentex=cmtex10

\font\inchhigh=cminch
\font\titlefont=cmssdc10 at 40pt

\font\ninerm=cmr9
\font\eightrm=cmr8
\font\sixrm=cmr6
\def\ninermhangul{\hangulfont="Noto Serif CJK KR Regular" at 9pt}
\def\eightrmhangul{\hangulfont="Noto Serif CJK KR Regular" at 8pt}
\def\sixrmhangul{\hangulfont="Noto Serif CJK KR Regular" at 6pt}

\font\ninei=cmmi9
\font\eighti=cmmi8
\font\sixi=cmmi6
\skewchar\ninei='177 \skewchar\eighti='177 \skewchar\sixi='177

\font\ninesy=cmsy9
\font\eightsy=cmsy8
\font\sixsy=cmsy6
\skewchar\ninesy='60 \skewchar\eightsy='60 \skewchar\sixsy='60

\font\eightss=cmssq8

\font\eightssi=cmssqi8

\font\ninebf=cmbx9
\font\eightbf=cmbx8
\font\sixbf=cmbx6
\def\ninebfhangul{\hangulfont="Noto Serif CJK KR Bold" at 9pt}
\def\eightbfhangul{\hangulfont="Noto Serif CJK KR Bold" at 8pt}
\def\sixbfhangul{\hangulfont="Noto Serif CJK KR Bold" at 6pt}

\font\ninett=cmtt9
\font\eighttt=cmtt8

\hyphenchar\tentt=-1 % inhibit hyphenation in typewriter type
\hyphenchar\ninett=-1
\hyphenchar\eighttt=-1

\def\ninetthangul{\hangulfont="Noto Sans Mono CJK KR Regular" at 9pt}
\def\eighttthangul{\hangulfont="Noto Sans Mono CJK KR Regular" at 8pt}

\font\ninesl=cmsl9
\font\eightsl=cmsl8
\def\nineslhangul{\hangulfont=UnBatang at 9pt}
\def\eightslhangul{\hangulfont=UnBatang at 8pt}

\font\nineit=cmti9
\font\eightit=cmti8
\def\nineithangul{\hangulfont=UnBatang at 9pt}
\def\eightithangul{\hangulfont=UnBatang at 8pt}

\font\tenu=cmu10 % unslanted text italic
\font\magnifiedfiverm=cmr5 at 10pt
\font\manual=manfnt % font used for the METAFONT logo, etc.
\font\cmman=cmman % font used for miscellaneous Computer Modern variations

\newskip\ttglue
\def\tenpoint{\def\rm{\fam0\tenrm\rmhangul}%
  \textfont0=\tenrm \scriptfont0=\sevenrm \scriptscriptfont0=\fiverm
  \textfont1=\teni \scriptfont1=\seveni \scriptscriptfont1=\fivei
  \textfont2=\tensy \scriptfont2=\sevensy \scriptscriptfont2=\fivesy
  \textfont3=\tenex \scriptfont3=\tenex \scriptscriptfont3=\tenex
  \def\it{\fam\itfam\tenit\ithangul}%
  \textfont\itfam=\tenit
  \def\sl{\fam\slfam\tensl\slhangul}%
  \textfont\slfam=\tensl
  \def\bf{\fam\bffam\tenbf\bfhangul}%
  \textfont\bffam=\tenbf \scriptfont\bffam=\sevenbf
   \scriptscriptfont\bffam=\fivebf
  \def\tt{\fam\ttfam\tentt\tthangul}%
  \textfont\ttfam=\tentt
  \tt \ttglue=.5em plus.25em minus.15em
  \normalbaselineskip=14pt
  \def\MF{{\manual META}\-{\manual FONT}}%
  \let\sc=\eightrm
  \let\big=\tenbig
  \setbox\strutbox=\hbox{\vrule height8.5pt depth3.5pt width\z@}%
  \normalbaselines\rm}

\def\ninepoint{\def\rm{\fam0\ninerm\ninermhangul}%
  \textfont0=\ninerm \scriptfont0=\sixrm \scriptscriptfont0=\fiverm
  \textfont1=\ninei \scriptfont1=\sixi \scriptscriptfont1=\fivei
  \textfont2=\ninesy \scriptfont2=\sixsy \scriptscriptfont2=\fivesy
  \textfont3=\tenex \scriptfont3=\tenex \scriptscriptfont3=\tenex
  \def\it{\fam\itfam\nineit\nineithangul}%
  \textfont\itfam=\nineit
  \def\sl{\fam\slfam\ninesl\nineslhangul}%
  \textfont\slfam=\ninesl
  \def\bf{\fam\bffam\ninebf\ninebfhangul}%
  \textfont\bffam=\ninebf \scriptfont\bffam=\sixbf
   \scriptscriptfont\bffam=\fivebf
  \def\tt{\fam\ttfam\ninett\ninetthangul}%
  \textfont\ttfam=\ninett
  \tt \ttglue=.5em plus.25em minus.15em
  \normalbaselineskip=13pt
  \def\MF{{\manual hijk}\-{\manual lmnj}}%
  \let\sc=\sevenrm
  \let\big=\ninebig
  \setbox\strutbox=\hbox{\vrule height8pt depth3pt width\z@}%
  \normalbaselines\rm}

\def\eightpoint{\def\rm{\fam0\eightrm\eightrmhangul}%
  \textfont0=\eightrm \scriptfont0=\sixrm \scriptscriptfont0=\fiverm
  \textfont1=\eighti \scriptfont1=\sixi \scriptscriptfont1=\fivei
  \textfont2=\eightsy \scriptfont2=\sixsy \scriptscriptfont2=\fivesy
  \textfont3=\tenex \scriptfont3=\tenex \scriptscriptfont3=\tenex
  \def\it{\fam\itfam\eightit\eightithangul}%
  \textfont\itfam=\eightit
  \def\sl{\fam\slfam\eightsl\eightslhangul}%
  \textfont\slfam=\eightsl
  \def\bf{\fam\bffam\eightbf\eightbfhangul}%
  \textfont\bffam=\eightbf \scriptfont\bffam=\sixbf
   \scriptscriptfont\bffam=\fivebf
  \def\tt{\fam\ttfam\eighttt\eighttthangul}%
  \textfont\ttfam=\eighttt
  \tt \ttglue=.5em plus.25em minus.15em
  \normalbaselineskip=11pt
  \def\MF{{\manual opqr}\-{\manual stuq}}%
  \let\sc=\sixrm
  \let\big=\eightbig
  \setbox\strutbox=\hbox{\vrule height7pt depth2pt width\z@}%
  \normalbaselines\rm}

\def\tenmath{\tenpoint\fam-1 } % use after $ in ninepoint sections
\def\tenbig#1{{\hbox{$\left#1\vbox to8.5pt{}\right.\n@space$}}}
\def\ninebig#1{{\hbox{$\textfont0=\tenrm\textfont2=\tensy
  \left#1\vbox to7.25pt{}\right.\n@space$}}}
\def\eightbig#1{{\hbox{$\textfont0=\ninerm\textfont2=\ninesy
  \left#1\vbox to6.5pt{}\right.\n@space$}}}

% Page layout
%\newdimen\pagewidth \newdimen\pageheight \newdimen\ruleht
%\hsize=29pc  \vsize=44pc  \maxdepth=2.2pt  \parindent=3pc
%\pagewidth=\hsize \pageheight=\vsize \ruleht=.5pt
\abovedisplayskip=6pt plus 3pt minus 1pt
\belowdisplayskip=6pt plus 3pt minus 1pt
\abovedisplayshortskip=0pt plus 3pt
\belowdisplayshortskip=4pt plus 3pt

%\newinsert\footins
\def\footnote#1{\edef\@sf{\spacefactor\the\spacefactor}#1\@sf
      \insert\footins\bgroup\eightpoint
      \interlinepenalty100 \let\par=\endgraf
        \leftskip=\z@skip \rightskip=\z@skip
        \splittopskip=10pt plus 1pt minus 1pt \floatingpenalty=20000
        \smallskip\item{#1}\bgroup\strut\aftergroup\@foot\let\next}
\skip\footins=12pt plus 2pt minus 4pt % space added when footnote is present
%\count\footins=1000 % footnote magnification factor (1 to 1)
\dimen\footins=30pc % maximum footnotes per page

\newinsert\margin
\dimen\margin=\maxdimen
%\count\margin=0 \skip\margin=0pt % marginal inserts take up no space

% Composition macros
\def\bull{\vrule height .9ex width .8ex depth -.1ex } % square bullet
\def\|{\leavevmode\hbox{\tt\char`\|}} % vertical line
\def\dn{\leavevmode\hbox{\tt\char'14}} % downward arrow
\def\up{\leavevmode\hbox{\tt\char'13}} % upward arrow
\def\]{\leavevmode\hbox{\tt\char`\ }} % visible space

\def\pt{\,{\rm pt}} % units of points, in math formulas
\def\em{\,{\rm em}} % units of ems, in math formulas
\def\<#1>{\leavevmode\hbox{$\langle$#1\/$\rangle$}} % syntactic quantity
\def\oct#1{\hbox{\rm\'{}\kern-.2em\it#1\/\kern.05em}} % octal constant
\def\hex#1{\hbox{\rm\H{}\tt#1}} % hexadecimal constant
\def\cstok#1{\leavevmode\thinspace\hbox{\vrule\vtop{\vbox{\hrule\kern1pt
        \hbox{\vphantom{\tt/}\thinspace{\tt#1}\thinspace}}
      \kern1pt\hrule}\vrule}\thinspace} % control sequence token

{\obeyspaces\gdef {\ }}
\def\parbreak{\hfil\break\indent\strut}
\def\stretch{\nobreak\hskip0pt plus2pt\relax}

% macros for non-centered displays
\outer\def\begindisplay{\obeylines\startdisplay}
{\obeylines\gdef\startdisplay#1
  {\catcode`\^^M=5$$#1\halign\bgroup\indent##\hfil&&\qquad##\hfil\cr}}
\outer\def\enddisplay{\crcr\egroup$$}

% (the following \begin...\end-type macros do not appear in Appendix E)
% macros for demonstrating math constructions
\outer\def\beginmathdemo{$$\advance\baselineskip by2pt
  \halign\bgroup\indent\hbox to 160pt{##\hfil}&$##$\hfil\cr\noalign{\vskip-2pt}}
\outer\def\begindisplaymathdemo {$$\advance\baselineskip by15pt
  \halign\bgroup\indent\hbox to 160pt{##\hfil}&$\displaystyle{##}$\hfil\cr
  \noalign{\vskip-15pt}}
\outer\def\beginlongmathdemo{$$\advance\baselineskip by2pt
  \halign\bgroup\indent\hbox to 210pt{##\hfil}&$##$\hfil\cr\noalign{\vskip-2pt}}
\outer\def\beginlongdisplaymathdemo {$$\advance\baselineskip by15pt
  \halign\bgroup\indent\hbox to 210pt{##\hfil}&$\displaystyle{##}$\hfil\cr
  \noalign{\vskip-15pt}}
\outer\def\endmathdemo{\egroup$$}

% macros for verbatim scanning
\chardef\other=12
\def\ttverbatim{\begingroup
  \catcode`\\=\other
  \catcode`\{=\other
  \catcode`\}=\other
  \catcode`\$=\other
  \catcode`\&=\other
  \catcode`\#=\other
  \catcode`\%=\other
  \catcode`\~=\other
  \catcode`\_=\other
  \catcode`\^=\other
  \obeyspaces \obeylines \tt}

\outer\def\begintt{$$\let\par=\endgraf \ttverbatim \parskip=\z@
  \catcode`\|=0 \rightskip-5pc \ttfinish}
{\catcode`\|=0 |catcode`|\=\other % | is temporary escape character
  |obeylines % end of line is active
  |gdef|ttfinish#1^^M#2\endtt{#1|vbox{#2}|endgroup$$}}

\catcode`\|=\active
{\obeylines \gdef|{\ttverbatim \spaceskip\ttglue \let^^M=\  \let|=\endgroup}}

% Indexing macros
\newif\ifproofmode
\proofmodetrue % this should be false when making camera-ready copy
\newwrite\inx
\immediate\openout\inx=index % file for index reminders
\newif\ifsilent
\def\specialhat{\ifmmode\def\next{^}\else\let\next=\beginxref\fi\next}
\def\beginxref{\futurelet\next\beginxrefswitch}
\def\beginxrefswitch{\ifx\next\specialhat\let\next=\silentxref
  \else\silentfalse\let\next=\xref\fi \next}
\catcode`\^=\active \let ^=\specialhat
\def\silentxref^{\silenttrue\xref}

\def\marginstyle{\vrule height6pt depth2pt width\z@ \sevenrm}

\chardef\bslash=`\\
\def\xref{\futurelet\next\xrefswitch}
\def\xrefswitch{\begingroup
  \ifx\next|\aftergroup\vxref % case 1 or 2, |arg| or |\arg|
  \else\ifx\next\<\aftergroup\anglexref % case 3, "\<arg>" means angle brackets
    \else\aftergroup\normalxref \fi\fi\endgroup} % case 0, "{arg}"
\def\vxref|{\catcode`\\=\active \futurelet\next\vxrefswitch}
\def\vxrefswitch#1|{\catcode`\\=0
  \ifx\next\empty\def\xreftype{2}%
    \def\next{{\tt\bslash\text}}% type 2, |\arg|
  \else\def\xreftype{1}\def\next{{\tt\text}}\fi % type 1, |arg|
  \edef\text{#1}\makexref}
{\catcode`\|=0 \catcode`\\=\active |gdef\{}}
\def\anglexref\<#1>{\def\xreftype{3}\def\text{#1}%
  \def\next{\<\text>}\makexref}
\def\normalxref#1{\def\xreftype{0}\def\text{#1}\let\next=\text\makexref}
\def\makexref{\ifproofmode\insert\margin{\hbox{\marginstyle\text}}%
   \xdef\writeit{\write\inx{\text\space!\xreftype\space
     \noexpand\number\pageno.}}\writeit
   \else\ifhmode\kern\z@\fi\fi
  \ifsilent\ignorespaces\else\next\fi}
% the \insert (which is done in proofmode only) suppresses hyphenation,
% so the \kern\z@ is put in to give the same effect in non-proofmode.


\tenpoint


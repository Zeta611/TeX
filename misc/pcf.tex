%
% http://mathworld.wolfram.com/PeriodicContinuedFraction.html

\documentclass{article}

\usepackage{luacode}

\pagenumbering{gobble}

\begin{luacode*}
local floor,sqrt = math.floor, math.sqrt
local concat = table.concat
local sformat = string.format

-- integer square root of n
function isqrt(n)
  return floor(sqrt(n))
end

-- is n square?
function isSqrt(n)
   local s = isqrt(n)
   return s*s == n
end

-- periodic continued fraction
function periodCF(n)
   local period = {}
   local a0 = isqrt(n)
   local m, d = a0, n-a0*a0
   if d == 0 then
      return a0
   end
   local m0, d0 = m, d

   repeat
      local a = floor((a0+m)/d)
      period[#period+1] = a
      m = d*a - m
      d = (n-m*m)/d
   until m==m0 and d==d0

   return sformat("[%d;\\overline{%s}]",a0,concat(period,','))
end
\end{luacode*}

\newcommand*{\periodCF}[2]{%
  \begin{tabular}{rl}
  \luaexec{
    for i=#1,#2 do
      if not isSqrt(i) then
        tex.sprint("$\\sqrt{"..i.."}$&$"..periodCF(i).."$\\\\")
      end
    end
  }%
  \end{tabular}
}

\begin{document}

\section*{Periodic continued fraction}
A continued fraction $[a_0;a_1,a_2,\ldots]$ is periodic
if the sequence eventually repeats,
i.e there exists some $m$, $n$ with $a_{m+i}=a_{n+i}$ for all $i\ge0.$
A continued fraction,
$[1;2,2,...],$ is periodic and turned out to be $\sqrt2.$

If $r > 1$ is a rational number that is not a perfect square, then
$$\sqrt r = [a_0;\overline{a_1,a_2,\ldots,a_2,a_1,2a_0}].$$

In particular, if $r$ is any non-square positive integer,
the regular continued fraction expansion of $\sqrt r$ contains
a repeating block of length $m$, in which the first
$m-1$ partial denominators form a palindromic string.

\bigskip

\hskip-3cm
\hbox{\hss
\periodCF{1}{35}~\periodCF{37}{68}~\periodCF{69}{99}
\hss}

\end{document}

